% Fonttiasetuksia
\usepackage[T1]{fontenc}
\usepackage[utf8]{inputenc}
% Document metadata
\begin{filecontents}[overwrite]{\jobname.xmpdata}
\Title{\mycourse\ \mytitle, \implementation}
\Author{\myauthor}
\Publisher{\myschool}
\Subject{A lecture diary for the Tampere Universiry course ''Proof Theory''.}
\end{filecontents}
% Tavutuksia
\usepackage[finnish, main=english]{babel}
% Parnneltu matikkapaketti
\usepackage{mathtools}
% Paperikokoasetuksia
\usepackage{geometry}
\usepackage{amssymb}
% Ei sisennettyjä kappaleita
\usepackage{parskip}
% Robustimpi komentosyntaksi
\usepackage{xparse}
% Kuvia
\usepackage{tikz}
\usetikzlibrary{arrows, automata}
% Taulukoita
\usepackage{tabularx}
\usepackage{booktabs}
% Fontteja
% Aktivoidaan itse harjoituspohjan aluksi
% komennolla \fontfamily{put}\selectfont
\usepackage{fourier}
\usepackage{fontspec}
\setmainfont{Heuristica}
\setmonofont{DejaVu Sans Mono}
% Kuva-, taulukko- sun muut tekstit
\usepackage{caption}
\captionsetup{labelfont=bf}
\usepackage{csquotes}
% Boldattua matematiikkaa
\usepackage{bm}
% Arkistomuotoinen PDF
\usepackage[a-2b,mathxmp]{pdfx}
  \hypersetup{hidelinks}
\usepackage{url}

% ----- Komentoja -----

\NewDocumentCommand{\Nset}{}{\mathbb{N}}
\NewDocumentCommand{\Zset}{}{\mathbb{Z}}
\NewDocumentCommand{\Qset}{}{\mathbb{Q}}
\NewDocumentCommand{\Rset}{}{\mathbb{R}}
\NewDocumentCommand{\Cset}{}{\mathbb{C}}
\NewDocumentCommand{\Pset}{}{\mathbb{P}}

