% Fonttiasetuksia
\usepackage[T1]{fontenc}
\usepackage[utf8]{inputenc}
% Document metadata
\begin{filecontents}[overwrite]{\jobname.xmpdata}
\Title{\mycourse\ \mytitle, \implementation}
\Author{\myauthor}
\Publisher{\myschool}
\Subject{A lecture diary for the Tampere Universiry course ''Proof Theory''.}
\end{filecontents}
% Tavutuksia
\usepackage[finnish, main=english]{babel}
% Parnneltu matikkapaketti
\usepackage{mathtools}
% Paperikokoasetuksia
\usepackage{geometry}
% Matematiikkasymboleita
\usepackage{amssymb}
% Vähemmän hienoja lauseympäristöjä
\usepackage{amsthm}
% Ei sisennettyjä kappaleita
\usepackage{parskip}
% Robustimpi komentosyntaksi
\usepackage{xparse}
% Kuvia
\usepackage{tikz}
\usetikzlibrary{arrows, automata}
% Taulukoita
\usepackage{tabularx}
\usepackage{booktabs}
% Fontteja
% Aktivoidaan itse harjoituspohjan aluksi
% komennolla \fontfamily{put}\selectfont
\usepackage{fourier}
\usepackage{fontspec}
\setmainfont{Heuristica}
\setmonofont{DejaVu Sans Mono}
% Kuva-, taulukko- sun muut tekstit
\usepackage{caption}
\captionsetup{labelfont=bf}
\usepackage{csquotes}
% Boldattua matematiikkaa
\usepackage{bm}
% Symboleita symboleiden päälle
\usepackage{scalerel, stackengine}
% Todistuspuita
\usepackage{ebproof}
% Arkistomuotoinen PDF
\usepackage[a-2b,mathxmp]{pdfx}
  \hypersetup{hidelinks}
\usepackage{url}

% ----- Komentoja -----

\NewDocumentCommand{\Nset}{}{\mathbb{N}}
\NewDocumentCommand{\Zset}{}{\mathbb{Z}}
\NewDocumentCommand{\Qset}{}{\mathbb{Q}}
\NewDocumentCommand{\Rset}{}{\mathbb{R}}
\NewDocumentCommand{\Cset}{}{\mathbb{C}}
\NewDocumentCommand{\Pset}{}{\mathbb{P}}
\NewDocumentCommand{\mult}{}{\times}
\NewDocumentCommand{\LK}{}{LK}
\NewDocumentCommand{\IK}{}{IK}
\NewDocumentCommand{\limplies}{}{\supset}
\NewDocumentCommand{\corr}{}{%
    \mathrel{%
        \stackon[1.5pt]{=}{%
            \stretchto{%
                \scalerel*[\widthof{=}]{\wedge}{\rule{1ex}{3ex}}%
            }{0.5ex}%
        }%
    }%
}
\NewDocumentCommand{\seq}{}{\longrightarrow}



% ----- Rajoittimia -----
\DeclarePairedDelimiter{\set}{\{}{\}}
\DeclarePairedDelimiter{\arcs}{(}{)}
\DeclarePairedDelimiter{\angles}{\langle}{\rangle}
% ----- Ympäristöjä -----
\theoremstyle{definition}
\newtheorem{example}{Example}[section]
\newtheorem{definition}[example]{Definition}
\newtheorem{theorem}[example]{Theorem}
\NewDocumentEnvironment{tabenv}{O{\textwidth}m}{%
    \medskip
    \begin{minipage}{#1}
    \centering
    \captionsetup{type=table}%
    \captionof{table}{#2}%
}{%
    \end{minipage}
    \medskip%
}
\NewDocumentEnvironment{figenv}{O{\textwidth}m}{%
    \medskip
    \begin{minipage}{#1}
    \dentering
    \captionsetup{type=figure}%
}{%
    \captionof{table}{#2}%
    \end{minipage}
    \medskip%
}

% ----- Numerointiasetuksia -----
\numberwithin{equation}{section}
\numberwithin{figure}{section}
\numberwithin{table}{section}
