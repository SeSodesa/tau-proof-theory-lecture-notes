\section{\texorpdfstring{The language of classical logic \LK}{The language of classical logic LK}}%
\label{sec:lang-of-classical-logic}

To be able to talk about any language, we must first define the set of symbols we are allowed to use.
This set is know as the \emph{alphabet}.

\begin{definition}[The alphabet of classical logic]%
    \label{def:classical alphabet}
    In the (formal) language of logic, valid strings consist of the following symbols:
    \begin{enumerate}
        \item\label{it:classical constants} Constants:
            \begin{enumerate}
                \item\label{it:classical constants individual}
                    Individual constants: \(k_j, j\in\Nset\cup\set 0\).
                \item\label{it:classical constants function}
                    Function constants with \(i\) argument places: \(f_j^i : i\in\Nset,j\in\Nset\cup\set0\).
                \item\label{it:classical constants predicates}
                    Predicate constants with \(i\) argument places: \(R_j^i : i, j\in\Nset\)
            \end{enumerate}
        \item\label{it:classical variables} Variables:
            \begin{enumerate}
                \item\label{it:classical variables free} Free variables \(a_j, j\in\Nset\cup\set0\),
                \item\label{it:classical variables bound} Bound variables \(x_j, j\in\Nset\cup\set0\)
            \end{enumerate}
        \item\label{it:classical logical symbols} The logical symbols \(\lnot\) (not), \(\land\) (and), \(\lor\) (or), \(\limplies\) (implies),
            \(\forall\) (for all) and \(\exists\). The first four are called \emph{propositional connectives}
            and the last two are so-called \emph{quantifiers}.
        \item\label{it:classical auciliary symbols} Delimiter symbols: parentheses \((\) and \()\), and the comma \(,\).
    \end{enumerate}
\end{definition}

\begin{definition}[Term]%
    \label{def:classical term}
    The \emph{terms} or \emph{words} of the language of classical logic are defined recursively or inductively as follows:
    \begin{enumerate}
        \item\label{it:classical term 1}
            Every individual constant \(k_j\) is a term.
        \item\label{it:classical term 2}
            Every free variable \(a_j\) is a term.
        \item\label{it:classical term 3}
            If \(f^i\) is a function constant with \(i\) argument places and \(t_1,\ldots,t_i\) are terms,
            then \(f\arcs{t_1,\ldots,t_i}\) is a term.
        \item\label{it:classical term4}
            Terms are defined only by the items~\ref{it:classical term 1}--\ref{it:classical term 3} of this definition.
    \end{enumerate}
\end{definition}

\begin{definition}[Formulae and their outermost logical symbols]%
    \label{def:formulae and outermost logical symbols}
    If \(R^i\) is a predicate constant with \(i\) argument places and \(t_1,\ldots,t_i\) are terms,
    then \(R\arcs{t_1,\ldots,t_i}\) is an \emph{atomic formula}.
    \emph{Formulae} and their \emph{outermost logical symbols} are defined recursively as follows:
    \begin{enumerate}
        \item\label{it:formula 1}
            Every atomic formula is a formula. It has no outermost logical symbol.
        \item\label{it:formula 2}
            If \(A\) and \(B\) are formulae, then \(\arcs{\lnot A}\), \(\arcs{A\land B}\),
            \(\arcs{A\lor B}\) and \(\arcs{A\limplies B}\) are formulae.
            Their outermost logical symbols are \(\lnot\), \(\land\),
            \(\lor\) and \(\limplies\), respectively.
        \item\label{it:formula 3}
            If \(A(a)\) is a formula, where \(a\) is a free variable and
            a bound variable \(x\notin A\),
            then \(\forall x A(x)\) and \(\exists x A(x)\) are formulae.
            Here \(A(x)\) refers to the expression obtained from \(A(a)\),
            by replacing every occurrence of \(a\) with \(x\), wherever \(a\)
            occurs in \(A\).
            The outermost logical symbols of these formulae are \(\forall\) and \(\exists\), respectively.
        \item\label{it:formula 4}
            Formulae are only those expressions obtained by applying
            the items~\ref{it:formula 1}--\ref{it:formula 3}
            of this definition.
    \end{enumerate}
\end{definition}

\begin{definition}[The alphabet of formulae]\label{def:alphabet of formulae}
    The capital letters \(A_j, B_j, C_j, \ldots\) with or without the subindices
    \(j \in \Nset\cup\set0\) denote formulae. The Greek capital letters
    \(\Gamma_j, \Delta_j, \Pi_j,\ldots\) with or without the subindices
    \(j\in\Nset\cup\set0\) denote (finite) \emph{sequences} of formulae.
\end{definition}

For example, \(\Gamma: A_1, A_2(a), \exists x A_3(x), \forall y A_4(y)\) is a finite sequence,
as is \(\Delta: B_1, \exists x B_2(x)\).


\subsection{\texorpdfstring{Sequents, inferences and proofs in \LK}{Sequents, inferences and proofs in LK}}

\begin{definition}[Sequents, antecedents and succedents]%
    \label{def:sequent}
    If \(\Gamma\) and \(\Delta\) are sequences of formulae,
    then the expression \(\Gamma\seq\Delta\) is the \emph{sequent} of the two sequences.
    Here \(\Gamma\) is the \emph{antecedent} and \(\Delta\) the \emph{succedent}
    of the sequent.
\end{definition}

The natural meaning of \(A_1, \ldots,A_m\seq B_1,\ldots,B_n\) is that
\begin{center}
''If \(A_1\) and \(\cdots\) and \(A_m\), then \(B_1\) or \(\cdots\) or \(B_n\)''.
\end{center}
Note that the symbol \(\seq\) does not correspond to \(\limplies\).

\begin{definition}[Axioms]%
    \label{def:axiom}
    Sequents of the form \(A\seq A\) are called \emph{initial sequents} or \emph{axioms}.
\end{definition}

\begin{definition}[Contradiction and provability]%
    \label{def:contradiction and provability}
    A sequent without a succedent such as \(A_1, \ldots,A_m\seq\)
    is used to denote that \(A_1, \ldots,A_m\) yields a \emph{contradiction}.
    A sequent without an antecedent such as \(\seq B_1,\ldots,B_m\) means that
    \(B_1,\ldots,B_m\) is a \emph{provable}, or that \(B_1,\ldots,B_m\) \emph{holds}.
    A sequent with neither an antecedent or a succedent, as in \(\seq\),
    means that \emph{There is a contradiction}.
\end{definition}

\begin{definition}[Inferences, lower and upper sequents]%
    \label{def:inference}
    An \emph{inference} is an expression
    \begin{center}
    \begin{prooftree}
        \hypo{S_1}\infer1{S}
    \end{prooftree}
    \quad
    or
    \quad
    \begin{prooftree}
        \hypo{S_1}\hypo{S_2}\infer2{S}
    \end{prooftree}
    \end{center}
    where \(S_1\) and \(S_2\) are \emph{upper sequents} or \emph{predecessors} and \(S\)
    is a \emph{lower sequent} or \emph{successor}.
\end{definition}

The meaning of this notation is that if \(S_1\) and \(S_2\) are given,
\(S\) can be inferred from them. This inference could consist of multiple
individual inferences in sequence.

\begin{definition}[Proofs in \LK]\label{def:proof in LK}
    In classical logic \LK, a \emph{proof} \(P\)
    is a \emph{tree} of sequents, where
    \begin{enumerate}
        \item
            the leaves or \emph{premises} are axioms \(A\seq A\),
        \item
            the root or \emph{conclusion} of \(P\) has no lower sequents (or successors) and
        \item
            for every other sequent \(\Gamma_i\seq\Delta_i\in P\)
            there exists a sequent \(\Gamma_j\seq\Delta_j\in P\), so that
            \begin{equation*}
            \begin{prooftree}
                \hypo{\Gamma_i\seq\Delta_i}
                \infer1{\Gamma_j\seq\Delta_j}
            \end{prooftree}\,.
            \end{equation*}
    \end{enumerate}
\end{definition}

\begin{definition}[Inference rules in \LK]%
\label{def:inference rules in LK}
The following \emph{left} and \emph{right inference rules} are allowed in classical logic.
\begin{enumerate}
    \item\label{it:structural rules} Structural rules:
        \begin{enumerate}
            \item\label{it:weakening} Weakening:
                \begin{equation}
                    \tag{weakening:left}
                    \label{eq:weakening:left}
                    \begin{prooftree}
                        \hypo{\Gamma\seq\Delta}
                        \infer1{D, \Gamma\seq\Delta}
                    \end{prooftree}
                \end{equation}
                \begin{equation}
                    \tag{weakening:right}
                    \label{eq:weakening:right}
                    \begin{prooftree}
                        \hypo{\Gamma\seq\Delta}
                        \infer1{\Gamma\seq\Delta, D}
                    \end{prooftree}
                \end{equation}
                Here \(D\) is called the \emph{weakening formula}.
            \item\label{it:contraction} Contraction:
                \begin{equation}
                    \tag{contraction:left}
                    \label{eq:contraction:left}
                    \begin{prooftree}
                        \hypo{D, D, \Gamma\seq\Delta}
                        \infer1{D, \Gamma\seq\Delta}
                    \end{prooftree}
                \end{equation}
                \begin{equation}
                    \tag{contraction:right}
                    \label{eq:contraction:right}
                    \begin{prooftree}
                        \hypo{\Gamma\seq\Delta, D, D}
                        \infer1{\Gamma\seq\Delta, D}
                    \end{prooftree}
                \end{equation}
            \item\label{it:exchange} Exchange:
                \begin{equation}
                    \tag{exchange:left}
                    \label{eq:exchange:left}
                    \begin{prooftree}
                        \hypo{\Gamma,C, D, \Pi\seq\Delta}
                        \infer1{\Gamma, D, C, \Pi\seq\Delta}
                    \end{prooftree}
                \end{equation}
                \begin{equation}
                    \tag{exchange:right}%
                    \label{eq:exchange:right}
                    \begin{prooftree}
                        \hypo{\Gamma\seq\Delta, C, D, \Lambda}
                        \infer1{\Gamma\seq\Delta, D, C, \Lambda}
                    \end{prooftree}
                \end{equation}
            \item\label{it:cut} Cut:
                \begin{equation}
                    \tag{cut}
                    \label{eq:cut}
                    \begin{prooftree}
                        \hypo{\Gamma\seq\Delta, D}
                        \hypo{D, \Pi\seq\Lambda}
                        \infer2{\Gamma, \Pi\seq\Delta, \Lambda}
                    \end{prooftree}
                \end{equation}
        \end{enumerate}
    \item\label{it:logical rules} Logical rules:
        \begin{enumerate}
            \item\label{it:negation} Negation:
                \begin{equation}
                    \tag{\(\lnot\):left}
                    \label{eq:lnot:left}
                    \begin{prooftree}
                        \hypo{\Gamma\seq\Delta, D}
                        \infer1{\lnot D, \Gamma\seq\Delta}
                    \end{prooftree}
                \end{equation}
                \begin{equation}
                    \tag{\(\lnot\):right}
                    \label{eq:lnot:right}
                    \begin{prooftree}
                        \hypo{D, \Gamma\seq\Delta}
                        \infer1{\Gamma\seq\Delta, \lnot D}
                    \end{prooftree}
                \end{equation}
                Here \(D\) and \(\lnot D\) are the \emph{auxiliary} and \emph{principal} formulae
                of this inference, respectively.
            \item\label{it:conjugation} Conjugation:
                \begin{equation}
                    \tag{\(\land\):left}
                    \label{eq:land:left}
                    \begin{prooftree}
                        \hypo{C, \Gamma\seq\Delta}
                        \infer1{C\land D, \Gamma\seq\Delta}
                    \end{prooftree}
                    \quad\text{and}\quad
                    \begin{prooftree}
                        \hypo{D,\Gamma\seq\Delta}
                        \infer1{C\land D, \Gamma\seq\Delta}
                    \end{prooftree}
                \end{equation}
                \begin{equation}
                    \tag{\(\land\):right}
                    \label{eq:land:right}
                    \begin{prooftree}
                        \hypo{\Gamma\seq\Delta, C}
                        \hypo{\Gamma\seq\Delta, D}
                        \infer2{\Gamma\seq\Delta, C\land D}
                    \end{prooftree}
                \end{equation}
                Here \(C\) and \(D\) are the \emph{auxiliary} formulae and \(C\land D\) the \emph{principal} formula
                of this inference.
            \item\label{it:disjunction} Disjunction:
                \begin{equation}
                    \tag{\(\lor\):left}
                    \label{eq:lor:left}
                    \begin{prooftree}
                        \hypo{C, \Gamma\seq\Delta}
                        \hypo{D, \Gamma\seq\Delta}
                        \infer2{C\lor D, \Gamma\seq\Delta}
                    \end{prooftree}
                \end{equation}
                \begin{equation}
                    \tag{\(\lor\):right}
                    \label{eq:lor:right}
                    \begin{prooftree}
                        \hypo{\Gamma\seq\Delta, C}
                        \infer1{\Gamma\seq\Delta, C\lor D}
                    \end{prooftree}
                    \quad\text{and}\quad
                    \begin{prooftree}
                        \hypo{\Gamma\seq\Delta, D}
                        \infer1{\Gamma\seq\Delta, C\lor D}
                    \end{prooftree}
                \end{equation}
                Here \(C\) and \(D\) are the \emph{auxiliary} formulae and \(C\lor D\) the \emph{principal} formula
                of this inference.
            \item\label{implication} Implication:
                \begin{equation}
                    \tag{\(\limplies\):left}
                    \label{eq:limplies:left}
                    \begin{prooftree}
                        \hypo{\Gamma\seq\Delta, C}
                        \hypo{D, \Pi\seq\Lambda}
                        \infer2{C\limplies D, \Gamma, \Pi\seq\Delta, \Lambda}
                    \end{prooftree}
                \end{equation}
                \begin{equation}
                    \tag{\(\limplies\):right}
                    \label{eq:limplies:right}
                    \begin{prooftree}
                        \hypo{C, \Gamma\seq\Delta, D}
                        \infer1{\Gamma\seq\Delta, C\limplies D}
                    \end{prooftree}
                \end{equation}
                Here \(C\) and \(D\) are the \emph{auxiliary} formulae and \(C\limplies D\) the \emph{principal} formula
                of this inference.
            \item\label{it:universality} Universality:
                \begin{equation}
                    \tag{\(\forall\):left}
                    \label{eq:forall:left}
                    \begin{prooftree}
                        \hypo{F(t), \Gamma\seq\Delta}
                        \infer1{\forall x F(x), \Gamma\seq\Delta}
                    \end{prooftree}
                \end{equation}
                \begin{equation}
                    \tag{\(\forall\):right}
                    \label{eq:forall:right}
                    \begin{prooftree}
                        \hypo{\Gamma\seq\Delta, F(a)}
                        \infer1{\Gamma\seq\Delta, \forall xF(x)}
                    \end{prooftree}
                \end{equation}
                Here \(t\) is an arbitrary term and \(a\) does not occur in the lower sequent.
                \(F(t)\) and \(F(a)\) are the \emph{auxiliary formulae}, whereas \(\forall xF(x)\)
                is the \emph{principal formula} of this inference.
            \item\label{it:existence} Existence:
                \begin{equation}
                    \tag{\(\exists\):left}
                    \label{eq:exists:left}
                    \begin{prooftree}
                        \hypo{F(a), \Gamma\seq\Delta}
                        \infer1{\exists{}x F(x), \Gamma\seq\Delta}
                    \end{prooftree}
                \end{equation}
                \begin{equation}
                    \tag{\(\exists\):right}
                    \label{eq:exists:right}
                    \begin{prooftree}
                        \hypo{\Gamma\seq\Delta, F(t)}
                        \infer1{\Gamma\seq\Delta, \exists{}xF(x)}
                    \end{prooftree}
                \end{equation}
                Again, here \(a\) does not occur in the lower sequent and \(t\) is an arbitrary term.
                \(F(t)\) and \(F(a)\) are the \emph{auxiliary formulae}, whereas \(\forall xF(x)\)
                is the \emph{principal formula} of this inference.
        \end{enumerate}
\end{enumerate}
\end{definition}

\paragraph{Note:}
Proof manipulation using inference rules always happens ''at the ends'' of sequents:
the first formula on the left of \(\seq\) and the last formula on the right of \(\seq\).

\begin{theorem}[Uniqueness of the axiom]%
    \label{the:uniqueness of the axiom}
    The sequent \(A\seq A\) is the only axiom.
\end{theorem}

\begin{proof}
    For any sequent \(\Gamma\seq\Lambda\), there exists a finite proof tree
    \begin{center}
    \begin{prooftree}
        \hypo{A\seq A}
        \infer1{\vdots}
        \hypo{B\seq B}
        \infer1{\vdots}
        \infer2{\Gamma_1\seq\Lambda_1}
        \hypo{C\seq C}
        \infer1{\vdots}
        \infer2{\vdots}
        \infer1{\Gamma\seq\Lambda}
    \end{prooftree}
    \end{center}
\end{proof}

If there is a tree whose root is \(\seq A\), then \(A\) is \LK-provable.
If a proof does not use the cut rule of definition~\ref{def:inference rules in LK}.\ref{it:cut},
then a proof is deemed \emph{cut free}.

\begin{definition}[Equivalence]\label{def:equivalence}
Two formulae \(A\) and \(B\) are \emph{equivalent},
if and only if
\begin{equation*}
\begin{prooftree}
    \hypo{\seq A\limplies B}
    \hypo{\seq B\limplies A}
    \infer2[\eqref{eq:land:right}]{\seq\arcs{A\limplies B}\land\arcs{B\limplies A}}
\end{prooftree}\,.
\end{equation*}
This is denoted with \(A\equiv B\).
\end{definition}

\subsection{Metalanguage versus object language}

In order to be able to discuss mathematics (or any other set of objects), we need to separate the ideas of
\emph{object language} and \emph{metalanguage}. Metalanguage is the natural language
that we use every day, whereas an object language consists of certain strings formed from
a certain alphabet, with a specific goal of describing a certain set of objects.
In this course, the object language is used to describe mathematical logic and
is formed from the definitions~\ref{def:classical alphabet}--\ref{def:inference}.
A hierarchical description of it is given in table~\ref{tab:language hierarchy}.

\begin{tabenv}{Object language hierarchy from highest to lowest. Objects higher in the hierarchy are constructed from objects that are lower in the hierarchy.}%
\label{tab:language hierarchy}
\small
\begin{tabular}{c|l|p{0.6\textwidth}}
    \toprule
    Object level& Object name & Examples\\
    \midrule
    6.  & Proofs    & {%
        \begin{prooftree}
            \hypo{A\seq{}A}
            \infer1{\vdots}
            \hypo{B\seq{}B}
            \infer1{\vdots}
            \infer2{\Gamma\seq\Delta}
        \end{prooftree}
        \qquad\qquad
        \begin{prooftree}
            \hypo{S_1}
            \ellipsis{}{}
            \hypo{S_2}
            \ellipsis{}{}
            \infer2{S_4}
            \hypo{S_3}
            \ellipsis{}{}
            \infer2{S}
        \end{prooftree}
    }\\
    \midrule
    5.  & Inferences & {
        \begin{prooftree}
            \hypo{S_1}
            \infer1{S}
        \end{prooftree}
        \qquad
        \begin{prooftree}
            \hypo{S_1}
            \hypo{S_2}
            \infer2{S}
        \end{prooftree}
    }\\
    \midrule
    4.  & Sequents  &   \(A_1,\ldots, A_m\seq B_1,\ldots,B_n\), \(\Gamma\seq\Delta\), \(\Pi\seq\Gamma\)\\
    \midrule
    3.  & Formulae  &   \(A, B, C,\ldots\), \(R\arcs{t_1,\ldots,t_n}\), \(a = b\), \(a + b = c\), \(0\mult a  = b \mult c\), \(a' = 0\),
                        \(\forall x\arcs{x + b = c}\), \(\exists x\arcs{x = a} \lor \lnot\exists z\arcs{z = b}\)\\
    \midrule
    2.  & Terms     &   \(0, \ldots\), \(a, b, c, \ldots\), \(0 + a\), \(a\mult b\), \(0'\),
i                       \(\arcs{a + b} \mult c\), \(t\), \(t_i\) \\
    \midrule
    1.  & Alphabet  &   \(0, \ldots\), \(a, b, c, \ldots\), \(x, y, z, \ldots\), \(+, \mult, ', \ldots\),
    \(\lnot, \land,\lor, \limplies, \forall, \exists, = \), \(R\arcs{,\ldots,}\), \((\), \()\)\\
    \bottomrule
\end{tabular}
\end{tabenv}

\subsection{\texorpdfstring{Examples of proofs in \LK}{Examples of proofs in LK}}

\begin{example}
    To prove \(\seq A \lor\lnot A\), we reason as follows:
\begin{equation*}
    \begin{prooftree}
        \hypo{A\seq A}
        \infer1[\eqref{eq:lnot:right}]{\seq A, \lnot A}
        \infer1[\eqref{eq:lor:right}]{\seq A, A\lor\lnot A}
        \infer1[\eqref{eq:exchange:right}]{\seq A\lor\lnot A, A}
        \infer1[\eqref{eq:lor:right}]{\seq A\lor\lnot A, A\lor\lnot A}
        \infer1[\eqref{eq:contraction:right}]{\seq A \lor\lnot A}
    \end{prooftree}
\end{equation*}
\end{example}

\begin{example}[Fully indicated axiom]\label{exa:fully indicated axiom}
    Fully indicated \(F(a)\) simply means that we may apply quantifier rules.
    To prove \(\seq\lnot\forall y\lnot F(y)\limplies\exists x F(x)\),
    we reason as follows:
\begin{equation*}
    \begin{prooftree}
        \hypo{F(a)\seq F(a)}
        \infer1[\eqref{eq:exists:right}]{F(a)\seq\exists x F(x)}
        \infer1[\eqref{eq:lnot:right}]{\seq\exists x F(x), \lnot F(a)}
        \infer1[\eqref{eq:forall:right}]{\seq\exists x F(x), \forall y \lnot F(y)}
        \infer1[\eqref{eq:lnot:left}]{\lnot\forall y\lnot F(y)\seq\exists x F(x)}
        \infer1[]{\seq\lnot\forall y\lnot F(y)\limplies\exists x F(x)}
    \end{prooftree}
\end{equation*}
\end{example}



\begin{example}[Exercise 2.5.1]\label{exa:2.5.1}
    To prove \(A\lor B\equiv \lnot\arcs{\lnot A\land\lnot B}\) we need to prove two things,
    according to the above note. First we show that
    \(\seq A\lor B\limplies\lnot\arcs{\lnot A\land\lnot B}\):

\begin{equation}\label{eq:equivalence example part 1}
    \begin{prooftree}
        % left branch
        \hypo{A\seq A}
        \infer1[\eqref{eq:lnot:left}]{\lnot A, A\seq}
        \infer1[\eqref{eq:land:left}]{\lnot A\land\lnot B, A\seq}
        \infer1[\eqref{eq:exchange:left}]{A, \lnot A\land\lnot B}
        % right branch
        \hypo{B\seq B}
        \infer1[\eqref{eq:lnot:left}]{\lnot B, B\seq}
        \infer1[\eqref{eq:land:left}]{\lnot A\land\lnot B, B\seq}
        \infer1[\eqref{eq:exchange:left}]{B, \lnot A\land\lnot B}
        % Trunk
        \infer2[\eqref{eq:lor:left}]{A\lor B, \lnot A \land\lnot B\seq}
        \infer1[\eqref{eq:exchange:left}]{\lnot A \land\lnot B, A\lor B\seq}
        \infer1[\eqref{eq:lnot:right}]{A\lor B \seq\lnot\arcs{\lnot A \land\lnot B}}
        \infer1[\eqref{eq:limplies:right}]{\seq A\lor B\limplies\lnot\arcs{\lnot A\land\lnot B}}
    \end{prooftree}
\end{equation}

Then we prove the statement \(\seq\lnot\arcs{\lnot A \land\lnot B} \limplies A\lor B\):

\begin{equation}\label{eq:equivalence example part 2}
    \begin{prooftree}
        \hypo{A\seq A}
        \infer1[\eqref{eq:lor:right}]{A\seq A\lor B}
        \infer1[\eqref{eq:lnot:right}]{\seq A\lor B, \lnot A}
        \hypo{B\seq B}
        \infer1[\eqref{eq:lor:right}]{B\seq A\lor B}
        \infer1[\eqref{eq:lnot:right}]{\seq A\lor B, \lnot B}
        \infer2[\eqref{eq:land:right}]{\seq A\lor B, \lnot A\land\lnot B}
        \infer1[\eqref{eq:lnot:left}]{\lnot\arcs{\lnot A \land\lnot B}\seq A\lor B}
        \infer1[\eqref{eq:limplies:right}]{\seq\lnot\arcs{\lnot A \land\lnot B}\limplies A\lor B}
    \end{prooftree}
\end{equation}

As the valid proofs~\ref{eq:equivalence example part 1} and \ref{eq:equivalence example part 2}
could be constructed, the claim holds. \qed
\end{example}

\begin{example}[Exercise 2.5.4]\label{exa:2.5.4}
    We wish to show that \(\lnot\forall y F(y)\equiv\exists x\lnot F(x)\).
    As per usual with equivalences, we need to do the proof in two parts.
    First we show that \(\seq\lnot\forall y F(y)\limplies\exists x\lnot F(x)\):

\begin{proof}
    \begin{equation}\label{exe:2.5.4:eq:1}
        \begin{prooftree}
            \hypo{F(a)\seq F(a)}
            \infer1[\eqref{eq:lnot:right}]{\seq F(a), \lnot F(a)}
            \infer1[\eqref{eq:exists:right}]{\seq F(a), \exists x\lnot F(x)}
            \infer1[\eqref{eq:exchange:right}]{\seq \exists x\lnot F(x), F(a)}
            \infer1[\eqref{eq:forall:right}]{\seq \exists x\lnot F(x), \forall y F(y)}
            \infer1[\eqref{eq:lnot:left}]{\lnot\forall y F(y)\seq \exists x\lnot F(x)}
            \infer1[\eqref{eq:limplies:right}]{\seq\lnot\forall y F(y)\limplies\exists x\lnot F(x)}
        \end{prooftree}
    \end{equation}
\end{proof}

    Then we prove the statement in the opposite direction,
    as in \(\seq\exists x\lnot F(x)\limplies\lnot\forall y F(y)\):

    \begin{proof}
        \begin{equation}\label{exe:2.5.4:eq:2}
            \begin{prooftree}
                \hypo{F(a)\seq F(a)}
                \infer1[\eqref{eq:forall:left}]{\forall y F(y)\seq F(a)}
                \infer1[\eqref{eq:lnot:right}]{\seq F(a), \lnot\forall y F(y)}
                \infer1[\eqref{eq:exchange:right}]{\seq\lnot\forall y F(y), F(a)}
                \infer1[\eqref{eq:lnot:left}]{\lnot F(a)\seq\lnot\forall yF(y)}
                \infer1[\eqref{eq:exists:left}]{\exists x\lnot F(x)\seq\lnot\forall yF(y)}
                \infer1[\eqref{eq:limplies:right}]{\seq\exists x\lnot F(x)\limplies\lnot\forall y F(y)}
            \end{prooftree}
        \end{equation}
    \end{proof}

    In light of proofs~\eqref{exe:2.5.4:eq:1} and \eqref{exe:2.5.4:eq:2}, the claim holds.
\end{example}

\begin{example}[Double negation in \LK]\label{exa:double negation in LK}
    We prove the law of double negation in \LK{} as follows:
\begin{equation*}
    \begin{prooftree}
        \hypo{A\seq A}
        \infer1[\eqref{eq:lnot:left}]{\lnot A, A\seq}
        \infer1[\eqref{eq:lnot:right}]{A\seq\lnot\lnot A}
        \infer1[\eqref{eq:limplies:right}]{\seq A\limplies\lnot\lnot A}
        \hypo{A\seq A}
        \infer1[\eqref{eq:lnot:right}]{\seq A, \lnot A}
        \infer1[\eqref{eq:lnot:left}]{\lnot\lnot A\seq A}
        \infer1[\eqref{eq:limplies:right}]{\seq\lnot\lnot A\limplies A}
        \infer2[\eqref{eq:land:right}]{\seq\arcs{A\limplies\lnot\lnot A}\land\arcs{\lnot\lnot A\limplies A}}
    \end{prooftree}
\end{equation*}
    Note that this also means that \(A\equiv A\).
    Note also that this proof is not possible in \LJ.
\end{example}

\begin{exercise}[2.7, A cut free proof]\label{exa:cut free proof}
    A cut free proof of \(\forall x A(x)\limplies B \seq \exists x \arcs{A(x)\limplies B}\),
    where \(A(a)\) and \(B\) are atomic and disjoint goes as follows:
    \begin{equation*}
        \begin{prooftree}
            \hypo{A(a)\seq A(a)}
            \infer1[\eqref{eq:weakening:right}]{A(a)\seq A(a), \forall x A(x)}
            \hypo{B\seq B}
            \infer2[\eqref{eq:limplies:left}]{A(a), \forall xA(x)\limplies B\seq A(a), B}
            \infer1[\eqref{eq:exchange:left}]{\forall xA(x)\limplies B, A(a)\seq, A(a), B}
            \infer1[\eqref{eq:limplies:right}]{\forall xA(x)\limplies B\seq A(a), A(a)\limplies B}
            \infer1[\eqref{eq:exists:right}]{\forall xA(x)\limplies B\seq A(a), \exists x\arcs{A(x)\limplies B}}
            \infer1[\eqref{eq:exchange:right}]{\forall xA(x)\limplies B\seq\exists x\arcs{A(x)\limplies B}, A(a)}
            \infer1[\eqref{eq:forall:right}]{\forall xA(x)\limplies B\seq\exists x\arcs{A(x)\limplies B}, \forall x A(x)}
            \hypo{B\seq B}
            \infer1[\eqref{eq:weakening:left}]{A(a), B\seq B}
            \infer1[\eqref{eq:limplies:right}]{B\seq A(a)\limplies B}
            \infer1[\eqref{eq:exists:right}]{B\seq\exists x\arcs{A(x)\limplies B}}
            \infer2[\eqref{eq:limplies:left}]{\forall x A(x)\limplies B, \forall x A(x)\limplies B \seq \exists x \arcs{A(x)\limplies B}, \exists x \arcs{A(x)\limplies B}}
            \infer1[\eqref{eq:contraction:left}]{\forall x A(x)\limplies B \seq \exists x \arcs{A(x)\limplies B}, \exists x \arcs{A(x)\limplies B}}
            \infer1[\eqref{eq:contraction:right}]{\forall x A(x)\limplies B \seq \exists x \arcs{A(x)\limplies B}}
            \infer1[\eqref{eq:limplies:right}]{\seq\arcs{\forall x A(x)\limplies B}\limplies\arcs{\exists x \arcs{A(x)\limplies B}}}
        \end{prooftree}
    \end{equation*}
\end{exercise}

\paragraph{Note:}
We also accept the inferences
\begin{equation*}
    \begin{prooftree}
        \hypo{\Gamma\seq\Delta}
        \infer1[\(\set{\eqref{eq:land:left},\eqref{eq:lor:left},\eqref{eq:limplies:left}}\)]{A\set{\land,\lor,\limplies } B, \Gamma\seq \Delta}
    \end{prooftree}
    \quad\text{and}\quad
    \begin{prooftree}
        \hypo{\Gamma\seq\Delta, A}
        \infer1[\eqref{eq:lor:right}]{\Gamma\seq\Delta, \arcs{B\land C}\lor A}
    \end{prooftree}
\end{equation*}

\begin{proposition}\label{prop:if provable, then atomic}
    If \(\Gamma\seq\Delta\) is provable without~\eqref{eq:cut},
    then we may assume that all of the related axioms are atomic,
    and that the proof is without cut.
\end{proposition}

\begin{proof}
    Omitted. See Takeuti's book~\cite[14-17]{Takeuti-1987} for the details.
\end{proof}

\begin{definition}[Alphabetical variants]\label{def:alphabetical variants}
    The formulae \(A\) and \(B\) are \emph{alphabetical variants}
    if they differ only in the names of their bound variables.
    This is denoted with \(A\varof{B}\). This is an equivalence relation.
\end{definition}

For example, it should be easily seen that
\begin{equation*}
    \exists x\arcs{f(x) = g(x)} \varof\exists y\arcs{f(y) = g(y)}\,.
\end{equation*}
It can be shown that if \(A\sim B\), then \(\seq\arcs{A\limplies B}\land\arcs{B\limplies A}\).
In other words, if \(A\varof B\) then \(A\equiv B\).



\subsection{Exercises}

\begin{exercise}[Exercise 2.5.2]\label{exe:2.5.2}
    Prove \(A\limplies B \equiv \lnot A\lor B\).
\end{exercise}

\begin{exercise}[Exercise 2.5.3]\label{exe:2.5.3}
    Prove \(\exists x F(x) \equiv\lnot\forall y\lnot F(y)\).
\end{exercise}
\begin{exercise}[Exercise 2.5.4]\label{exe:2.5.4}
    Prove \(\lnot\forall yF(y) \equiv\exists x\lnot F(x)\).
\end{exercise}
\begin{exercise}[Exercise 2.5.5]\label{exe:2.5.5}
    Prove \(\lnot\arcs{A\land B} \equiv \lnot A\lor\lnot B\).
\end{exercise}
\begin{exercise}[Exercise 2.6.1]\label{exe:2.6.1}
    Prove \(\exists x\arcs{A\limplies B(x)}\equiv A\limplies\exists xB(x)\).
\end{exercise}
\begin{exercise}[Exercise 2.6.2]\label{exe:2.6.2}
    Prove \(\exists x\arcs{A(x)\limplies B}\equiv\forall xA(x)\limplies B\),
    where \(B\) does not contain \(x\).
\end{exercise}
\begin{exercise}[Exercise 2.6.3]\label{exe:2.6.3}
    Prove \(\exists x\arcs{A(x)\limplies B(x)}\equiv\forall xA(x)\limplies\exists xB(x)\).
\end{exercise}
\begin{exercise}[Exercise 2.6.4]\label{exe:2.6.4}
    Prove \(\lnot A\limplies B\seq\lnot B\limplies A\)
\end{exercise}
\begin{exercise}[Exercise 2.6.5]\label{exe:2.6.5}
    Prove \(\lnot A\limplies\lnot B\seq B\limplies A\)
\end{exercise}
