\section{What is mathematics about?}%
\label{sec:what-is-math-about}

To answer the question set in the title of this chapter, \emph{proofs}.

\begin{example}[Proof by contradiction]
The infinite set of natural numbers \(\Nset = \set{1, 2, 3, \ldots}\) should already be familiar to you.
It is closed under addition \(+\) and multiplication \(\mult\).
In other words, if \(a, b\in\Nset\), then \(a + b \in\Nset\) and \(a\mult b\in\Nset\).
But what about division \(\div\)? We know that \(6 \div 2 = 3 \in\Nset\),
but \(3\div 6 = 1 / 2 \notin\Nset\).

It is known that some natural numbers are products of two \emph{other} natural numbers,
while others are not. For example, \(6 = 2 \mult 3\) and \(20 = 4 \mult 5\),
but
\[
2 = 2\mult 1,
\quad
3 = 3\mult 1,
\quad
5 = 5 \mult 1,
\quad
7 = 7 \mult 1,
\ldots\,.
\]
\end{example}
These numbers whose factors only consist of themselves and the number \(1\) are called \emph{prime numbers}
and denoted with \(\Pset\). How many prime numbers are there? We show that there are infinitely many via
\emph{proof by contradiction}.

Before we begin, let us analyze the products of prime numbers a bit more closely:
\begin{align*}
    2 \mult 3                   &= 6,   & 6 + 1     &= 7 \in\Pset \\
    2\mult 3 \mult 5            &= 30   & 30 + 1    &= 31 \in\Pset \\
    2 \mult 3 \mult 5 \mult 7   &= 210  & 210 + 1   &= 211\in\Pset\\
                                &\vdots &           &\vdots
\end{align*}
It seems like when the number \(1\) is added to any product of primes, the resulting number is prime.
We might also ask ourselves, whether the product of primes
\begin{equation*}\tag{hypothesis}
    \prod = 2 \mult 3 \mult 5 \mult\cdots\mult p_n + 1 \in\Pset \,.
\end{equation*}
for all \(n\in\Nset\).

To prove this hypothesis, a natural seeming approach is to then make the \emph{contrapositive assumption},
that if \(p_i, p_n\in\Pset\), then
\begin{equation}
    \prod = 2\mult 3 \mult 5 \mult \cdots \mult p_i\mult\cdots\mult p_n + 1 = p_i\mult c\notin\Pset
\end{equation}
for some \(c\in\Nset\). But if this were the case, then
\begin{equation*}
    \frac\prod {p_i} = c = \frac{2 \mult 3 \mult 5 \mult \cdots \mult p_n + 1}{p_i} \notin\Nset\,.
\end{equation*}
This is in contradiction with our assumptions, which proves that there are infinitely many primes
(as the hypothesis provides us with a way of constructing them indefinitely).

\begin{example}[The derivative of a quadratic polynomial]
    See \url{https://www.youtube.com/watch?v=OhmexZXPVkE}.
\end{example}

\begin{example}[There exist non-rational real numbers]
    A rational number \(q\in\Qset\) is a real number,
    such that \(q = m\div n\) for some \(m,n\in\Zset\), where \(n\neq 0\).
    Real numbers \(r\in\Rset\) that are not rational are irrational,
    which can be stated bluntly as
    \begin{equation*}
        \nexists m, n \in \Zset, n\neq 0 : r = \frac m n : r \in \Rset - \Qset\,.
    \end{equation*}
    For example \(\sqrt 2\notin\Qset\).
\end{example}

\begin{theorem}\label{the:a to b in Q}
    There are \(a,b\notin\Qset\), so that \(a^b\in\Qset\).
\end{theorem}

\begin{proof}
    Let \(a, b = \sqrt 2\). If \(\sqrt 2 ^{\sqrt 2} \in\Qset\), then we are done.
    If this is not the case, let \(a = \sqrt2^{\sqrt 2}\) and \(b = \sqrt 2\).
    Then
    \begin{equation*}
        a^b = \arcs*{\sqrt2^{\sqrt 2}}^{\sqrt 2} = \sqrt 2 ^ 2 = 2 \in\Qset\,.
    \end{equation*}
    This concludes the proof.
\end{proof}

The proof of Theorem~\ref{the:a to b in Q} is an example of a \emph{non-constructive proof},
which are not allowed in \emph{intuitionistic} or \emph{constructive logic} \LJ.

Classical logic \LK{} only concerns itself with the notions of ''true'' and ''false''.
In other words, classical logic functions based on the following principle:
\begin{quote}
    ''Any well-formed mathematical object \(A\) is either true or false;
    if \(A\) is false, then not-\(A\) is true.''
\end{quote}
\emph{Tertium non datur}, ''\(A\) or not \(A\)'' is always true.
In intuitionistic logic \LJ, this is not the case.
It imposes the additional constraint that for an object \(A\)
to exist in the first place, one has to be able to \emph{construct}
such an \(A\).

All of logic is based on an abstract language, that we now study.
