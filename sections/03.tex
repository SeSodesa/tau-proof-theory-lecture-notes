\section{\texorpdfstring{Intuitionistic predicate calculus \LJ}{Intuitionistic predicate calculus LJ}}%
\label{sec:intuitionistic-predicate-calculus}

Intuitionistic logic \LJ{} is very similar to classical logic \LK,
except that in its inference rules, the succedents \(\Delta\) of upper sequents \(\Gamma\seq\Delta\)
may only contain \emph{at most} a single formula. For example, the inferences
\begin{equation*}
    \begin{prooftree}
        \hypo{\Gamma\seq D, D}
        \infer1[\eqref{eq:contraction:right}]{\Gamma\seq D}
    \end{prooftree}
    \quad\text{and}\quad
    \begin{prooftree}
        \hypo{\Gamma\seq C, D}
        \infer1[\eqref{eq:exchange:right}]{\Gamma\seq D, C}
    \end{prooftree}
\end{equation*}
are \emph{not} allowed in \LJ{}. This of course means, that \LJ{} is a true subset of \LK:
some proofs~\(P\) that are in \LK{} are not in \LJ{}.

\subsection{Examples}

\begin{example}[A proof in \LJ]\label{exa:a proof in LJ}
    \begin{equation*}
        \begin{prooftree}
            \hypo{A\seq A}
            \infer1[\eqref{eq:land:left}]{A\land\lnot A&\seq A}
            \infer1[\eqref{eq:lnot:left}]{\lnot A, A\land\lnot A&\seq}
            \infer1[\eqref{eq:land:left}]{A\land\lnot A, A\land\lnot A&\seq}
            \infer1[\eqref{eq:contraction:left}]{A\land\lnot A&\seq}
            \infer1[\eqref{eq:lnot:right}]{&\seq \lnot\arcs{A\land\lnot A}}
        \end{prooftree}
    \end{equation*}
\end{example}

\begin{example}[Another proof in \LJ]\label{exa:another proof in LJ}
    Prove that \(\lnot\exists xF(x)\seq\forall yF(y)\).
    \begin{equation*}
        \begin{prooftree}
            \hypo{F(a)&\seq F(a)}
            \infer1[\eqref{eq:exists:right}]{F(a)&\seq\exists xF(x)}
            \infer1[\eqref{eq:lnot:left}]{\lnot\exists xF(x), F(a)&\seq}
            \infer1[\eqref{eq:exchange:left}]{F(a), \lnot\exists xF(x)&\seq}
            \infer1[\eqref{eq:lnot:right}]{\lnot\exists xF(x)&\seq \lnot F(a)}
            \infer1[\eqref{eq:forall:right}]{\lnot\exists xF(x)&\seq \forall x\lnot F(x)}
            \infer1[\eqref{eq:limplies:right}]{&\seq \lnot\exists xF(x)\limplies\forall x\lnot F(x)}
        \end{prooftree}
    \end{equation*}
\end{example}

\subsection{Exercises}

\begin{exercise}[3.9.1]\label{exe:3.9.1}
    Prove in \LJ{:} \(\lnot A\lor B\seq A\limplies B\).
\end{exercise}
\begin{exercise}[3.9.2]\label{exe:3.9.2}
    Prove in \LJ{:} \(\exists xF(x)\seq\lnot\forall y\lnot F(y)\).
\end{exercise}
\begin{exercise}[3.9.3]\label{exe:3.9.3}
    Prove in \LJ{:} \(A\land B\seq A\).
\end{exercise}
\begin{exercise}[3.9.4]\label{exe:3.9.4}
    Prove in \LJ{:} \(A\seq A\lor B\).
\end{exercise}
\begin{exercise}[3.9.5]\label{exe:3.9.5}
    Prove in \LJ{:} \(\lnot A\lor\lnot B\seq \lnot\arcs{A\land B}\).
\end{exercise}
\begin{exercise}[3.9.6]\label{exe:3.9.6}
    Prove in \LJ{:} \(\lnot\arcs{A\lor B}\equiv\lnot A\land\lnot B\).
\end{exercise}
\begin{exercise}[3.9.7]\label{exe:3.9.7}
    Prove in \LJ{:} \(\arcs{A\lor C}\land\arcs{B\lor C} \equiv \arcs{A\land B}\lor C\).
\end{exercise}
\begin{exercise}[3.9.8]\label{exe:3.9.8}
    Prove in \LJ{:} \(\exists x\lnot F(x)\seq\lnot\forall xF(x)\).
\end{exercise}
\begin{exercise}[3.9.9]\label{exe:3.9.9}
    Prove in \LJ{:} \(\forall x\arcs{F(x)\land G(x)}\equiv\forall xF(x)\land\forall xG(x)\).
\end{exercise}
\begin{exercise}[3.9.10]\label{exe:3.9.10}
    Prove in \LJ{:} \(A\limplies\lnot B\seq B\limplies\lnot A\).
\end{exercise}
\begin{exercise}[3.9.11]\label{exe:3.9.11}
    Prove in \LJ{:} \(\exists x\arcs{A\limplies B(x)}\seq A\limplies\exists xB(x)\).
\end{exercise}
\begin{exercise}[3.9.12]\label{exe:3.9.12}
    Prove in \LJ{:} \(\exists x\arcs{A(x)\limplies B}\seq\forall xA(x)\limplies B\).
\end{exercise}
\begin{exercise}[3.9.13]\label{exe:3.9.13}
    Prove in \LJ{:} \(\exists x\arcs{A(x)\limplies B(x)}\seq\forall xA(x)\limplies\exists xB(x)\).
\end{exercise}
\begin{exercise}[3.10.1]\label{exe:3.10.1}
    Prove in \LJ{:} \(\lnot\lnot\arcs{A\limplies B}, A\seq\lnot\lnot B\).
\end{exercise}
\begin{exercise}[3.10.2]\label{exe:3.10.2}
    Prove in \LJ{:} \(\lnot\lnot B\limplies B, \lnot\lnot\arcs{A\limplies B}\seq A\limplies B\).
\end{exercise}
\begin{exercise}[3.10.3]\label{exe:3.10.3}
    Prove in \LJ{:} \(\lnot\lnot\lnot A\equiv\lnot A\).
\end{exercise}


\begin{example}[3.11]\label{exa:3.11}
    Let \(R\) be an atomic formula, and let \(J'\) be obtained from \LJ{} by adding the sequent~\(\lnot\lnot R\seq R\) as an axiom.
    Also let \(A\) be a formula which does not contain the symbols \(\lor\) or \(\exists\).
    Then \(\lnot\lnot A\seq A\) is \(J'\)-provable.

    \begin{proof}
        Let \(n\) be the number of logical symbols in \(A\).
        We prove the claim by structurally inducing on \(n\):
        \begin{enumerate}
            \item
                If \(n = 0\), then \(A = R\), as \(R\) is an atomic formula.
                By assumption \(\lnot\lnot R \seq R\), so the claim holds for the base case.
            \item
                Now assume that the inductive hypothesis \(\lnot\lnot A\seq A, \lnot\lnot B\seq B \in J''\) holds for the formulae \(A\) and \(B\),
                with at most \(n = k\) logical symbols. We add one additional allowed symbol to these formulae one at at time to close the induction:
                \begin{description}
                    \item[\(\lnot\)]
                        By exercise~\ref{exe:3.10.3}, \(\lnot\lnot\arcs{\lnot A}\seq\lnot A\) can be proven.
                    \item[\(\limplies\)]
                        We show that the claim \(\lnot\lnot\arcs{A\limplies B} \seq A\limplies B\) can be proven.
                        In the proof, IH refers to the inductive hypothesis:
                        \begin{equation*}
                            \begin{prooftree}
                                \hypo{B\seq B}
                                \ellipsis{\IH}{\lnot\lnot B\seq B}
                                \infer1[\eqref{eq:limplies:right}]{\seq\lnot\lnot B\limplies B}
                                \hypo{A\seq A}
                                \hypo{B\seq B}
                                \infer2{}
                                \ellipsis{(Exercise~\ref{exe:3.10.2})}{\lnot\lnot\arcs{A\limplies B} \seq A\limplies B}
                                \infer2[\eqref{eq:cut}]{\lnot\lnot\arcs{A\limplies B} \seq A\limplies B}
                            \end{prooftree}
                        \end{equation*}
                    \item[\(\land\)]
                        Again, we show that the claim \(\lnot\lnot\arcs{A\land B} \seq A\land B\) can be proven.
                        Also again, IH refers to the inductive hypothesis:
                        \tiny
                        \begin{equation*}
                            \begin{prooftree}
                                \hypo{A&\seq A}
                                \infer1[\eqref{eq:land:left}]{A\land B&\seq A}
                                \infer1[\eqref{eq:lnot:left}]{\lnot A, A\land B&\seq}
                                \infer1[\eqref{eq:exchange:left}]{A\land B,\lnot A&\seq}
                                \infer1[\eqref{eq:lnot:right}]{\lnot A&\seq\lnot\arcs{ A\land B}}
                                \infer1[\eqref{eq:lnot:left}]{\lnot\lnot\arcs{A\land B}, \lnot A&\seq}
                                \infer1[\eqref{eq:exchange:left}]{\lnot A, \lnot\lnot\arcs{A\land B}&\seq}
                                \infer1[\eqref{eq:lnot:right}]{\lnot\lnot\arcs{A\land B}&\seq\lnot\lnot A}
                                \hypo{A\seq A}
                                \ellipsis{\IH}{\lnot\lnot A\seq A}
                                \infer2[\eqref{eq:cut}]{\lnot\lnot\arcs{A\land B}\seq A}
                                \hypo{B&\seq B}
                                \infer1[\eqref{eq:land:left}]{A\land B&\seq B}
                                \infer1[\eqref{eq:lnot:left}]{\lnot B, A\land B&\seq}
                                \infer1[\eqref{eq:exchange:left}]{A\land B,\lnot B&\seq}
                                \infer1[\eqref{eq:lnot:right}]{\lnot B&\seq\lnot\arcs{ A\land B}}
                                \infer1[\eqref{eq:lnot:left}]{\lnot\lnot\arcs{A\land B}, \lnot B&\seq}
                                \infer1[\eqref{eq:exchange:left}]{\lnot B, \lnot\lnot\arcs{A\land B}&\seq}
                                \infer1[\eqref{eq:lnot:right}]{\lnot\lnot\arcs{A\land B}&\seq\lnot\lnot B}
                                \hypo{B\seq B}
                                \ellipsis{\IH}{\lnot\lnot B\seq B}
                                \infer2[\eqref{eq:cut}]{\lnot\lnot\arcs{A\land B}\seq B}
                                \infer2[\eqref{eq:land:right}]{\lnot\lnot\arcs{A\land B} \seq A\land B}
                            \end{prooftree}
                        \end{equation*}
                        \normalsize
                    \item[\(\forall\)]
                        Lastly, we show that the claim \(\lnot\lnot\forall xA(x) \seq \forall xA(x)\) is a theorem:
                        \footnotesize
                        \begin{equation*}
                            \begin{prooftree}
                                \hypo{A(a)\seq A(a)}
                                \infer1[\eqref{eq:forall:left}]{\forall xA(x)&\seq A(a)}
                                \infer1[\eqref{eq:lnot:left}]{\lnot A(a),\forall xA(x)&\seq}
                                \infer1[\eqref{eq:exchange:left}]{\forall xA(x),\lnot A(a)&\seq}
                                \infer1[\eqref{eq:lnot:right}]{\lnot A(a)&\seq\lnot\forall xA(x)}
                                \infer1[\eqref{eq:lnot:left}]{\lnot\lnot\forall xA(x), \lnot A(a)&\seq}
                                \infer1[\eqref{eq:exchange:left}]{\lnot A(a), \lnot\lnot\forall xA(x)&\seq}
                                \infer1[\eqref{eq:lnot:right}]{\lnot\lnot\forall xA(x)&\seq \lnot\lnot A(a)}
                                \hypo{A(a)\seq A(a)}
                                \ellipsis{\IH}{\lnot\lnot A(a)\seq A(a)}
                                \infer2[\eqref{eq:cut}]{\lnot\lnot\forall xA(x) \seq A(a)}
                                \infer1[\eqref{eq:forall:right}]{\lnot\lnot\forall xA(x) \seq \forall xA(x)}
                            \end{prooftree}
                        \end{equation*}
                        \normalsize
                \end{description}
        \end{enumerate}
        By the inductive principle, we are then done.
    \end{proof}
\end{example}

\subsection{\texorpdfstring{The relation between \LK{} and \LJ}{The relation between LK and LJ}}

Our last goal in discussing what the specific connection between \LK{} and \LJ{} is.
For this purpose we declare the notational definitions in definition~\ref{def:starred formulae in LK}.

\begin{definition}[Starred formula variants in \LK]\label{def:starred formulae in LK}
    Table~\ref{tab:starred variants in LK} defines a recursive mapping,
    between formulae \(A\) and their starred variants \(A^\ast\).

    \begin{tabenv}{Mapping between formulae \(A\) and their transformations \(A^\ast\).}%
        \label{tab:starred variants in LK}
        \begin{tabular}{C|C}
            \toprule
            \text{Formula } A & \text{Transformation } A^\ast \\
            \midrule
            \text{atomic } A & \lnot\lnot A \\
            \lnot B & \lnot\arcs{B^\ast} \\
            B\land C & B^\ast\land C^\ast \\
            B\lor C & \lnot\arcs{\lnot B^\ast\land\lnot C^\ast} \\
            \forall xB(x) & \forall xB^\ast(x) \\
            \exists xB(x) & \lnot\forall x\lnot B^\ast(x) \\
            B\limplies C & B^\ast\limplies C^\ast \\
            \bottomrule
        \end{tabular}
    \end{tabenv}
\end{definition}

With definition~\ref{def:starred formulae in LK} in place, we can proceed to make the propositions that follow.

\begin{proposition}[Claim 1]\label{prop:claim 1 in LK vs LJ}
    \(A\equiv A^\ast\) for all \(A\) in \LK.
\end{proposition}

\begin{proof}
    Just like with example~\ref{exa:3.11}, the proof can be achieved via induction over the number of logical symbols in \(A\).
    \begin{enumerate}
        \item
            For the base case, we observe that by the double negation law proved in example~\ref{exa:double negation in LK},
            \(A\equiv\lnot\lnot A = A^\ast\).
        \item
            For the induction hypothesis \IH, we assume that for the formulae \(B\) and \(C\),
            \(B\equiv B^\ast\) and \(C\equiv C^\ast\). To close the hypothesis,
            we add one allowed logical symbol at a time to \(A\) formed from \(B\) and/or \(C\),
            as necessary:
            \begin{description}
                \item[\lnot]
                    If \(A = \lnot B\), then \(A^\ast = \lnot\lnot\lnot B\),
                    and \(A\equiv A^\ast\) by exercise~\ref{exe:3.10.3}.
                \item[\land]
                    If \(A = B\land C\), then \(A^\ast = B^\ast\land C^\ast\) and we argue as follows:
                    \begin{equation*}
                        \begin{prooftree}
                            \hypo{B\seq B}
                            \ellipsis{\IH}{B\seq B^\ast}
                            \infer1{B\land C\seq B^\ast}
                            \hypo{C\seq C}
                            \ellipsis{\IH}{C\seq C^\ast}
                            \infer1{B\land C\seq C^\ast}
                            \infer2{B\land C \seq B^\ast\land C^\ast}
                            \hypo{B\seq B}
                            \ellipsis{\IH}{B^\ast\seq B}
                            \infer1{B^\ast\land C^\ast\seq B}
                            \hypo{C\seq C}
                            \ellipsis{\IH}{C^\ast\seq C}
                            \infer1{B^\ast\land C^\ast\seq C}
                            \infer2{B^\ast\land C^\ast \seq B\land C}
                            \infer2{B\land C \equiv B^\ast\land C^\ast}
                        \end{prooftree}
                    \end{equation*}
                \pagebreak
                \item[\lor]
                    \begin{exercise}[3.12.1 part iii]\label{exe:3.12.1.iii}
                        This is left as an exercise. Show that \(A \equiv A^\ast\), when \(A=B\lor C\).
                    \end{exercise}
                \item[\forall]
                    If \(A = \forall xB(x)\), then \(A^\ast = \forall xB^\ast(x)\).
                    Now we can argue as follows:
                    \begin{equation*}
                        \begin{prooftree}
                            \hypo{B(a\seq B(a))}
                            \ellipsis{\IH}{B(a)\seq B^\ast(a)}
                            \infer1[\eqref{eq:forall:left}]{\forall xB(x)\seq B^\ast(a)}
                            \infer1[\eqref{eq:forall:right}]{\forall xB(x)\seq\forall xB^\ast(x)}
                            \hypo{B(a\seq B(a))}
                            \ellipsis{identical to left branch}{\forall xB^\ast(x)\seq\forall xB(x)}
                            \infer2{\forall xB^\ast(x)\equiv\forall xB(x)}
                        \end{prooftree}
                    \end{equation*}
                \item[\limplies]
                    If \(A = B\limplies C\), then \(A^\ast = B^\ast\limplies C^\ast\).
                    The argument is then
                    \begin{equation*}
                        \begin{prooftree}
                            \hypo{B\seq B}
                            \ellipsis{\IH}{B^\ast\seq B}
                            \hypo{C\seq C}
                            \ellipsis{\IH}{C\seq C^\ast}
                            \infer[left label=\eqref{eq:limplies:left}]2{B\limplies C, B^\ast\seq C^\ast}
                            \infer[left label=\eqref{eq:exchange:left}]1{B^\ast, B\limplies C\seq C^\ast}
                            \infer[left label=\eqref{eq:limplies:right}]1{B\limplies C\seq B^\ast\limplies C^\ast}
                            \hypo{B\seq B}
                            \ellipsis{\IH}{B\seq B^\ast}
                            \hypo{C\seq C}
                            \ellipsis{\IH}{C^\ast\seq C}
                            \infer[left label=\eqref{eq:limplies:left}]2{B^\ast\limplies C^\ast, B\seq C}
                            \infer[left label=\eqref{eq:exchange:left}]1{B, B^\ast\limplies C^\ast\seq C}
                            \infer[left label=\eqref{eq:limplies:right}]1{B^\ast\limplies C^\ast\seq B\limplies C}
                            \infer2{B^\ast\limplies C^\ast\equiv B\limplies C}
                        \end{prooftree}
                    \end{equation*}
                \item[\exists]
                    \begin{exercise}[3.12.1 part vi]\label{exe:3.12.1.vi}
                        This is left as an exercise. Show that if \(A = \exists xB(x)\), then \(A\equiv A^\ast\).
                    \end{exercise}
            \end{description}
    \end{enumerate}
    By the inductive principle, we are then done.
\end{proof}

\begin{proposition}[Claim 2]\label{prop:claim 2 in LK vs LJ}
    If \(S\) is the sequent \(A_1,\ldots,A_m\seq B_1,\ldots,B_n\),
    then we may define the sequent
    \(S'\coloneqq A_1^\ast,\ldots,A_m^\ast,\lnot B_1^\ast,\ldots,\lnot B^\ast_n\seq\).
    Now \(S\) is \LK-provable, if and only if \(S'\) is \LK-provable.
\end{proposition}

\begin{proof}[Sketch of proof]
    We start by proving that if \(S\) is provable (meaning we can use it as a premise of a proof),
    then \(S'\) is also provable:
    \tiny

    \begin{equation*}
        \begin{prooftree}
            \hypo{A_2^\ast\seq\lnot\lnot A_2^\ast}
            \hypo{A_1^\ast\seq\lnot\lnot A_1^\ast}
            % Tallest branch
            \hypo{\overbrace{A_1,\ldots,A_m\seq B_1,\ldots,B_n}^{S}}
            \hypo{B_n\seq B_n^\ast}
            \infer2[\eqref{eq:cut}]{A_1,\ldots,A_m\seq B_1,\ldots,B_{n-1}, B_n^\ast}
            \infer1[\eqref{eq:exchange:right}]{A_1,\ldots,A_m\seq B_1,\ldots,B_n^\ast, B_{n-1}}
            \hypo{B_{n-1}\seq B_{n-1}^\ast}
            \infer2[\eqref{eq:cut}]{A_1,\ldots,A_m\seq B_1,\ldots,B_n^\ast, B_{n-1}^\ast}
            \infer[double]1[\eqref{eq:exchange:right}s and \eqref{eq:cut}s]{A_1,\ldots,A_m\seq B_n^\ast,\ldots,B_1^\ast}
            \infer[double]1[\eqref{eq:lnot:left}s and \eqref{eq:exchange:left}s]{A_1,\ldots,A_m, \lnot B_n^\ast,\ldots,\lnot B_1^\ast\seq}
            \infer[double]1[\eqref{eq:lnot:right}s and \eqref{eq:exchange:right}s]{\lnot B_n^\ast,\ldots,\lnot B_1^\ast\seq\lnot A_m,\ldots,\lnot A_1}
            \hypo{\lnot A_1\seq \lnot A^\ast_1}
            \infer[left label=\eqref{eq:cut} and \eqref{eq:exchange:right}]2{\lnot B_n^\ast,\ldots,\lnot B_1^\ast\seq\lnot A_m,\ldots,\lnot A_1^\ast,\lnot A_2}
            \hypo{\lnot A_2\seq\lnot A_2^\ast}
            \infer[double,left label=\eqref{eq:cut}s and \eqref{eq:exchange:right}s]2{\lnot B_n^\ast,\ldots,\lnot B_1^\ast\seq\lnot A_1^\ast,\ldots,\lnot A_m^\ast}
            \infer[double,left label=\eqref{eq:lnot:right}s and \eqref{eq:exchange:left}s]1{\lnot\lnot A_1^\ast,\ldots,\lnot\lnot A_m^\ast, \lnot B_n^\ast,\ldots,\lnot B_1^\ast\seq}
            \infer2[\eqref{eq:cut}]{\lnot\lnot A_2^\ast,\ldots,\lnot\lnot A_m^\ast, \lnot B_n^\ast,\ldots,\lnot B_1^\ast\seq}
            \infer[double]2[multiple \eqref{eq:cut}s]{A_1^\ast,\ldots,A_m^\ast, \lnot B_n^\ast,\ldots,\lnot B_1^\ast\seq}
        \end{prooftree}
    \end{equation*}
    \normalsize
    On the last line, \(A_1^\ast,\ldots,A_m^\ast, \lnot B_n^\ast,\ldots,\lnot B_1^\ast\seq = S'\).
    The proof in the other direction is performed in a similar manner.
\end{proof}

\begin{proposition}[Claim 3]\label{prop:claim 3 in LK vs LJ}
    \(A^\ast\equiv\lnot\lnot A^\ast\) in \LJ.
\end{proposition}

\begin{proof}
    In \LJ{}, the proof
    \begin{equation*}
        \begin{prooftree}
            \hypo{A\seq A}
            \infer1{\lnot A, A\seq}
            \infer1{A\seq\lnot\lnot A}
        \end{prooftree}
    \end{equation*}
    is valid for all formulae \(A\). This means it must also be true for \(A^\ast\seq\lnot\lnot A^\ast\) as well.

    By example~\ref{exa:3.11}, \(\lnot\lnot A\seq A\) in \LJ{'}.
    Also, by exercise~\ref{exe:3.10.3} we have \(\lnot\lnot\lnot\arcs{\lnot R}\equiv\lnot\lnot R\) in \LJ{},
    for all atomic formula \(R\).
    As the formula \(A^\ast\) does not contain the symbols \(\lor\) or \(\exists\) by assumption,
    we can conclude that \(\lnot\lnot A^\ast\seq A\).
    Therefore \(A^\ast\equiv \lnot\lnot A^\ast\) for all \(A\) in \LJ{}.
\end{proof}

\begin{proposition}[Claim 4]\label{prop:claim 4 in LK vs LJ}
    If the sequent \(S\) of proposition~\ref{prop:claim 2 in LK vs LJ} is provable in \LK,
    then the sequent \(S'\) defined in the same proposition is provable in \LJ.
\end{proposition}

\begin{proof}[A sketch of a proof]
    Let \(n\) refer to the length of a proof, as in the number of inferences of a proof in \LK{}.
    We induce on this \(n\):
    \begin{enumerate}
        \item
            For the base case \(n=0\), all proofs \(P = A\seq A\) in \LK{}.
            Then in \LJ{} we have \(A^\ast\seq A^\ast\) and
            \begin{equation*}
                \begin{prooftree}
                    \hypo{A^\ast\seq A^\ast}
                    \infer1[\eqref{eq:lnot:left}]{\lnot A^\ast, A^\ast\seq}
                    \infer1[\eqref{eq:exchange:left}]{A^\ast, \lnot A^\ast\seq}
                \end{prooftree}
            \end{equation*}
        \item
            For the inductive hypothesis \IH, let the claim in \LK{} hold for a proof of length \(n \leq p\).
            Note that based on the number of inference rules in definition~\ref{def:inference rules in LK},
            the number of \emph{subcases} of \(n = p+1\) is \(21\). Let us cover some of them:
            \begin{enumerate}
                \item
                    In \LK{} we have
                    \begin{equation*}
                        \begin{prooftree}
                            \hypo{A_1,\ldots,A_k\seq B_1,\ldots,B_l}
                            \infer1[\eqref{eq:weakening:left}]{A, A_1,\ldots,A_k\seq B_1,\ldots B_1,\ldots,B_l}
                        \end{prooftree}
                    \end{equation*}
                    whereas in \LJ{} this means that
                    \begin{equation*}
                        \begin{prooftree}
                            \hypo{A_1^\ast,\ldots, A_k^\ast, \lnot B_1^\ast,\ldots,\lnot B_l^\ast\seq}
                            \infer1[\eqref{eq:weakening:left}]{A^\ast, A_1,\ldots,A_k, \lnot B_1^\ast,\ldots,\lnot B_l^\ast\seq}
                        \end{prooftree}
                    \end{equation*}
                    by the inductive hypothesis.
                \item
                    In \LK{} we have
                    \begin{equation*}
                        \begin{prooftree}
                            \hypo{A_1,\ldots, A_i,A_j,\ldots,A_k\seq B_1,\ldots,B_l}
                            \infer1[\eqref{eq:exchange:left}]{A_1,\ldots, A_j,A_i,\ldots,A_k\seq B_1,\ldots,B_l}
                        \end{prooftree}
                    \end{equation*}
                    whereas in \LJ{} this means that
                    \begin{equation*}
                        \begin{prooftree}
                            \hypo{A_1^\ast,\ldots, A_i^\ast,A_j^\ast,\ldots,A_k^\ast, \lnot B_1^\ast,\ldots,\lnot B_l^\ast\seq}
                            \infer1[\eqref{eq:exchange:left}]{A_1^\ast,\ldots, A_j^\ast,A_i^\ast,\ldots,A_k^\ast, \lnot B_1^\ast,\ldots,\lnot B_l^\ast\seq},
                        \end{prooftree}\,,
                    \end{equation*}
                    again by the inductive hypothesis.
                \item
                    In \LK{}, by the rule~\eqref{eq:cut} (the \(p+1\)st inference):
                    \begin{equation*}
                        \begin{prooftree}
                            \hypo{A_1,\ldots,A_i\seq B_1,\ldots,B_j, D}
                            \hypo{D, A_{i+1},\ldots,A_k\seq B_{j+1},\ldots,B_l, D}
                            \infer2{A_1,\ldots,A_k\seq B_1,\ldots,B_l}
                        \end{prooftree}
                    \end{equation*}
                    Then by the inductive hypothesis the inference
                    \scriptsize
                    \begin{equation*}
                        \begin{prooftree}
                            \hypo{A_1^\ast,\ldots,A_i^\ast, \lnot B_1^\ast,\ldots,\lnot B_j^\ast,\lnot D^\ast\seq}
                            \infer[double]1[\eqref{eq:exchange:left}]{\lnot D^\ast, A_1^\ast,\ldots,A_i^\ast, \lnot B_1^\ast,\ldots,\lnot B_j^\ast\seq}
                            \infer1[\eqref{eq:lnot:right}]{A_1^\ast,\ldots,A_i^\ast, \lnot B_1^\ast,\ldots,\lnot B_j^\ast\seq\lnot\lnot D^\ast}
                            \hypo{\text{proposition}~\ref{prop:claim 3 in LK vs LJ}}
                            \ellipsis{}{\lnot\lnot D^\ast\seq D^\ast}
                            \infer2[\eqref{eq:cut}]{A_1^\ast,\ldots,A_i^\ast, \lnot B_1^\ast,\ldots,\lnot B_j^\ast\seq D^\ast}
                            \hypo{\IH}
                            \ellipsis{}{D^\ast, A_{i+1}^\ast,\ldots,A_k^\ast, \lnot B_{j+1}^\ast,\ldots,\lnot B_l^\ast\seq}
                            \infer[left label=\eqref{eq:cut}]2{A_1^\ast,\ldots,A_i^\ast, \lnot B_1^\ast,\ldots,\lnot B_j^\ast,A_{i+1}^\ast,\ldots,A_k^\ast, \lnot B_{j+1}^\ast,\ldots,\lnot B_l^\ast\seq}
                            \infer[rule style=double]1[\eqref{eq:exchange:left}]{A_1^\ast,\ldots,A_k^\ast, \lnot B_1^\ast,\ldots,\lnot B_l^\ast\seq}
                        \end{prooftree}
                    \end{equation*}
                    \normalsize
                \pagebreak
                \item
                    \begin{exercise}[3.12.4 part vii]\label{exe:3.12.4.vii}
                    This is left as an exercise.
                    If \(S\) is the sequent \(A_1,\ldots,A_k\seq B_1,\ldots,B_l\),
                    then we may define the sequent
                    \(S'\coloneqq A_1^\ast, \ldots, A_k^\ast,\linebreak
                    \lnot B_1^\ast, \ldots, \lnot B^\ast_l\seq\).
                    Show that if the sequent \(S\) is provable in \LK,
                    then the sequent \(S'\) is provable in \LJ{}
                    for the case, where the \(\arcs{p+1}\)st inference is
                    by~\eqref{eq:lnot:right} in \LK{}.
                    \end{exercise}
            \end{enumerate}
    \end{enumerate}

    The rest of the applications of inference rules needed in the inductive step
    are omitted in these notes, but may be done as an exercise.
    By the inductive principle, we can now conclude that if the sequent
    \(A_1,\ldots,A_k\seq B_1,\ldots,B_l\) is in \LK{},
    then the sequent \(A_1^\ast,\ldots,A_k^\ast, \lnot B_1^\ast,\ldots,\lnot B_l^\ast\seq\) is in \LJ{}.
\end{proof}

The converse of proposition~\ref{prop:claim 4 in LK vs LJ} also holds,
but the proof is just as long and will therefore be omitted.
Lastly, we cover the most important fact related to this section.

\begin{theorem}[Connection between \LK{} and \LJ]\label{the:connection between LK and LJ}
    For all formulae \(A\), \(\seq A\) in \LK{}, if and only if \(\seq A^\ast\) in \LJ{}.
\end{theorem}

\begin{proof}
    A particular instance of exercise~\ref{exe:3.12.4.vii} is the following:
    if \(\seq A\) in \LJ{}, then \(\lnot A^\ast\seq\) in LJ{}.
    Therefore
    \begin{equation*}
        \begin{prooftree}
            \hypo{\lnot A^\ast\seq}
            \infer1[\eqref{eq:lnot:right}]{\seq\lnot\lnot A^\ast}
            \hypo{A\seq A}
            \ellipsis{Proposition~\ref{prop:claim 3 in LK vs LJ}}{A^\ast\seq A}
            \infer2[\eqref{eq:cut}]{\seq A^\ast}
        \end{prooftree}
    \end{equation*}

    Conversely, if \(\seq A^\ast\) in \LJ{} and by proposition~\ref{prop:claim 1 in LK vs LJ} we know that \(A\equiv A^\ast\) in \LK{},
    we also have \(A^\ast\seq A\) in \LK{}. Thus
    \begin{equation*}
        \begin{prooftree}
            \hypo{\seq A^\ast}
            \hypo{A^\ast\seq A}
            \infer2[\eqref{eq:cut}]{\seq A}
        \end{prooftree}\,.
        \qedhere
    \end{equation*}
\end{proof}
