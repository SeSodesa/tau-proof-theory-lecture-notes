\section{\texorpdfstring{Completeness of \LK}{Completeness of LK}}%
\label{sec:completeness-of-lk}

We would like to show the following in \LK: for any formula \(A\), if \(\seq
A\) is provable, then A is a \emph{valid} formula. This leads to the
\emph{soundness} and eventually \emph{completeness} of \LK. To establish what
these concepts mean, we give the following (rather long) sequence of
definitions.

\begin{definition}[Structure of language]\label{def:structure of language}
    Let \(L\) be a language as described in table~\ref{tab:language hierarchy}.
    The \emph{structure} of such a language is an ordered pair \(\tuple{D,\phi}\),
    where \(D\) is a non-empty set and \(\phi\) a map from the constants of \(L\)
    such that
    \begin{enumerate}
        \item
            if \(k\) is an individual constant, then \(\phi(k)\in D\),
        \item
            if \(f\) is a function constant of \(n\) arguments,
            then \(\phi(f): D^n\to D\) and
        \item
            if \(R\) is a predicate constant of \(n\) arguments,
            then \(\phi(R)\subseteq D^n\).
    \end{enumerate}
\end{definition}

\begin{definition}[Interpretation of language]\label{def:interpretation of language}
    An \emph{interpretation} \(\interp\) of a language \(L\) is an ordered pair \(\tuple{S,\phi_0}\),
    where \(S\) is a structure \(\tuple{D,\phi}\) and \(\phi_0\)
    is a mapping from the variables of \(L\) to \(D\).
    The map \(\phi_0\) is called an \emph{assignment} from \(D\)
    and is defined as follows:
    \begin{enumerate}
        \item
            \(\phi_0\arcs{a}\in D\) for all free variables \(a\) and
        \item
            \(\phi_0\arcs{x}\in D\) for all bound variables \(x\).
    \end{enumerate}
\end{definition}

\begin{definition}[Extension of \(\phi\)]\label{def:extension of structure map}
    We define \(\phi\arcs{t}\) in a structure \(S = \tuple{D,\phi}\)
    for every \emph{semi-term} \(t\) inductively
    as follows:
    \begin{enumerate}
    \item
        \(\phi\arcs{a} = \phi_0\arcs{a}\) and \(\phi\arcs{x} = \phi_0\arcs{x}\)
        for all free variables \(a\) and bound variables \(x\).
    \item
        In addition,
        if \(f\) is a function constant and \(t\) is a semi-term for which \(\phi\)
        is already defined, then \(\phi\arcs{f\arcs{t}} = \arcs{\phi f}\arcs{\phi t}\).
    \end{enumerate}
\end{definition}

\begin{definition}[Formula-satisfying interpretation]\label{def:formula satisfying interpretation}
    An interpretation \(\interp = \tuple{S,\phi_0} = \tuple{\tuple{D,\phi},\phi_0}\)
    \emph{satisfies} a formula \(A\), if the following are true:
    \begin{enumerate}
        \item
            If \(R\) is a predicate constant of \(n\) arguments
            and \(t_1,\ldots,t_n\) are semi-terms, then \(\interp\)
            satisfies \(R\arcs{t_1,\ldots,t_n}\), if and only if
            \(\tuple{\phi t_1,\ldots,\phi t_n}\in\phi R \subseteq D^n\).
        \item
            The interpretation \(\interp\) satisfies \(\lnot A\),
            if and only if it does \emph{not} satisfy \(A\).
        \item
            \(\interp\) satisfies \(A\land B\), if and only if it satisfies
            both \(A\) and \(B\).
        \item
            The formula \(A\lor B\)
            is satisfied by \(\interp\), if and only if it
            satisfies either \(A\) or \(B\).
        \item
            The implication \(A\limplies B\) is satisfied by \(\interp\),
            if and only if \(A\) is not satisfied,
            or \(B\) is satisfied.
        \item
            The interpretation \(\interp = \tuple{\tuple{D,\phi},\phi_0}\)
            satisfies \(\forall xB\arcs{x}\), if and only if for \emph{every} \(\phi_0'\)
            such that \(\phi_0'(x) = \phi_0(x)\) almost everywhere,
            the interpretation \(\interp'=\tuple{\tuple{D,\phi},\phi_0'}\) satisfies \(B\).
        \item
            The interpretation \(\interp = \tuple{\tuple{D,\phi},\phi_0}\)
            satisfies \(\exists xB\arcs{x}\), if and only if for \emph{some} \(\phi_0'\)
            such that \(\phi_0'(x) = \phi_0(x)\) almost everywhere,
            the interpretation \(\interp'=\tuple{\tuple{D,\phi},\phi_0'}\) satisfies \(B\).
    \end{enumerate}
    If \(\interp = \tuple{\tuple{D,\phi},\phi_0}\) satisfies a formula \(A\),
    \(A\) is said to be satisfied in \(\tuple{D,\phi}\) by the assignment \(\phi_0\).
\end{definition}

Note that in the above definition, \emph{almost everywhere} means that there might
exist some \(a\) for which \(\phi_0'(a)\neq\phi_0(a)\).

\begin{definition}[Valid formula]\label{def:valid formula}
    A formula \(A\) is said to be \emph{valid in} the structure \(\tuple{D,\phi}\),
    if and only if for every \(\phi_0\),
    the interpretation \(\tuple{\tuple{D,\phi},\phi_0}\)
    satisfies \(A\). The formula \(A\) is simply \emph{valid}, if
    it is valid in every structure.
\end{definition}

\begin{definition}[Satisfied and valid sequents]%
\label{def:satisfied and valid sequents}
A sequent \(\Gamma\seq\Delta\) is \emph{satisfied in} the structure
\(\tuple{D,\phi}\) \emph{by} an interpretation \(\interp\),
if and only if either some formula in \(\Gamma\) is \emph{not}
satisfied by \(\interp\), or some formula in \(\Delta\) \emph{is}
satisfied by \(\interp\). A sequent is \emph{valid}, if and only if
it is satisfied in every interpretation.
\end{definition}

\begin{definition}[Model and Counter-Model]%
\label{def:model and counter-model}
A structure \(S\) is called a \emph{model} of an axiom system
\(\Gamma\), if every sentence of \(\Gamma\) is valid in \(S\).
On the other hand, \(S\) is a \emph{counter-model} of \(\Gamma\),
if there is a sentence of \(\Gamma\) that is not valid in \(S\).
\end{definition}

The definitions needed in proving the completeness of \LK{},
namely theorems~\ref{the:provability implies validity}
and \ref{the:validity implies provability},
are now mostly in place. In essence, the completeness of \LK{}
means that a formula \(A\) is provable,
if and only if it is valid.

\begin{theorem}[Soundness of \LK: provability implies validity]\label{the:provability implies validity}
    If a formula \(A\) is provable in \LK, then it is valid in \LK.
    In other words \LK{} is a \emph{sound} logical system.
\end{theorem}

\begin{proof}
    The proof is by structural induction on the number inferences
    in the proof of \(\Gamma\seq\Delta\), denoted by \(n\).
    If \(n=0\) a proof only consists of axioms of the form \(A\seq A\),
    where the antecedent \(A\) is never satisfied in the interpretation
    \(\interp\), whereas the succedent \(A\) always is. Therefore the
    sequent \(A\seq A\) is always satisfied in all \(\interp\) and is then valid.

    We now make the induction hypothesis \IH, that if the number of inferences
    \(n\leq k\) in the proofs of the sequents \(\Gamma\seq\Delta\) and \(\Pi\seq\Lambda\),
    then both of them are valid. In the inductive step, we add each of the inference
    rules in definition \ref{def:inference rules in LK} to the proof one at a time
    (except \eqref{eq:cut}), and check that validity is preserved.

    \paragraph{Structural rules.}
    These follow directly from the definition of validity.
    In the inferences
    \begin{equation*}
        \begin{prooftree}
            \hypo{\Gamma\seq\Delta}
            \infer1[\eqref{eq:weakening:left}]{A, \Gamma\seq\Delta}
        \end{prooftree}
        \quad\text{and}\quad
        \begin{prooftree}
            \hypo{\Gamma\seq\Delta}
            \infer1[\eqref{eq:weakening:right}]{\Gamma\seq\Delta, A}
        \end{prooftree}\,,
    \end{equation*}
    the sequence \(\Gamma\) contains at least one \emph{invalid} formula \(B\),
    whereas \(\Delta\) contains at least one \emph{valid} formula \(B\) by the inductive hypothesis.
    The other two types of non-\eqref{eq:cut} structural inferences
    \begin{equation*}
        \begin{prooftree}
            \hypo{A,A,\Gamma\seq\Delta}
            \infer1[\eqref{eq:contraction:left}]{A, \Gamma\seq\Delta}
        \end{prooftree}
        \quad\text{and}\quad
        \begin{prooftree}
            \hypo{\Gamma\seq\Delta,A,A}
            \infer1[\eqref{eq:contraction:right}]{\Gamma\seq\Delta, A}
        \end{prooftree}\,,
    \end{equation*}
    and
    \begin{equation*}
        \begin{prooftree}
            \hypo{C,D,\Gamma\seq\Delta}
            \infer1[\eqref{eq:exchange:left}]{D,C,\Gamma\seq\Delta}
        \end{prooftree}
        \quad\text{and}\quad
        \begin{prooftree}
            \hypo{\Gamma\seq\Delta,C,D}
            \infer1[\eqref{eq:exchange:right}]{\Gamma\seq\Delta, D,C}
        \end{prooftree}\,,
    \end{equation*}
    also preserve validity by the same logic.

    \paragraph{Logical rules.}
    We start with the rules
    \begin{equation*}
        \begin{prooftree}
            \hypo{\Gamma\seq\Delta,A}
            \infer1[\eqref{eq:lnot:left}]{\lnot A, \Gamma\seq\Delta}
        \end{prooftree}
        \quad\text{and}\quad
        \begin{prooftree}
            \hypo{A, \Gamma\seq\Delta}
            \infer1[\eqref{eq:lnot:right}]{\Gamma\seq\Delta, \lnot A}
        \end{prooftree}\,.
    \end{equation*}
    Given an interpretation \(\interp\),
    if the satisfied formula is in \(\Delta\) or the unsatisfied
    one in \(\Gamma\), then there is nothing to prove.
    Otherwise, if \(A\) is the satisfied formula,
    then \(\lnot A\) is not satisfied by definition,
    meaning \(\lnot A\) must contain a formula \(C\) that is not satisfied.
    This implies that the sequent \(\lnot A,\Gamma\seq\Delta\) is satisfied by definition.
    The proof of the sequent \(\Gamma\seq\Delta,\lnot A\) being satisfied
    in the inference \eqref{eq:lnot:right} is entirely symmetrical to this.

    \begin{exercise}[Exercise 8.2.1]\label{exe:8.2.1}
        Prove the inductive step in the inferences~\eqref{eq:land:left},
        \eqref{eq:land:right}, \eqref{eq:lor:left}, \eqref{eq:lor:right},
        \eqref{eq:limplies:left} and \eqref{eq:limplies:right}.
    \end{exercise}

    \begin{exercise}[Exercise 8.2.2]\label{exe:8.2.2}
        Prove the inductive step in the inferences~\eqref{eq:forall:left},
        \eqref{eq:forall:right}, \eqref{eq:exists:left} and \eqref{eq:exists:right}.
    \end{exercise}

    With all of the inference rules covered,
    the induction is closed and the proof is now complete.
\end{proof}

To prove the converse, that the validity of the sequent \(\seq A\)
implies the provability of \(\seq A\), we will need on moredefinition:
the concept of \emph{reduction trees}.

\begin{definition}[Reduction tree]\label{def:reduction tree}
    Let \(\Gamma\seq\Delta\) be a sequent.
    The \emph{reduction tree} \(T\arcs{\Gamma\seq\Delta}\)
    for this sequent is constructed
    programmatically as follows:
    \begin{enumerate}
        \item\label{it:reduction tree init}
            Place the sequent \(\Gamma\seq\Delta\) at the root of the tree.
        \item\label{it:reduction tree loop}
            The branches of the tree are defined by looping over the current
            leaves:
            \begin{enumerate}
                \item\label{it:reduction tree finished}
                    If every topmost sequent contains a same formula
                    in both its antecedent and succedent,
                    then halt the program.
                \item\label{it:reduction tree unfinished}
                    Else iterate over the leaves of the current branches
                    and perform one of the following \emph{reductions} or additions,
                    depending on which logical symbols are the outermost symbols in \(\Pi\) or \(\Lambda\),
                    when \(\Pi\seq\Lambda\) is the sequent in each leaf during each iteration:
                    \begin{enumerate}
                        \item\label{it:reduction tree lnot left}
                            Let \(\lnot A_1,\ldots,\lnot A_n\in\Pi\) with
                            \(\lnot\) as the outermost logical symbol.
                            Then the following push operation can be performed:
                            \begin{equation}
                                \begin{prooftree}
                                    \tag{reduction:\(\lnot\):left}
                                    \label{eq:reduction:lnot:left}
                                    \hypo{\Pi\seq\Lambda, A_1,\ldots,A_n}
                                    \infer1{\Pi\seq\Lambda}
                                \end{prooftree}
                            \end{equation}
                        \item\label{it:reduction tree lnot right}
                            Let \(\lnot A_1,\ldots,\lnot A_n\in\Lambda\) with
                            \(\lnot\) as the outermost logical symbol.
                            Then the following push operation can be performed:
                            \begin{equation}
                                \begin{prooftree}
                                    \tag{reduction:\(\lnot\):right}
                                    \label{eq:reduction:lnot:right}
                                    \hypo{A_1,\ldots,A_n,\Pi\seq\Lambda}
                                    \infer1{\Pi\seq\Lambda}
                                \end{prooftree}
                            \end{equation}
                        \item\label{it:reduction tree land left}
                            Let \(A_1\land B_1,\ldots,A_n\land B_n\in\Pi\) with
                            \(\land\) as the outermost logical symbol.
                            Then the following push operation can be performed:
                            \begin{equation}
                                \begin{prooftree}
                                    \tag{reduction:\(\land\):left}
                                    \label{eq:reduction:land:left}
                                    \hypo{A_1, B_1,\ldots,A_n, B_n,\Pi\seq\Lambda}
                                    \infer1{\Pi\seq\Lambda}
                                \end{prooftree}
                            \end{equation}
                        \item\label{it:reduction tree land right}
                            Let \(A_1\land B_1,\ldots,A_n\land B_n\in\Lambda\) with
                            \(\land\) as the outermost logical symbol.
                            Then the following push operation can be performed:
                            \begin{equation}
                                \begin{prooftree}
                                    \tag{reduction:\(\land\):right}
                                    \label{eq:reduction:land:right}
                                    \hypo{\Pi\seq\Lambda,C_1}
                                    \hypo{\ldots}
                                    \hypo{\Pi\seq\Lambda,C_n}
                                    \infer3{\Pi\seq\Lambda}
                                \end{prooftree}
                            \end{equation}
                            where each \(C_i\in\set{A_i, B_i}\).
                        \item\label{it:reduction tree lor left}
                            Let \(A_1\lor B_1,\ldots,A_n\lor B_n\in\Pi\) with
                            \(\lor\) as the outermost logical symbol.
                            Then the following push operation can be performed:
                            \begin{equation}
                                \begin{prooftree}
                                    \tag{reduction:\(\lor\):left}
                                    \label{eq:reduction:lor:left}
                                    \hypo{C_1,\Pi\seq\Lambda}
                                    \hypo{\ldots}
                                    \hypo{C_n,\Pi\seq\Lambda}
                                    \infer3{\Pi\seq\Lambda}
                                \end{prooftree}
                            \end{equation}
                            where each \(C_i\in\set{A_i, B_i}\).
                        \item\label{it:reduction tree lor right}
                            Let \(A_1\lor B_1,\ldots,A_n\lor B_n\in\Lambda\) with \(\lor\) as the outermost logical symbol.
                            Then the following push operation can be performed:
                            \begin{equation}
                                \begin{prooftree}
                                    \tag{reduction:\(\lor\):right}
                                    \label{eq:reduction:lor:right}
                                    \hypo{\Pi\seq\Lambda,A_1, B_1,\ldots,A_n, B_n}
                                    \infer1{\Pi\seq\Lambda}
                                \end{prooftree}
                            \end{equation}
                        \item\label{it:reduction tree limplies left}
                            Let \(A_1\limplies B_1,\ldots,A_n\limplies B_n\in\Pi\) with
                            \(\limplies\) as the outermost logical symbol.
                            Then the following push operation can be performed:
                            \begin{equation}
                                \begin{prooftree}
                                    \tag{reduction:\(\limplies\):left}
                                    \label{eq:reduction:limplies:left}
                                    \hypo{B_1,\Pi\seq\Lambda,A_1}
                                    \hypo{\ldots}
                                    \hypo{B_n,\Pi\seq\Lambda, A_n}
                                    \infer3{\Pi\seq\Lambda}
                                \end{prooftree}
                            \end{equation}
                        \item\label{it:reduction tree limplies right}
                            Let \(A_1\limplies B_1,\ldots,A_n\limplies B_n\in\Lambda\) with
                            \(\limplies\) as the outermost logical symbol.
                            Then the following push operation can be performed:
                            \begin{equation}
                                \begin{prooftree}
                                    \tag{reduction:\(\limplies\):right}
                                    \label{eq:reduction:limplies:right}
                                    \hypo{A_1,\ldots,A_n,\Pi\seq\Lambda, B_1, \ldots, B_n}
                                    \infer1{\Pi\seq\Lambda}
                                \end{prooftree}
                            \end{equation}
                        \item\label{it:reduction tree forall left}
                            Let \(\forall x_1A_1(x_1),\ldots,\forall x_nA_n(x_n)\in\Pi\)
                            with \(\forall\) as the outermost logical symbol,
                            and let \(a_i\) be an independent variable that has not been used in
                            reductions of \(\forall x_iA_i(x_i) : i \in\set{1,\ldots,n}\)
                            at this branch height.
                            \begin{equation}
                                \begin{prooftree}
                                    \tag{reduction:\(\forall\):left}
                                    \label{eq:reduction:forall:left}
                                    \hypo{A_1(a_1),\ldots,A_n(a_n),\Pi\seq\Lambda}
                                    \infer1{\Pi\seq\Lambda}
                                \end{prooftree}
                            \end{equation}
                        \item\label{it:reduction tree forall right}
                            Let \(\forall x_1A_1(x_1),\ldots,\forall x_nA_n(x_n)\in\Lambda\)
                            with \(\forall\) as the outermost logical symbol,
                            and let \(a_1,\ldots,a_n\) be the first \(n\) independent
                            variables not available at this branch height.
                            \begin{equation}
                                \begin{prooftree}
                                    \tag{reduction:\(\forall\):right}
                                    \label{eq:reduction:forall:right}
                                    \hypo{\Pi\seq\Lambda,A_1(a_1),\ldots,A_n(a_n)}
                                    \infer1{\Pi\seq\Lambda}
                                \end{prooftree}
                            \end{equation}
                        \item\label{it:reduction tree exists left}
                            Let \(\exists x_1A_1(x_1),\ldots,\exists x_nA_n(x_n)\in\Pi\)
                            with \(\exists\) as the outermost logical symbol,
                            and let \(a_1,\ldots,a_n\) be the first \(n\) independent
                            variables not available at this branch height.
                            \begin{equation}
                                \begin{prooftree}
                                    \tag{reduction:\(\exists\):left}
                                    \label{eq:reduction:exists:left}
                                    \hypo{A_1(a_1),\ldots,A_n(a_n),\Pi\seq\Lambda}
                                    \infer1{\Pi\seq\Lambda}
                                \end{prooftree}
                            \end{equation}
                        \item\label{it:reduction tree exists right}
                            Let \(\exists x_1A_1(x_1),\ldots,\exists x_nA_n(x_n)\in\Lambda\)
                            with \(\exists\) as the outermost logical symbol,
                            and let \(a_i\) be the first independent variable that has not
                            been used in reductions of
                            \(\exists x_iA_i(x_i) : i \in\set{1,\ldots,n}\)
                            at this branch height.
                            \begin{equation}
                                \begin{prooftree}
                                    \tag{reduction:\(\exists\):right}
                                    \label{eq:reduction:exists:right}
                                    \hypo{\Pi\seq\Lambda,A_1(a_1),\ldots,A_n(a_n)}
                                    \infer1{\Pi\seq\Lambda}
                                \end{prooftree}
                            \end{equation}
                        \item\label{it:reduction tree else}
                            If \(\Pi\) and \(\Lambda\) have \emph{any} formulae in common,
                            push the \emph{halting symbol} \halt{} onto the branch:
                            \begin{equation}
                                \begin{prooftree}
                                    \tag{reduction:halt}
                                    \label{eq:reduction:halt}
                                    \hypo{\halt}
                                    \infer1{\Pi\seq\Lambda}
                                \end{prooftree}
                            \end{equation}
                            If the sequences \(\Pi\) and \(\Lambda\) have \emph{nothing}
                            in common, perform the following push operation:
                            \begin{equation}
                                \begin{prooftree}
                                    \tag{reduction:else}
                                    \label{eq:reduction:else}
                                    \hypo{\Pi\seq\Lambda}
                                    \infer1{\Pi\seq\Lambda}
                                \end{prooftree}
                            \end{equation}
                    \end{enumerate}
            \end{enumerate}
            If there are unhandled leaves still left in this iteration loop,
            continue iterating over the unhandled leaves, else jump
            to~\ref{it:reduction tree loop}.
    \end{enumerate}
\end{definition}

Note that in the construction of a reduction tree,
the case~\eqref{eq:reduction:else} provides a possibility
for the formation of an infinite loop. If this happens and
some branch simply keeps growing indefinitely,
as new sequents \(\Pi\seq\Lambda\) keep getting pushed on top of it,
the sequent \(\Gamma\seq\Delta\) at the root of the tree
is not provable.

\begin{example}\label{exa:reduction tree}
    We construct a reduction tree for the sequent
    \(A\limplies B\seq\lnot A\lor B\).
    \begin{equation*}
        \begin{prooftree}
            \hypo{\halt}
            \infer1{B, A\limplies B\seq\lnot A\lor B, \lnot A, B}
            \hypo{\halt}
            \infer1{A, A\limplies B\seq\lnot A\lor B, \lnot A, B, A}
            \infer1[\eqref{eq:reduction:lnot:right}]{A\limplies B\seq\lnot A\lor B, \lnot A, B, A}
            \infer2[\eqref{eq:reduction:limplies:left}]{A\limplies B\seq\lnot A\lor B, \lnot A, B}
            \infer1[\eqref{eq:reduction:lor:right}]{A\limplies B\seq\lnot A\lor B}
        \end{prooftree}
    \end{equation*}
    As both of the branches of the tree halt,
    the sequent is provable.
\end{example}

\begin{exercise}[8.3.1]\label{exe:8.3.1}
    Construct a reduction tree for the sequent
    \(\lnot\exists xF(x)\seq \forall y\lnot F(y)\) in \LK{}.
\end{exercise}

\begin{exercise}[8.3.2]\label{exe:8.3.2}
    Construct a reduction tree for the sequent
    \(\exists x\arcs{A(x)\limplies B(x)}\seq \forall xA(x)\limplies\exists xB(x)\) in \LK{}.
\end{exercise}

\begin{example}[Infinite reduction tree]\label{exa:infinite reduction tree}
    Let us observe the sequent \(\forall x\exists yA(x,y)\seq\) and its reduction tree:
\begin{equation*}
    \begin{prooftree}
        \hypo{\infty}
        \ellipsis{\eqref{eq:reduction:else}}{A(a,b),\exists yA(a,y),\forall x\exists yA(x,y)\seq}
        \infer1[\eqref{eq:reduction:else}]{A(a,b),\exists yA(a,y),\forall x\exists yA(x,y)\seq}
        \infer1[\eqref{eq:reduction:else}]{A(a,b),\exists yA(a,y),\forall x\exists yA(x,y)\seq}
        \infer1[\eqref{eq:reduction:exists:left}]{\exists yA(a,y), \forall x\exists yA(x,y)\seq}
        \infer1[\eqref{eq:reduction:forall:left}]{\forall x\exists yA(x,y)\seq}
    \end{prooftree}
\end{equation*}
Therefore the contradiction at the root of the tree cannot be proven.
\end{example}

Equipped with this new tool, we may now show that validity implies provability.

\begin{theorem}[Completeness of \LK{}: validity implies provability]\label{the:validity implies provability}
    If a formula \(A\) is valid in \LK, then it is provable in \LK.
    In other words, \LK{} is a \emph{complete} logical system.
\end{theorem}

\begin{proof}
    Let \(\Gamma\seq\Delta\) be an \emph{unprovable} sequent in \LK{}.
    We construct an interpretation \(\interp=\tuple{\tuple{D,\phi},\phi_0}\),
    which does \emph{not} satisfy \(\Gamma\seq\Delta\).

    By the first assumption, there is an infinite branch in
    the reduction tree \(T\arcs{\Gamma\seq\Delta}\).
    If \(\Gamma_i,\Delta_i\in T\arcs{\Gamma\seq\Delta}\) and we define
    \begin{equation*}
        \cup\Gamma = \set{A \mid A\in\Gamma_i}
        \quad\text{and}\quad
        \cup\Delta = \set{B \mid B\in\Delta_i}
        \,,
    \end{equation*}
    then there can be no predicate constants \(R\arcs{\ldots}\) nor
    free independent variables \(a_i\), so that
    \begin{equation*}
        R\arcs{a_1,\ldots,a_n}\in \cup\Gamma\cap\cup\Delta,
    \end{equation*}
    as the sets \(\cup\Gamma\) and \(\cup\Delta\) are separate.

    For simplicity, we assume that there are no individual
    function constants \(f\) in our language \(L\), so the only
    free individual variables \(a\) are the only terms of \(L\).
    We can then define our interpretation
    \(\interp=\tuple{\tuple{D,\phi},\phi_0}\) as follows:
    we let \(D = \set{a_0,a_1,\ldots}\) and define the relation
    \begin{equation*}
        \phi = \left\{
            \begin{aligned}
                &\arcs{a_1,\ldots,a_n}\in\phi(R) \iff R(a_1,\ldots,a_n)\in\cup\Gamma, \\
                &\phi(a) = \phi_0(a) = a \\
                &\phi(x) = \phi_0(x) = x \\
            \end{aligned}
        \right.\,.
    \end{equation*}
    Notice that if \(\arcs{a_1,\ldots,a_n}\in\phi(R)\),
    then \( R(a_1,\ldots,a_n)\notin\cup\Delta\).
    Also, if \(R(a_1,\ldots,a_n)\in\cup\Delta\),
    then \( R(a_1,\ldots,a_n)\notin\cup\Gamma\) and
    \(\arcs{a_1,\ldots,a_n}\notin\phi(R)\).

    We now induce on the number of logical symbols \(n\) in a formula \(A\),
    and show that the intrepretation \(\interp\) defined above
    satisfies all formulae in \(\cup\Gamma\),
    but none in \(\cup\Delta\). In the base case \(n=0\),
    so \(A\) is a predicate constant \(R\). By definition,
    any predicate constant \(R\arcs{a_1,\ldots,a_n}\) is
    in \(\cup\Gamma\) and satisfied by \(\interp\),
    whereas any \(R\arcs{a_1,\ldots,a_n}\in\cup\Delta\)
    is not satisfied. The claim then holds for \(n=0\).

    We then make the inductive hypothesis \IH{} that the interpretation
    \(\interp\) satisfies all logical formulae \(A\in\cup\Gamma\) with \(n\leq k\)
    logical symbols, whereas it does not satisfy any formulae \(B\in\cup\Delta\)
    with the same constraint on \(n\). In the inductive step,
    we once again add one logical symbol at a time, and show
    that satisfiability is preserved.

    If \(A = \lnot C\in\cup\Gamma\), then it appears in \(\Gamma_i\)
    during some build step \(i\), and will remain there from then on.
    However, it will never appear on the right in \(\cup\lambda\),
    because by definition, the reduction tree would then \halt{},
    which contradicts our assumptions.
    We now produce the formula \(C\in\cup\Delta\)
    by applying \eqref{eq:reduction:lnot:left}.
    According to \IH{}, \(\interp\) does not satisfy \(C\in\cup\Delta\),
    meaning \(\lnot C\in\cup\Gamma\) is satisfied.
    The case where \(A = \lnot C\in\cup\Delta\)
    is entirely symmetric to this proof, so induction is closed
    on the symbol \(\lnot\).

    Now let \(A = C\land D\in\cup\Gamma\). This again means that
    after some build step \(i\) the formula \(C\land D\) appears
    in the reduction tree, and we can apply \eqref{eq:reduction:land:left}
    to produce the sequence \(C,D\).
    Now according to the \IH{}, the individual formulae \(C\)
    and \(D\) are satisfied by \(\interp\), meaning that
    \(C\land D\) is satisfied by definition.

    If \(A = C\land D\in\cup\Delta\), it will appear on the right
    side of the reduction tree after some build step \(i\),
    and we can use \eqref{eq:reduction:land:right} to produce \(C\) and \(D\).
    According to \IH{}, \(\interp\) satisfies neither \(C\) nor \(D\).
    Again, this leads to \(C\land D\in\cup\Delta\) \emph{not} being satisfied
    by definition, closing the induction on the symbol \(\land\).

    \begin{exercise}[8.3.3]\label{exe:8.3.3}
        Show that the inductive step maintains
        the satisfiability in  the interpretation \(\interp\)
        for the symbol \(\lor\).
    \end{exercise}

    If \(A = C\limplies D\in\cup\Delta\), then by the inductive hypothesis
    \(\interp\) does not satisfy \(D\), but satisfies \(C\).
    Therefore \(C\limplies D\) is not satisfied in \(\interp\).
    A more complex case is when \(A = C\limplies D\in\cup\Gamma\),
    as \eqref{eq:reduction:limplies:left} causes a split in the reduction tree:
    \begin{enumerate}
        \item
            If the tree does not \halt{} on the right,
            then \(C\in\cup\Delta\) is not satisfied in \(\interp\),
            and \(C\limplies D\in\cup\Gamma\) must be satisfied.
        \item
            If the tree does not \halt{} on the left,
            then \(D\in\cup\Gamma\) is satisfied and \(C\limplies D\in\cup\Gamma\)
            must be satisfied by definition.
    \end{enumerate}
    This closes the induction on the symbol \(\limplies\).

    Assuming that \(A = \forall xC(x)\in\cup\Gamma\),
    at some point we may apply \eqref{eq:reduction:forall:left}
    to produce \(C(a)\in\cup\Gamma\). By the inductive hypothesis,
    \(C(a)\) is satisfied by \(\interp\) for any assignment \(\phi_0\),
    and therefore \(\forall xC(x)\in\cup\Gamma\) is satisfied in
    \(\interp\) by definition.

    To save page space, the rest of the cases are omitted,
    as they are \emph{very} similar and based
    entirely on the definitions. This closes the induction.
\end{proof}

Together with theorem~\ref{the:provability implies validity},
theorem~\ref{the:validity implies provability}
shows that \LK{} is complete.
