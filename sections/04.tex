\section{\texorpdfstring{Axiom systems based on \LK}{Axiom systems based on LK}}%
\label{sec.axiom-systems-based-on-lk}

Let \(\sset A = \set{A_1,\ldots,A_n}\) or \(\sset A = \set{A_1,A_2,\ldots}\)
be respectively finite or infinite sets of formulae.
An \emph{axiom system} in classical logic \LK, denoted \(\LKsys[\sset A]\),
is the set of theorems provable from the axioms formed from the formulae in \(\sset A\),
using the inference rules of definition~\ref{def:inference rules in LK}.
Such an axiom is either a sequent \mbox{\(A\seq A\)},
or a single formula \(A_i\), for all formulae \(A,A_i\in\sset A\).

\begin{definition}[Consistency]\label{def:consistency}
    An axiom system \(\LKsys\) is \emph{consistent}, if the empty sequent \(\seq\)
    is \emph{not} provable in \(\LKsys\). The system \(\LKsys\) is \emph{inconsistent}
    if the empty sequent \emph{can} be derived in it.
\end{definition}

In other words, consistent axiom systems cannot produce contradictions,
as per definition~\ref{def:contradiction and provability}.
The following proposition concerns equivalencies in inconsistent systems.

\begin{proposition}[Equivalencies in inconsistent systems]%
\label{prop:equivalencies in inconsistent systems}
The following are equivalent statements:
\begin{enumerate}[label={(\alph*)}]
    \item\label{it:equivalencies in inconsistent systems 1}
        \(\LKsys\) is inconsistent,
    \item\label{it:equivalencies in inconsistent systems 2}
        \(\seq A\) is provable for all formulae \(A\) and
    \item\label{it:equivalencies in inconsistent systems 3}
        for some \(A\), both \(\seq A\) and \(\seq \lnot A\) are provable in \(\LKsys\).
\end{enumerate}
\end{proposition}

\begin{proof}
\item[\ref{it:equivalencies in inconsistent systems 1}\(\implies\)\ref{it:equivalencies in inconsistent systems 2}]
    If the empty sequent \(\seq\) is provable, then it can be used as a leaf of a proof.
    Therefore the following argument can be made:
    \begin{prooftree}
        \hypo{\seq}
        \infer1[\eqref{eq:weakening:right}]{\seq A}
    \end{prooftree}\,.
\item[\ref{it:equivalencies in inconsistent systems 2}\(\implies\)\ref{it:equivalencies in inconsistent systems 3}]
    If \(\seq A\) is provable for all \(A\),
    we can set \(A = B\) and \(A = \lnot C\) for some \(B\) and \(C\),
    so that \(\seq B\) and \(\seq\lnot C\) are provable.
\item[\ref{it:equivalencies in inconsistent systems 3}\(\implies\)\ref{it:equivalencies in inconsistent systems 1}]
    By assumption, there exists an \(A\), so that
    \begin{equation*}
        \begin{prooftree}
            \hypo{\seq\lnot A}
            \hypo{\seq A}
            \infer1[\eqref{eq:lnot:left}]{\lnot A\seq}
            \infer2[\eqref{eq:cut}]{\seq}
        \end{prooftree}
    \end{equation*}
    Therefore the system \(\LKsys\) in which this proof was made is inconsistent
    by definition~\ref{def:consistency}.
\end{proof}
