\section{The incompleteness of Peano arithmetic}%
\label{sec:incompleteness-of-peano-arithmetic}

We now get to the actual meat of the course. We wish to show that Peano
arithmetic \PA{} fulfills the following definition.

\begin{definition}[Incompleteness of a logical system]\label{def:incompleteness}
    An axiom system \(S\) is \emph{incomplete}, if it does not contain proofs
    for \(\seq A\) nor \(\seq\lnot A\) for some sentence \(A\).
\end{definition}

In fact, we wish to show that any system \(S\) that contains Peano arithmetic
(is its superset) ends up being incomplete because of it. To do this, we need
several different definitions and results. The first one of these,
\emph{primitive recursive functions}, works on the metalanguage level.

\begin{definition}[Primitive recursive functions]\label{def:PRF}
    Primitive recursive functions or \PRF{s} is the smallest
    class of functions \(f\) generated by the following function
    skeletons or schemata:
    \begin{align}
        \tag{PRF:successor}\label{prf:successor}
        f(x) &= x', \\
        \intertext{where \(\prime\) is the successor function,}
        \tag{PRF:constant}\label{prf:constant}
        f(x_1,\ldots,x_n) &= k,\quad n\geq 1,\quad k\in\Nset, \\
        \tag{PRF:projection}\label{prf:projection}
        f(x_1,\ldots,x_n) &= x_i,\quad i\in\set{1,\ldots,n} \\
        \tag{PRF:compound}\label{prf:compound}
        f(x_1,\ldots,x_n) &= g\arcs[\big]{
            h_1(x_1.\ldots,x_n),\ldots,h_m(x_1.\ldots,x_n)
        } \intertext{where \(g\) and \(h_1\)..\(h_m\) are \PRF{s},}
        \tag{PRF:simple recursion}\label{prf:simple recursive}
        f(0) &= k,\quad f(x') = g(x, f(x))
        \intertext{where \(k\in\Nset\) and \(g\) is a \PRF, and finally}
        \tag{PRF:compound recursion}\label{prf:compound recursive}
        f(0, x_2,\ldots,x_n) &=g(x_2,\ldots,x_n),\\
        \nonumber
        f(x', x_2,\ldots,x_n) &= h\arcs[\big]{x,f(x_2,\ldots,x_n),x_2,\ldots,x_n},
    \end{align}
    where \(g\) and \(h\) are \PRF{}s.
\end{definition}

The fact that \PRF{s} work on the metalanguage level means that
they are actual functions in the usual sense, not just function symbols.
The \PRF{}s of definition \ref{def:PRF} are used to build other functions recursively,
and they function as the base cases of this process.
There is also a concept strongly related to \PRF{s},
\emph{primitive recursive relations}, that we must discuss.

\begin{definition}[Primitive recursive relation]\label{def:primitive recursive relation}
    A relation \(R\) of length \(n\) is said to be \emph{primitive recursive},
    if there is a characteristic \PRF{} \(f: X\to\set{0,1}\),
    and \(f(a_1,\ldots,a_n) = 0\), if and only if
    \(R(a_1,\ldots,a_n)\) is true or \((a_1,\ldots,a_n)\in R\).
\end{definition}

\begin{example}[I]\label{exa:PRF successor}
    By definition, the successor function \('\) is
    a \PRF{}: \(0'=1, 1'=2, 2'=3,\ldots,n'=n+1,\ldots\).
\end{example}

\begin{example}[II]\label{exa:PRF sum}
    The addition of natural numbers is
    of type \eqref{prf:compound recursive}:
    \begin{itemize}
        \item \(f(0,n) = p_2(0,n) = n\), where \(p_2\)
            is a \PRF{} of type \eqref{prf:projection}.
        \item
            Set \(h = k\circ p_2\) and
            \(k\circ p_2(m,f(m,n,n)) = k(f(m,n)) = f(m,n)'\).
            Notice that \(h\) is of type \eqref{prf:compound}
            and \(k\) of type \eqref{prf:successor}.
    \end{itemize}
    Now set \(f(m',n) = h(m, f(m,n),n) = f(m,n)'\).
    Then \(m+n = f(m,n)\).
\end{example}

For further clarification of example \ref{exa:PRF sum}, consider
\begin{align*}
    f(5,9) &= h(4,f(4,9),9) = f(4,9)' = 14, \\
    f(4,9) &= h(3,f(3,9),9) = f(3,9)' = 13, \\
    f(3,9) &= h(2,f(2,9),9) = f(2,9)' = 12, \\
    f(2,9) &= h(1,f(1,9),9) = f(1,9)' = 11, \\
    f(1,9) &= h(0,f(0,9),9) = f(0,9)' = 10 \,.
\end{align*}

\begin{example}[III]\label{exa:PRF constant}
    Constant functions \(v_0(n)\equiv 0\),
    \(v_1(n)\equiv 1\), \(v_2(n)\equiv 2\) and so forth
    are of type \eqref{prf:constant}.
\end{example}

\begin{exercise}[10.2.1]\label{exe:10.2.1}
    Show that multiplication of natural numbers \(\times\)
    is a primitive recursive function.
\end{exercise}

\begin{example}[V]\label{exa:PRF predecessor}
    Predecessor or \emph{Vorgänger} functions
    are of type \eqref{prf:simple recursive}:
    \begin{enumerate}
        \item First define \(\PRED(0) = v_0(0) = 0\),
            where \(v_0\) is of type \eqref{prf:constant}.
        \item Then set \(\PRED(n') = p_1(n,\PRED(n)) = n\),
            where \(p_1\) is of type \eqref{prf:projection}.
    \end{enumerate}
    In other words,
    \begin{equation*}
        \PRED(n) =
        \left\{
            \begin{aligned}
                0&, \quad n=0\\
                n-1&,\quad n>0
            \end{aligned}
        \right.\,.
    \end{equation*}
\end{example}

\begin{example}[VI]\label{exa:PRF sign}
    The \emph{signum} function is of type \eqref{prf:simple recursive}:
    \begin{enumerate}
        \item First define \(\SIGN(0) = v_0(0)  = 0\),
            where \(v_0\) is of type \eqref{prf:constant}.
        \item Then set \[\SIGN(n') = v_1\circ p_2(n,\SIGN(n))
            = v_1(p_2(n,\SIGN(n))) = v_1(\SIGN(n)) = 1\,.\]
            Note that \(v_1\) is of type \eqref{prf:constant},
            \(p_2\) of type \eqref{prf:projection}
            and \(v_1\circ p_2\) of type \eqref{prf:compound}.
    \end{enumerate}
\end{example}

A concrete example of the application of the signum function
is seen below:
\begin{align*}
    \SIGN(1) &= v_1(\SIGN(0)) = v_1(0) = 1 \\
    \SIGN(2) &= v_1(\SIGN(1)) = v_1(1) = 1 \\
             &\vdots \\
    \SIGN(n) &= 1, \quad n\in\Nset\setminus\set0\,.
\end{align*}

\begin{example}[VII]\label{exa:PRF permutation}
    Given a \PRF{} \(f(m,n)\), the \emph{permutation} of
    \(f\), \(\PERM(f(m,n)) = f(n,m)\), is of type
    \eqref{prf:compound recursive}. It can be constructed by setting
    \begin{equation*}
        \PERM(f(m,n)) = f(p_2(m,n), p_1(m,n)) = f(n,m)\,.
    \end{equation*}
    Here \(p_1\) and \(p_2\) are of type \eqref{prf:projection}.
\end{example}

\begin{example}[VIII]\label{exa:PRF cut of subtraction}
    A so calle \emph{cut of subtraction} \(\CS\)
    is of type \eqref{prf:compound recursive}.
    It is generally defined as
    \begin{equation*}
        \CS(m,n) = \left\{
            \begin{aligned}
                0&,\quad m > n, \\
                n-m&,\quad m\leq n
            \end{aligned}\,.
        \right.
    \end{equation*}
    It is defined \emph{recursively} for \(m,n\in\Nset\) by setting
    \begin{align*}
        \CS(0,n)  &= p_1(n) = n\\
        \CS(m',n) &= \PRED(p_2(m,\CS(m,n),n)) = \PRED(\CS(m,n))\,,
    \end{align*}
    where \(p_1\) is of type \eqref{prf:projection} and \(\PRED\)
    of type \eqref{prf:compound recursive}.
\end{example}

An example of the application of cut of subtraction goes as follows:
\begin{align*}
    \CS(1,n)    &= \PRED(\CS(0,n)) = n-1 \\
    \CS(2,n)    &= \PRED(\CS(1,n)) = \PRED(n-1) = n-2 \\
                &\vdots \\
    \CS(n-1,n)  &= \PRED(n-(n-2)) = \PRED(2) = 1 \\
    \CS(n,n)  &= \PRED(n-(n-1)) = \PRED(1) = 0 \,.
\end{align*}
Finally, \(\CS(m,n) = 0\) for all \(m > n\).

\begin{example}[IX]\label{exa:PRF absolute difference}
    The \emph{absolute difference} \(\ADF\) function is
    of type \eqref{prf:compound}:
    \begin{equation*}
        \ADF(m,n) = \CS(m,n) + \CS(n,m) = \abs{m-n}\,.
    \end{equation*}
    Here \(\CS(m,n)\) is either \(n-m\) or \(0\),
    and \(\CS(n,m)\) either \(m-n\) or \(0\).
    As stated above, both \(\CS\) and \(+\)
    are \PRF{}s.
\end{example}

We now proceed to show that equality \(=\) is a primitive recursive
relation \(\PRR\) in \(\Nset^2\). Define the function \(f(m,n) = \SIGN(\ADF(m,n))\),
which is of type~\eqref{prf:compound} and
\begin{equation*}
    f(m,n) = \SIGN(\abs{m-n}) = \left\{
        \begin{aligned}
            0&,\quad m = n, \\
            1&,\quad m \neq n\,.
        \end{aligned}
    \right.
\end{equation*}
There then exists a (characteristic) \PRF{} for the relation \(=\),
whose values are in the set \(\set{0,1}\). Therefore the
relation \(=\) is a \PRR, as per definition~\ref{def:primitive recursive relation}.

\paragraph{Note:}
We say that \(m = n\) or \(m,n\in=\) (\(m\) and \(n\) are in an equality relation),
if and only if the above \PRF{} \(f(m,n) = 0\).
This is usually how equality is defined in computer algebra systems
such as \emph{Maxima}. This is then of immediate practical use as well!

\paragraph{Note 2:}
The relation \(<\) is also a \PRR{} in \(\Nset^2\):
\(m < n\), if and only if \(\CS(m,n) = 0\).

\subsection{General primitive recursive relations}

Assume \(R\) is a primitive recursive relation \PRR.
Now either the variables \(x_1,\ldots,x_n\)
are in \(R\), as in \(R(x_1,\ldots,x_n)\), or they are not.
Because \(R\) is a \PRR, there exists a characteristic function
of \(R\), \(f_R\), so that
\begin{equation*}
    f_R(x_1,\ldots,x_n) = \left\{
        \begin{aligned}
            0&,\quad (x_1,\ldots,x_n)\in R, \\
            1&,\quad (x_1,\ldots,x_n)\notin R\,.
        \end{aligned}
    \right.
\end{equation*}

\begin{lemma}(Consistency of \PA)\label{lem:10.5}
    Peano arithmetic \PA{} (and all of its supersets) are \emph{consistent},
    if and only if the sequent \(\seq\bar0=\bar1\)
    is \emph{not} provable in \PA. In other words,
    any formula \(A\) is provable, if \(\seq\bar0=\bar1\)
    is provable in \PA{}.
\end{lemma}

\begin{proof}
    Omitted.
\end{proof}

\begin{proposition}[10.6.1]\label{prop:10.6.1}
    Primitive recursive functions \PRF{} can be expressed
    in \LN{}, the language of \PA{}. More generally:

    \begin{tabenv}{Correspondence of \PRF{}s and \(\LN\).}
        \begin{tabular}{c|c|c}
            \toprule
            \(\Nset\) &Metalevel \(\Nset\) & \(\LN\) of \PA{}\\
            \midrule
            1. &Natural numbers \(n\in\Nset\)   & Numerals \(\bar n\in\LN\)\\
            2. &\PRF{s} \(f : \Nset^n\to\Nset\)   & \(n\)-ary function symbols \(\bar f\) \\
            3. &\PRR{s} \(R\subseteq \Nset^n\)  & \(n\)-ary atomic formulae or predicates \(\bar R\)\\
            \bottomrule
        \end{tabular}
    \end{tabenv}
\end{proposition}

\begin{proof}[A sketch of the proof]
\item[1.] This should be clear by declaration.

\item[2.]
    In the set \(\Nset\), primitive recursive functions can be expressed
    in terms of the successor \('\), addition \(+\) and multiplication \(\times\).
    These correspond to the function symbols \(\bar'\), \(\bar+\) and \(\bar\times\)
    in \(\LN\). In particular,
    \begin{equation*}
        f(m_1,\ldots,m_n)   = p\in\Nset,
        \quad\text{if and only if}\quad
        \bar f(m_1,\ldots,m_n) = \bar p \in\LN\,.
    \end{equation*}
\item[3.]
    First notice that \(=\arcs{,\bar m}\) or \(=\bar m\)
    can be seen as a right-associative unary predicate symbol
    \(\bar R_{\bar m}\), and that \(\bar R_{\bar m}(t)\)
    is an atomic formula in \PA{}. Moreover, either
    \(\seq\bar R_{\bar m}(t)\) (\(\seq t=\bar m\)),
    or \(\bar R_{\bar m}(t)\seq\) (\(t=\bar m\seq\))
    is provable by proposition~\ref{prop:9.6.2},
    as \(t\) is a closed term.

    Then recall that \(m_1,\ldots,m_n\in\Nset\) either are in relation
    \(R\subseteq\Nset^n\), or they are not, if and only if \(R(m_1,\ldots,m_n)\)
    holds in \(\Nset^n\), if and only if
    \begin{equation*}
        f_R(m_1,\ldots,m_n) = \left\{
            \begin{aligned}
                0&,\quad m_1,\ldots,m_n \in R,\\
                1&,\quad m_1,\ldots,m_n\notin R\,,
            \end{aligned}
        \right.
    \end{equation*}
    where \(f_R\) is the characteristic function of \(R\), a \PRF{}.
    We then have the following: \(R(m_1,\ldots,m_n)\) holds in \(\Nset^n\),
    if and only if \(f_R(m_1,\ldots,m_n) = 0\), if and only if
    \begin{equation*}
        \seq f_R(m_1,\ldots,m_n) = 0
    \end{equation*}
    is provable in \(\PA\) and its supersets.

    We denote \(f_R(t_1,\ldots,t_n) = \bar0\) by a \emph{unary} predicate
    \(R_\delta(t_1,\ldots,t_n)\). Then either \(\seq R_\delta(t_1,\ldots,t_n)\)
    is provable is \PA{}, or it is not. In particular,
    \(\seq R_\delta(m_1,\ldots,m_n)\) in \PA{}.
\end{proof}

The converse of proposition~\ref{prop:10.6.1} also holds.
This we prove as follows.

\begin{proof}
    By lemma \ref{lem:10.5} the consistency of \PA{}
    can be expressed by saying that \(\seq\bar1=\bar0\) is not provable in \PA{}.
    If the sequent \(\seq\bar R(\bar m_1,\ldots,\bar m_n)\)
    is provable in \PA{}, then the relation \(R(m_1,\ldots,m_n)\)
    holds in \(\Nset^n\), as in \(m_1,\ldots,m_n\in R\).
    We make the contrapositive assumption that \(m_1,\ldots,m_n\notin R\).

    Now the characteristic function \(f_R(m_1,\ldots,m_n) = 1\),
    which when expressed in \PA{} means that
    \(\bar f_R(\bar m_1,\ldots,\bar m_n) = \bar1\),
    or simply \(\bar f_R = \bar1\).
    Since \(\seq\bar R(\bar m_1,\ldots,\bar m_n)\) in \PA{},
    \(\seq\bar f_R =\bar0\) must be provable. Now
\begin{equation*}
    \begin{prooftree}
        \hypo{\seq\bar f_R=\bar1}
        \hypo{\bar f_R=\bar1\seq\bar1=\bar f_R}
        \infer2[\eqref{eq:cut}]{\seq\bar1=\bar f_R}
    \end{prooftree}
\end{equation*}
and therefore
\begin{equation*}
    \begin{prooftree}
        \hypo{\seq\bar f_R=\bar0}
        \hypo{\seq\bar1=\bar f_R}
        \hypo{\bar1=\bar f, \bar f=\bar0\seq\bar1=\bar0}
        \infer2[\eqref{eq:cut}]{\bar f=\bar0\seq\bar1=\bar0}
        \infer2[\eqref{eq:cut}]{\seq\bar1=\bar0}
    \end{prooftree}
\end{equation*}
This makes \PA{} inconsistent, meaning \(R(m_1,\ldots,m_n)\) holds in \(\Nset^n\).
\end{proof}

\subsection{Gödel numbering}

Recall the prime number factorization of natural numbers:
any natural number \(n\geq 2\) can be expressed \emph{uniquely}
as the finite product of primes:
\begin{equation*}
    n = 2^{p_1}\times 3^{p_2}\times 5^{p_3}\times\cdots\times p^{p_m},
\end{equation*}
where the numbers \(2\), \(3\), \(5\), \ldots, \(p\) are the first \(m\)
prime numbers and \(p_1,p_2,p_3,\ldots,p_m \in\Nset\), with \(p_m\neq 0\).
For example,
\begin{alignat*}{4}
    &2 = 2^1,           &&\quad3 = 2^0\times3^1,                 &&\quad4 = 2^2, &&\quad5 = 2^0\times 3^0\times 5^1, \\
    &6 = 2^1\times 3^1, &&\quad7= 2^0\times3^0\times5^0\times7^1,&&\quad8 = 2^3, &&\quad9 = 2^0\times3^2,\ldots
\end{alignat*}
We now make the following definition based on this knowledge.

\begin{definition}[Gödel numbering]\label{def:gödel numbering}
    The formulae of \PA{} are composed of a finite set of symbols
    \(\set{x,y,\limplies,\land,\lor\lnot,\exists,\ldots,\forall}\),
    so they can be enumerated as follows:

    \begin{tabenv}{An enumeration of symbols in \PA{}}
        \label{tab:an enumeration of symbols in PA}
        \begin{tabular}[]{c|C|C|C|C|C|C|C|C|C|C|C|C|C}
            \toprule
            \(\Nset\) & 1 & 2 & 3 & 4 & 5 & 6 & 7 & 8 & 9 & 10  & 11 & \cdots & n\\
            \midrule
            Symbol  & = & \bar0 & \exists & x & \prime & ( & ) & + & \seq & ,  & -- & \cdots & \forall\\
            \bottomrule
        \end{tabular}
    \end{tabenv}

    Based on this enumeration, \emph{which is not unique},
    we can write the sentence ''\(\bar0\) has a successor'',
    expressed in \PA{} as \(A:\exists x(x=0')\), as the sequence
    \(34641257\), based on table~\ref{tab:an enumeration of symbols in PA}.
    The unique (in enumeration~\ref{tab:an enumeration of symbols in PA})
    \emph{Gödel number} of \(A\), \(G(A)\), is then
    \begin{equation*}
        G(A) = \Gnum A = 2^3\times3^4\times5^6\times7^4\times11^1\times13^2\times17^5\times19^7 \,.
    \end{equation*}
\end{definition}

In a given enumeration, each formula \(A\in\LN\) has a unique Gödel number \(\Gnum A\).
On the other hand, given a natural number \(n\in\Nset\), we could in principle
find its prime factorization and hence discover, whether \(n\) is the
Gödel number of \(A\) for some formula \(A\).
For example
\begin{equation*}
    6 = 2^1\times 3^1\,,
\end{equation*}
meaning it corresponds to the string \(==\) in
the enumeration~\ref{tab:an enumeration of symbols in PA}
However, for most \(n\) this is not in any way practical,
because of technical limitations.

\begin{exercise}[Gödel 1]\label{exe:gödel 1}
    What is the Gödel number \(\Gnum A\) for the formula
    \(A: \exists x(x+\bar0 = x)\)? Is the number \(4\,261\,409\,460\)
    the Gödel number of some formula?
\end{exercise}

Next we define the process of constructing Gödel numbers for
sequences of formulae, such as \(\Gamma: A_1,A_2,A_3,\ldots,A_n\).
Here we simply write
\begin{equation*}
    G(\Gamma) = \Gnum\Gamma =
    2^{\Gnum{A_1}}\times3^{\Gnum{A_2}}\times5^{\Gnum{A_3}}\times\cdots\times p^{\Gnum{A_n}} \,.
\end{equation*}
In other words, the prime numbers from \(2\) to \(p\),
where \(p\) is the \(n\)th prime number,
are raised to the Gödel numbers of the individual formulae \(A_i\),
to produce the Gödel number for the sequence \(\Gamma\).

Sequents \(\Gamma\seq\Delta\), inferences \(Q\) of definition~\ref{def:inference}
and proofs \(P\) of definition~\ref{def:proof in LK} can also be given
their own unique Gödel numbers \(\Gnum{\Gamma\seq\Delta}\),
\(\Gnum Q\) and \(\Gnum P\), by converting them to suitable strings of symbols
and applying the procedure above. For example, for the proofs
\begin{equation*}
    P_1:
    \begin{prooftree}
        \hypo{\Gamma\seq\Delta}
        \infer1{S}
    \end{prooftree}
    \quad\text{and}\quad
    P_2:
    \begin{prooftree}
        \hypo{Q_1}
        \hypo{Q_2}
        \infer2{S}
    \end{prooftree}
\end{equation*}
the Gödel numbers are
\[
\Gnum{P_1} = 2^{\Gnum\Gamma}\times3^{\Gnum\seq}\times5^{\Gnum\Delta}\times7^{\Gnum-}\times11^{\Gnum S}
\] and
\[\Gnum{P_2} = 2^{\Gnum Q_1}\times3^{\Gnum Q_2}\times5^{\Gnum-}\times7^{\Gnum S}\]
respectively. Note that \(\seq\) and \(-\) are the symbols for the sequent arrow
and inference line, again respectively.

\paragraph{Note:}
The Gödel number \(\Gnum{\bar n} \neq n\).
For example, \(\bar3 = \bar0'''\)
and therefore in the enumeration of table~\ref{tab:an enumeration of symbols in PA},
\(\Gnum{\bar3} = 2^2\times3^5\times5^5\times7^5\neq3\).

\subsection{Arithmetization of Peano Arithmetic}

We have now given a procedure for \emph{arithmetizing} formulae,
sequences of formulae, sequents and proofs. We now make this more
formal via the following definition.

\begin{definition}[Arithmetization]\label{def:arithmetization}
    If an object, action or operation \(\Phi\) has a Gödel number \(\Gnum\Phi\),
    then it has a unique \emph{arithmetization} which is equal to that number.
\end{definition}

In other words, all objects can be given a counterpart in the set
of natural numbers \(\Nset\). For example, if \(\Phi\) is a predicate
\(\bar R(\bar x_1,\ldots,\bar x_n)\) in \PA{},
the sequent \(\seq\bar R(\bar x_1,\ldots,\bar x_n)\) is provable in \PA{},
\emph{and} there is a \PRR{} \(R\subseteq \Nset^n\)
such that \(R(\Gnum{\bar x_1},\ldots,\Gnum{\bar x_n})\)
holds in \(\Nset^n\), then \(R\) is the arithmetization of \(\bar R\).

Another example of an arithmetization goes as follows:
let \(\Phi\) be an operation applied to objects
\(\bar x_1,\ldots,\bar x_n \in\PA\),
so that the resulting object \(\bar y\in\PA{}\).
If there is a \PRF{} \(f\) in \(\Nset\) so that
\(f(\Gnum{\bar x_1},\ldots,\Gnum{\bar x_n},)) = \Gnum{\bar y}\),
then \(f\) is the arithmetization of \(\Phi\).

Notice that the converse is also true:
if \(a_1,\ldots,a_n,b\in\Nset\) happen to be
Gödel numbers, so that
\begin{equation*}
    a_1 = \Gnum{\bar x_1},\ldots, a_n = \Gnum{\bar x_n}, b = \Gnum{\bar y}\,,
\end{equation*}
then there are unique corresponding numerals of definition~\ref{def:numerals},
\begin{equation*}
    \num{\Gnum{\num x_1}},\ldots,\num{\Gnum{\num x_n}} \in\PA\,.
\end{equation*}
If there also exists a mapping
\(f(\Gnum{\num x_1},\ldots,\Gnum{\num x_n}) = \Gnum{\num y} \in\Nset\),
then there exists an operation \(\Phi\), that performs the mapping
\(\arcs{\num{\Gnum{\num x_1}},\ldots,\num{\Gnum{\num x_n}}}\mapsto\num{\Gnum{\num y}}\in\PA{}\).

\paragraph{Note:}
This converse relation is rather complex, and is not proven
in the book of Takeuti. We therefore also skip the proof. {\Large\trollface}

The proofs of tht following three lemmas are omitted.
They are however stated here for completeness.

\begin{lemma}[10.8.1]\label{lem:10.8.1}
    The rule~\eqref{eq:substitution} can be arithmetized:
    if \(\num x(a_0)\) is a fully indicated expression of \PA,
    and \(\num x(\num y)\) is obtained from \(\num x(a_0)\)
    by substituting all instances of \(a_0\) with \(\num y\)
    in \(\num x\), then there is a unique \PRF{} \(\sub: \Nset^2\to\Nset\),
    such that
    \begin{equation*}
        \sub\arcs*{\Gnum{\num y}, \Gnum{\num x(a_0)}} = \num{\Gnum{\num x(\num y)}}\,,
    \end{equation*}
    and there is a unique function symbol \(\num{\sub{}}\) in \PA{},
    such that \(\seq\sub\arcs*{\Gnum*{\num y}, \Gnum*{\num x(a_0)}} = \num{\Gnum*{\num x(\num y)}}\)
    is provable in \PA{}.
\end{lemma}
\begin{lemma}[10.8.2]\label{lem:10.8.2}
    There is a \PRF{} \(\nu:\Nset\to\Nset\),
    such that \(\nu(m)=\Gnum*{\num m}\). There also exists
    a unique \emph{function symbol} \(\num\nu\in\PA\) (or \(\num{\nu(a)}\)),
    such that the sequent \(\seq\num\nu(\num m) = \num{\Gnum*{\num m}}\)
    is provable in \PA{}.
\end{lemma}

Again, notice the difference between numerals (symbols representing numbers)
and numbers themselves. In lemma~\ref{lem:10.8.2}, \(\nu(m)\neq\num m\).
For example, in the enumeration of table~\ref{tab:an enumeration of symbols in PA},
\(\nu(0) = \Gnum*{\num0} = 4 \neq \num0\) and \(\nu(1) = \Gnum*{\num0'} = 2^2\times3^5 = 972\).
As for the function symbols, notice \(\seq\num\nu(\num0) = \num4\)
and \(\seq\num\nu(\num1) = \num{972}\), and so forth.

\begin{lemma}[10.8.3]\label{lem:10.8.3}
    The proof \(P\) of a sequent \(\Gamma\seq\Delta\)
    can be arithmetized. In other words, there exists a \PRR{}
    \(\Prov\subseteq\Nset^2\), such that \(\Prov(P,a)\) holds
    in \(\Nset^2\), if and only if \(P = \Gnum P\) and \(a = \Gnum{\Gamma\seq\Delta}\).
    There also exists a binary \emph{predicate symbol} \(\num\Prov\in\PA\),
    such that \(\seq\num\Prov(\num{\Gnum P}, \num{\Gnum A})\) is provable in \PA.
\end{lemma}

The lemmae~\ref{lem:10.8.1}--\ref{lem:10.8.3} together show,
that a proof \(P\) can be arithmetized in a pri\-mi\-tive-re\-cur\-sive
manner. Notice, that in the proof of lemma~\ref{lem:10.8.3}
we must assume that \PA{} is axiomatizable, as in the number
of axioms in the system is finite. This is of course the case,
as per the axioms~\ref{axi:logical axioms of PA}--\ref{axi:inductive axiom}.

The question now arises: What else can be arithmetized?
Can \emph{everything} be arithmetized in \PA?

\paragraph{Summary:}
We can effectively go from the formal objects (strings)
of \PA{} to their Gödel numbers in \(\Nset\) and back:
to decide whether a given \(n\in\Nset\) is a Gödel number,
and if it is, which string in \PA{} or its superset does it correspond to.
In particular, we are justified in making the following claim.

\begin{proposition}[10.9.1]\label{prop:10.9.1}
    If \(\seq A\) is provable in \PA,
    so is \(\seq\exists x\Prov(x,\num{\Gnum A})\).
\end{proposition}

\begin{proof}
    If \(P\) is a proof of \(A\), then \(\Prov(\Gnum P, \Gnum A)\)
    holds in \(\Nset^2\) and therefore \(\num\Prov(\Gnum P, \Gnum A)\)
    is provable in \PA. We then have the following proof:
    \begin{equation*}
        \begin{prooftree}
            \hypo{}
            \ellipsis{}{}
            \infer1{\Prov(\num{\Gnum P},\num{\Gnum A})}
            \infer1[\eqref{eq:exists:right}]{\seq\exists x\Prov(x,\num{\Gnum A})}
        \end{prooftree}
    \end{equation*}
\end{proof}

Note that in the above application of \eqref{eq:exists:right},
the term in the formula the inference is applied to can be any term,
and \emph{numerals are terms}. The statement of proposition~\ref{prop:10.9.1},
\(A\seq\exists x\Prov(x,\num{\Gnum A})\),
can be written more succinctly as follows:
\begin{equation*}
    \begin{prooftree}
        \hypo{\seq A}
        \hypo{A\seq\exists x\Prov(x,\num{\Gnum A})}
        \infer2[\eqref{eq:cut}]{\seq\exists x\Prov(x,\num{\Gnum A})}
    \end{prooftree}
\end{equation*}

\begin{proposition}[10.9.2]\label{prop:10.9.2}
    If \(A\equiv B\), then \(\exists x\Prov(x,\num{\Gnum A})\equiv\exists x\Prov(x,\num{\Gnum B})\)
\end{proposition}

\begin{proof}
    If \(A\equiv B\), then necessarily \(A\seq B\) is provable.
    Now if \(\seq A\) is provable in \PA, then the following
    proof \(T\) exists for \(\seq B\) as well:
    \begin{equation*}
        \begin{prooftree}
            \hypo{}
            \ellipsis{}{}
            \hypo{}
            \ellipsis{}{}
            \infer2{\seq A}
            \hypo{}
            \ellipsis{}{}
            \hypo{}
            \ellipsis{}{}
            \infer2{A\seq B}
            \infer2[\eqref{eq:cut}]{\seq B}
        \end{prooftree}
    \end{equation*}
    Therefore \(\Prov(\Gnum T, \Gnum B)\) holds in \(\Nset^2\),
    and
    \begin{equation*}
        \begin{prooftree}
            \hypo{\seq\Prov(\num{\Gnum T}, \num{\Gnum B})}
            \infer1[\eqref{eq:exists:right}]{\seq\exists x\Prov(x, \num{\Gnum B})}
            \infer1[\eqref{eq:weakening:left}]{\exists x\Prov(x, \num{\Gnum A})\seq\exists x\Prov(x, \num{\Gnum B})}
            \infer1[\eqref{eq:limplies:right}]{\seq\exists x\Prov(x, \num{\Gnum A})\limplies\exists x\Prov(x, \num{\Gnum B})}
        \end{prooftree}
    \end{equation*}
    In a very similar manner, the claim
    \begin{equation*}
        \seq\exists x\Prov(x, \num{\Gnum B})\limplies\exists x\Prov(x, \num{\Gnum A})\,.
    \end{equation*}
    can be shown to hold in \PA{}. The wanted result then directly follows.
\end{proof}

We now make the following notational agreement,
in order to make our lives easier:
\begin{equation}\tag{vdash}\label{eq:vdash}
    \vdash\num{\Gnum{A}}
    \quad\text{expands to}\quad
    \exists x\Prov(x, \num{\Gnum A}) \,.
\end{equation}
With this agreement in place, we can proceed to make the following proposition.

\begin{proposition}[10.9.3]\label{prop:10.9.3}
    If \(\seq A\) is provable in \PA{,}
    then so is \(\vdash\num{\Gnum A}\seq\arcs{\vdash\num{\Gnum{\vdash\num{\Gnum{A}}}}}\).
\end{proposition}

\begin{proof}
    We reason as in the proof of proposition~\ref{prop:10.9.1}.
    In the sequent \(B\seq\exists x\num\Prov(x,\num{\Gnum B})\),
    suppose \(B\) is \(\exists x\num\Prov(x,\num{\Gnum A})\).
    More precisely:
    \begin{enumerate}
        \item
            If \(\seq A\) is provable via proof \(P\) in \PA,
            then \(\Prov(x,\Gnum A)\) holds in \(\Nset^2\).
        \item
            If \(\Prov(x,\num{\Gnum A})\) holds in \(\Nset^2\),
            then the following proof \(Q\) can be made in \PA:
            \begin{equation*}
                \begin{prooftree}
                    \hypo{\seq\num\Prov(\num{\Gnum P},\num{\Gnum A})}
                    \infer1[\eqref{eq:exists:right}]{\seq\exists x\Prov(x, \num{\Gnum A})}
                \end{prooftree}\,,
            \end{equation*}
            where the lower sequent is the same thing as
            \(\vdash\num{\Gnum{A}}\), as per \eqref{eq:vdash}.
        \item
            Because the above proof can be made,
            then \(\Prov(\Gnum Q, \Gnum*{\vdash\num{\Gnum{A}}})\)
            holds in \(\Nset^2\), where \(Q\) is the proof above.
        \item
            If \(\Prov(\Gnum Q, \Gnum*{\vdash\num{\Gnum{A}}})\) holds in \(\Nset^2\),
            then the following proof can be made in \PA:
            \begin{equation*}
                \begin{prooftree}
                    \hypo{\seq\Prov\arcs*{\Gnum Q, \Gnum*{\vdash\num{\Gnum{A}}}}}
                    \infer1[\eqref{eq:exists:right}]{\seq\exists y\Prov\arcs*{y, \Gnum*{\vdash\num{\Gnum{A}}}})}
                \end{prooftree}\,,
            \end{equation*}
            where again the lower sequent is the same thing as
            \(\Gnum*{\vdash\num{\Gnum{A}}}\), again as per \eqref{eq:vdash}.
    \end{enumerate}
    This concludes the proof, as we have been able to construct the sequent we wanted,
    by jumping between the spaces \PA{} and \(\Nset^2\).
\end{proof}

\paragraph{Remark:}
The consistency of \PA{} refers to the fact that for \emph{no}
\(A\) in \LN, the statement \(A\equiv\lnot A\) holds.
This can even be proved with the tools we have learned
thus far, but the proof is omitted.


\subsection{Tarski's theorem}

Recall that the function \(\nu(m) = \Gnum{\num m}\) is a \PRF{} in \(\Nset\),
and that a corresponding function constant \(\num\nu\) exists in \PA{},
such that \(\seq\num\nu(m) = \num{\Gnum{\num m}}\). We can also show that
\eqref{eq:substitution} is arithmetizable: there exists a binary function
\(\sub:\Nset^2\to\Nset\), such that
\(\sub(\Gnum{\num y},\Gnum{\num x(a_0)})=\Gnum{\num x(\num y)}\),
and the corresponding function symbol \(\num\sub\) in \PA,
such that \(\sub(m,\Gnum{\nu(m)})=\Gnum{\num m}\) and
\(\seq\num\sub(\num m,\Gnum{\num\nu(m)})=\num{\Gnum{\num m}}\)
is provable in \PA{}.

Now assume that \PA{} contains the unary formula \(T(a_0)\),
with \(a_0\) as the only free variable, such that
\begin{equation}\tag{truth}\label{eq:truth}
   T(\Gnum{\num A})\equiv A
\end{equation}
for all sentences \(A\).
In particular, consider the formula  \(\lnot T(\num\sub(\num a_0, \num\nu(\num a_0)))\).
It is a valid string in \PA{} and therefore it must have a Gödel number \(p\in\Nset\):
\begin{equation*}
    \Gnum*{\lnot T\arcs*{\num\sub\arcs*{\num a_0, \num\nu(\num a_0)}}} = p\in\Nset\,.
\end{equation*}

Now let \(A_T =\Gnum*{\lnot T\arcs*{\num\sub\arcs*{\num p, \num\nu(\num p)}}} =
\lnot T(\num{\Gnum*{\num p}})\). Since \(A_T\) is obtained via
\eqref{eq:substitution}, we have \(\Gnum{A_T} = \Gnum*{\num p}\). But now we
know that \(A_T = \lnot T\arcs{\num{\Gnum{A}}}\), and therefore \(\lnot
A_T\equiv\lnot T\arcs{\num{\Gnum{A}}}\equiv A_T\). This is impossible, as \PA{}
is consistent. We have then proved the following theorem.

\begin{theorem}[Tarski]\label{the:Tarski}
    No consistent axiom system \(S\), where \(\PA\subseteq S\), can contain a
    formula \(T\) such that \eqref{eq:truth} is satisfied.
\end{theorem}

Tarski's theorem has the following \(3\) implications:
\begin{enumerate}
    \item
        The notion of \emph{arithmetical truth} is not arithmetical. If we
        assert that the axioms of \PA{} are true, then by Tarski's theorem we
        obtain that there is no formula \(T(a_0)\) that would make \(A\) true,
        if and only if \(T(\num{\Gnum A})\) true for all formulae \(A\in\LN\).
    \item
        \eqref{eq:truth} cannot be proved inside \PA{}.
    \item
        Validity cannot be arithmetized.
\end{enumerate}


\subsection{Gödel's incompleteness theorems}

At this point, we should remind ourselves of the point of this chapter. To this
end we present the following more precise, but equivalent version of
Definition~\ref{def:incompleteness}.

\begin{definition}[Incompleteness]\label{def:incompleteness-alternative}
    An axiom system \(S\) is \emph{incomplete}, if an only if there is a closed
    formula~\(A\), such that \emph{neither} \(\seq A\) or \(\seq\lnot A\) can
    be proved in \(S\).
\end{definition}

To show that such a formula exists in \LN, we introduce the \emph{Gödel trick}
and \emph{Gödel sentences}.

\begin{definition}[Gödel trick]\label{def:Gödel trick}
    Assume \(R\) is a predicate symbol that does \emph{not} belong to \LN, and
    therefore does not have a Gödel number. Now assume that \(F(R)\) is
    obtained by using \(R\) as an atomic formula, \(F(R)\notin\LN\), so it has
    no Gödel number. Then proceed as follows:
    \begin{enumerate}
        \item
            Substitute \(R\) in \(F(R)\) with
            \(\exists x\num\Prov(x, \num\sub(a_0, \num\nu(a_0)))\)
            or \(\vdash\num\sub(a_0,\num\nu(a_0))\).
        \item
            Now \(F(\vdash\num\sub(a_0,\num\nu(a_0)))\in\LN\),
            so it must have a Gödel number, call it
            \begin{equation*}
                \Gnum*{F(\vdash\num\sub(a_0,\num\nu(a_0)))} = p\in\Nset\,.
            \end{equation*}
        \item
            Now define a sentence \(A_F\in\LN\) by
            \begin{equation*}
            A_F = \vdash\num\sub(\num p,\num\nu(\num p))) \,.
            \end{equation*}
            This has a Gödel number, and since \(A_F\) is again obtained
            by substitution, \(\Gnum{A_F} = \sub(p,\nu(p))\).
            But then \(\num{\Gnum{A_F}} = \num\sub(\num p,\num\nu(\num p))\),
            and \(A_F\) must be \(F(\vdash\num{\Gnum{A}})\).
    \end{enumerate}
\end{definition}

By applying Gödel's trick, we have obtained the following lemma.

\begin{lemma}[10.14]\label{lem:10.14}
    \(A_F = F(\vdash\num{\Gnum{A}})\).
\end{lemma}

\begin{definition}[\(\omega\)-consistency]\label{def:omega consistency}
    An axiom system \(S\) with \PA{} as its subset is \emph{\(\omega\)-consistent},
    if and only if the following holds for all \(A(a_0)\in S\):
    if the sequent \(\seq\lnot A(\num n)\) is provable in \(S\) ,
    then \(\seq\exists xA(x)\) is \emph{not} provable in \(S\).
\end{definition}

Notice that \(\omega\)-consistency implies consistency. If a system is
\(\omega\)-consistent, then it is also consistent. We now make the following
notational convention related to the implied sequent in the definition
\ref{def:omega consistency}:
\begin{equation*}
    \vdash A
    \quad\text{expands into}\quad
    \vdash\Gnum*{\num A}\,.
\end{equation*}
Armed with this new notation, we can proceed to prove Gödel's incompleteness
theorems.

\begin{theorem}[Gödel's first incompleteness theorem]%
    \label{the:Gödel's first incompleteness theorem}
    If an axiom system is \(\omega\)-consistent,
    then it is incomplete. In particular, \PA{} is incomplete.
\end{theorem}

\begin{proof}
    When applying the Gödel trick of definition~\ref{def:Gödel trick},
    let \(F(R) = \lnot R\). It then defines what is called a \emph{Gödel sentence}
    \(A_G\), and by lemma~\ref{lem:10.14}, \(A_G \equiv\lnot\vdash A_G\).
    We then proceed as follows:
    \begin{enumerate}
        \item
            Suppose \(\seq A_G\) is provable in our system.
            Then by proposition~\ref{prop:10.9.1}, \(\seq\vdash A_G\)
            is also provable in it. Since \(A_G \equiv\lnot\vdash A_G\),
            we then also have \(\lnot A_G \equiv \lnot\lnot\vdash A_G \equiv\vdash A_G\).
            But this means that \(\seq\lnot A_G\) is provable in our system,
            which implies that \(\seq A_G\) is \emph{not} provable in it.
        \item
            Then suppose that \(\seq\lnot A_G\) is not provable in our system.
            This means that for all \(n\in\Nset\) and by extension the corresponding
            numerals \(\num n\in\LN\), that \(\seq\lnot\num\Prov(\num n, \num{\Gnum{A_G}})\)
            or \(\seq\vdash\num{\Gnum{A_G}}\) is provable in the system.

            Now by \(\omega\)-consistency,
            \(\seq\lnot\num\Prov(\num n, \num{\Gnum{A_G}})\)
            or \(\seq\vdash\num{\Gnum{A_G}}\) is \emph{not} provable in the system.
            Therefore \(\seq\vdash A_G\) is also \emph{not} provable.

            Since \(A_\equiv\lnot\vdash A_G\),
            we have \(\lnot A_G\equiv\lnot\lnot\vdash A_G\equiv\vdash A_G\),
            meaning \(\seq\lnot A_G\) is also not provable in our system.
    \end{enumerate}
\end{proof}

\paragraph{Remark:}
Gödel sentences \(A_G\) are not provable in the above axiom system. They
themselves claim that they are not provable. In other words, Gödel sentences
''say'' an intuitively true statement, but the axioms system that contains them
cannot prove this state of affairs.

\paragraph{Remark 2:}
From a semantic or validity point of view, either \(A_G\) or \(\lnot A_G\) is a
valid formula. However, we cannot \emph{prove} \(\seq A_G\), nor can we prove
\(\seq\lnot A_G\) in the axiom system, that contains \PA{}.

\begin{definition}[Symbol for consistency]\label{def:symbol for consistency}
    Let the symbol \(\consis\) denote the closed formula
    \(\lnot\vdash\num0=\num1\), or \(\exists x\num\Prov(x,\num{\Gnum{\num0=\num1}})\).
    In other words, the symbol maps to the statement ''\(S\supseteq\PA\) is consistent.''.
\end{definition}

\begin{theorem}[Gödel's second incompeteness theorem]\label{the:Gödel's second incompleteness theorem}
    If the axiom system \(S\) is consistent, then the sequent
    \(\seq\exists x\num\Prov(x,\num{\Gnum{\num0=\num1}})\) is \emph{not}
    provable in \(S\). In particular, the consistency of \PA{}
    cannot be proved in \PA.
\end{theorem}

\begin{proof}[Sketch of proof]
    We show that \(\consis\equiv A_G\) for any Gödel sentence \(A_G\).
    Since \(\seq A_G\) is \emph{not} provable in \(S\), the neither should
    \(\seq\consis\) be.
\end{proof}

Together, Gödel's incompleteness theorems state that it is not possible to
construct a formal axiom system \(S\) containing Peano arithmetic \PA{}, in
which we could always prove that either \(A\) or \(\lnot A\) is provable, but
not both at the same time, for any well-formed sentence \(A\). This is
especially shocking, as Peano arithmetic is such a simple system, with only a
couple of axioms.

This completely ruins the idea of Leibnitz's \emph{Calculus Reticinator} from
1685. It also nullifies Hilbert's ''theory of everything'', the second problem
''Ignorabimus'' he presented in Paris, 1900.

When Gödel at the age of \(24\) ended his talk on these theorems in a 1930
seminar, someone exclaimed ''It is all over now!''. He could have meant at
least two of different things:
\begin{enumerate}
    \item either the seminar itself was about to end or
    \item the end of Leibniz's program.
\end{enumerate}
Either way, it has come time for us to also state: ''It is over.''.
