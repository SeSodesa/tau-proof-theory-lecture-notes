\section{Peano arithmetic or PA}%
\label{sec:peano-arithmetic}

This is also called ''The Theory of natural numbers''.
This is a special case of the general language of \LK{},
where we restrict our alphabet and terms according to the following
definitions. We use the notation \LN{} to refer to this language.

\begin{definition}[Alphabet of \PA{}]\label{def:alphabet of peano arithmetic}
    We have the following symbols at our disposal,
    when talking in the language of Peano Arithmetic \PA:
    \begin{itemize}
        \item
            There is only one individual constant \(\bar0\),
            known in our meta language as ''zero''.
        \item
            The only three function constants are the unary left-associative
            \emph{successor} \('\), and the binary operations
            \emph{addition} \(+\) and \emph{product} \(\times\).
        \item
            The only binary predicate symbol is \(=\),
            the equality defined in the previous chapter.
    \end{itemize}
\end{definition}

\begin{definition}[Numerals]\label{def:numerals}
    In order to simplify our notations,
    we make the following bijective notational agreement:
    \begin{equation*}
        \bar0\mapsto\bar0,\quad
        \bar1\mapsto\bar0',\quad
        \bar2\mapsto\bar0'',\quad
        \bar3\mapsto\bar0''',\quad
        \ldots\quad,\quad
        \bar{n}\mapsto\bar0^{\overbrace{\prime\cdots\prime}^{n\text{ times}}}\,.
    \end{equation*}
    In other words, \(\bar{n}\) refers to the natural number \(n\in\Nset\),
    familiar to us from our meta language.
\end{definition}

\begin{definition}[Terms of \PA{}]\label{def:terms of peano arithmetic}
    We define the words or \emph{terms} of \PA{} recursively as follows:
    \begin{enumerate}
        \item
            Free variables and \(\bar0\) are terms.
        \item
            If \(s\) and \(t\) are terms, then so are \(s'\),
            \(s+t\) and \(s\times t\).
    \end{enumerate}
\end{definition}
\begin{definition}[Closed terms of \PA{}]\label{def:closed terms of peano arithmetic}
    \emph{Closed terms} are composed of the numerals of definition~\ref{def:numerals}
    and the function constants and equality of definition~\ref{def:alphabet of peano arithmetic}:
    \begin{enumerate}
        \item
            Individual numerals are closed terms.
        \item
            If \(\bar m\) and \(\bar n\) are closed terms,
            then \(\bar m '\), \(\bar m +\bar n\), \(\bar m\times\bar n\)
            and \(\bar m = \bar n\) are closed terms as well.
    \end{enumerate}
\end{definition}

Note that a statement like \(\bar2+\bar3=\bar5\) is something that needs to be
proven in this language. We may not take it for granted. To be able to prove
things in \PA{}, we must first agree on our axioms, which are listed below.

\begin{axiom}[Logical axioms of \PA{}]\label{axi:logical axioms of PA}
    There is only a single logical axiom: if \(A\) is a formula,
    then \(A\seq A\) is this axiom.
\end{axiom}
\begin{axiom}[Equality axioms of \PA{}]\label{axi:equality axioms of PA}
    The equality axioms of \PA{} are special instances of the axioms
    of \LK=. These are
    \begin{align}
        \tag{\LN:eq:ax:1}\label{LN:eq:ax:1}
        &\seq s=s \\
        \tag{\LN:eq:ax:2}\label{LN:eq:ax:2}
        s = t &\seq s'=t', \\
        \tag{\LN:eq:ax:3}\label{LN:eq:ax:3}
        s_1=t_1, s_2=t_2 &\seq s_1+s_2 =t_1+t_2,
        \quad\text{and} \\
        \tag{\LN:eq:ax:4}\label{LN:eq:ax:4}
        s_1=t_1, s_2=t_2 &\seq s_1\times s_2 =t_1\times t_2,
    \end{align}
\end{axiom}
\begin{axiom}[Mathematical axioms of \PA{}]\label{axi:mathematical axioms of PA}
    \begin{align}
        \tag{\LN:m:ax:1}\label{LN:m:ax:1}
        s'=t'&\seq s=t, \\
        \tag{\LN:m:ax:2}\label{LN:m:ax:2}
        s' = 0 &\seq, \\
        \tag{\LN:m:ax:3}\label{LN:m:ax:3}
        &\seq s + \bar0 = s, \\
        \tag{\LN:m:ax:4}\label{LN:m:ax:4}
        &\seq s + t' = (s+t)', \\
        \tag{\LN:m:ax:5}\label{LN:m:ax:5}
        &\seq s\times\bar0 = \bar0,
        \quad\text{and} \\
        \tag{\LN:m:ax:6}\label{LN:m:ax:6}
        &\seq s\times t' = s \times t + s\,.
    \end{align}
\end{axiom}

Finally, the inductive principle can also be formalized.
We will not be needing the formalization in this course,
but it is shown here for completeness.

\begin{axiom}[Inductive axiom]\label{axi:inductive axiom}
    Induction is formalized as follows:
    let \(\bar k\) and \(\bar n\) be closed terms. Then
    \begin{equation}\tag{induction}\label{induction}
        P(\bar0), P(\bar k)\limplies P(\bar k') \seq \forall\bar nP(\bar n)\,.
    \end{equation}
\end{axiom}

\paragraph{Note:}
Before we dive in further and start proving things,
please remember that \(n\in\Nset\) is a natural number
in our meta language, whereas the \emph{numerals} or
\emph{closed terms} \(\bar n\) are the representations of
these numbers in our \emph{object language}.

\paragraph{Note 2:}
The allowed inference rules are the ones given in
definition~\ref{def:inference rules in LK}.

\begin{exercise}[PA.1]\label{exe_PA.1}
    Prove that \(\bar2 + \bar1 = \bar3\).
\end{exercise}

It can be shown more generally, that if \(\bar m\) and \(\bar n\)
are numerals, then
\begin{equation*}
    \seq\bar m + \bar n = \overline{m + n}
    \quad\text{and}\quad
    \seq\bar m \times \bar n = \overline{m \times n}\,.
\end{equation*}

\begin{proposition}[9.4]\label{prop:9.4}
    \PA{} is consistent, as in the empty sequent \(\seq\)
    is not provable in it.
\end{proposition}

\begin{proof}
    Omitted. This is taken for granted.
\end{proof}

\begin{proposition}[9.6.1]\label{prop:9.6.1}
    For any closed term \(s\), there is a unique numeral \(\bar n\),
    so that the sequent \(\seq s = \bar n\) is provable.
\end{proposition}

\begin{proof}.
    Let \(k\) be the number of instances of the functions
    \('\), \(+\) and \(\times\) in \(s\). We induce on this \(k\).

    In the base case \(k=0\), we only have the single numeral \(\bar m\),
    for which the claim holds: axiomatically \(\seq s=s\), so \(\seq\bar m=\bar m\).
    For the inductive hypothesis, we assume that the claims \(\seq s=\bar m\)
    and \(\seq t=\bar n\) hold, each with \(k\leq p\) instances of the functions are provable.
    In the inductive step we add one more instance of the functions to the proofs
    and show that the cclaim still holds to close the induction.

    In the case of the successor \(\prime\),
    \begin{equation*}
        \begin{prooftree}
            \hypo{\seq s=\bar n}
            \hypo{s=\bar n \seq s'=\bar n'}
            \infer2[\eqref{eq:cut}]{\seq s'=\bar n'}
        \end{prooftree} \,.
    \end{equation*}
    Notice that \(\bar n'\) is \(\overline{n+1}\),
    means that \(\seq s'=\overline{n+1}\), which closes the induction on
    the successor \(\prime\).

    For the sum \(+\), we introduce axioms as needed:
    \scriptsize
    \begin{equation*}
        \begin{prooftree}
            \hypo{\seq \bar m+\bar n = \overline{m + n}}
            % --------------------------------
            \hypo{\seq t = \bar m}
            % --------------------------------
            \hypo{\seq s = \bar n}
            % --------------------------------
            \hypo{s = \bar n, t = \bar m \seq s + t = \bar m + \bar n}
            \infer2[\eqref{eq:cut}]{t = \bar m \seq s + t = \bar m + \bar n}
            \infer2[\eqref{eq:cut}]{\seq s+t = \bar m + \bar n}
            \hypo{s+t = \bar m + \bar n, \bar n + \bar m = \overline{n + m}\seq s+t = \overline{m + n}}
            % --------------------------------
            \infer2[\eqref{eq:cut}]{\bar m+\bar n = \overline{m + n}\seq s+t = \overline{m + n}}
            % --------------------------------
            \infer2[\eqref{eq:cut}]{\seq s+t=\overline{m + n}}
        \end{prooftree}
    \end{equation*}
    \normalsize
    The induction is then closed on the sum \(+\).
    To prove the case of the product \(\times\), simply replace \(+\)
    with \(\times\) in the above inductive step. This is left as an optional exercise,
    which closes the induction when completed.

    To prove the uniqueness of the numeral \(\bar n\),
    make the contrapositive assumption, so that
    there are differing numerals \(\bar p\) and \(\bar q\),
    so that by proposition~\ref{prop:7.2}
    \begin{equation*}
        \begin{prooftree}
            \hypo{\seq s = \bar p}
            % ------------------------
            \hypo{\seq s = \bar q}
            % ------------------------
            \hypo{s = \bar q\seq \bar q = s}
            % ------------------------
            \hypo{\bar q = s, s = \bar p\seq \bar q = \bar p}
            \infer2[\eqref{eq:cut}]{s = \bar q, s = \bar p\seq \bar q = \bar p}
            \infer2[\eqref{eq:cut}]{s = \bar p\seq \bar q = \bar p}
            \infer2{\seq \bar q = \bar p}
        \end{prooftree} \,.
    \end{equation*}
    Thus supposing that the numerals differ in the given manner
    leads to them being the same numeral. We are then done with
    the proof.
\end{proof}

\begin{proposition}[9.6.2]\label{prop:9.6.2}
    In \PA{}, if \(s\) and \(t\) are closed terms, then either
    the sequent \(\seq s=t\) or \(s=t\seq\) holds. The terms
    are either equal, or they are not.
\end{proposition}

\begin{proof}
    As we have already proved proposition \ref{prop:9.6.1},
    it is enough to show that if \(\bar m\) and \(\bar n\) are
    numerals, then either \(\seq \bar m = \bar n\) or \(\bar m = \bar n\seq\) holds.
    In other words, either the two numerals are equal, or they are not.
    We induce on the number \(k\) of successors \(\prime\) in the numerals.

    In the base case \(k=0\), so the numerals are both zeroes,
    and \(\seq\bar0=\bar0\) holds according to the axioms of \PA{}.
    The claim then holds for \(k=0\).

    In the inductive hypothesis, we assume that the claim \(\seq\bar m=\bar n\)
    holds for \(k = p\) successors in the numerals \(\bar m\) and \(n\).
    We then add one more successor to either \(\bar m\) or \(\bar n\)
    and demonstrate that the claim still holds after the mattter.

    By the axioms of \PA{} and inductive hypotheses,
    \begin{equation*}
        \begin{prooftree}
            \hypo{\seq\bar m=\bar n}
            \hypo{m = n\seq\bar m'=\bar n'}
            \infer2[\eqref{eq:cut}]{\seq\bar m'=\bar n'}
        \end{prooftree}\,,
    \end{equation*}
    so if \(m\) and \(n\) are the same natural numbers,
    then \(\seq\bar m=\bar n\).

    Now, as by the axioms of \PA{}, zero cannot be the successor
    of any numeral, as in
    \begin{equation*}
        \bar1'=0\seq,\quad
        \bar2'=0\seq,\quad
        \bar3'=0\seq,\quad
        \ldots\,,
    \end{equation*}
    and the axioms also allow us to make the proof
    \begin{equation*}
        \begin{prooftree}
            \hypo{\bar n''=\bar 0'\seq \bar n' = \bar0}
            \hypo{\bar n'=0\seq}
            \infer2{\bar n''=\bar 0'\seq}
        \end{prooftree}\,,
    \end{equation*}
    we know that
    \begin{equation*}
        \bar2'=\bar1\seq,\quad
        \bar3'=\bar1\seq,\quad
        \bar4'=\bar1\seq,\quad
        \ldots\,.
    \end{equation*}
    Similarly, but in a more concrete and examplary manner,
    we can argue that
    \begin{equation*}
        \begin{prooftree}
            \hypo{3 = 2 \seq 2 = 1}
            \hypo{2 = 1 \seq}
            \infer2{3\seq 2\seq}
        \end{prooftree}\,,
    \end{equation*}
    where the numeral \(\bar3\) may be replaced with any numeral \(\bar n\), so
    \begin{equation*}
        \bar3'=\bar2\seq,\quad
        \bar4'=\bar2\seq,\quad
        \bar5'=\bar2\seq,\quad
        \ldots\,.
    \end{equation*}

    Proceeding in the above manner ad infinitum,
    we can produce all sequences \(\bar m=\bar n\seq\),
    for the natural numbers \(m\) and \(n\), for which \(m > n\) holds.
    To handle the case where \(m < n\), simply exchange the roles
    of \(m\) and \(n\) in the proofs, which produces \(\bar n=\bar m\seq\).

    We have now shown that, speaking in metalanguage,
    either \(m=n\), \(m>n\) or \(m < n\) holds,
    but none at the same time. Therefore
    \begin{equation*}
        \bar m = \bar n\seq,\quad
        \seq\bar m=\bar n,
        \quad\text{or}\quad
        \bar n=\bar m\,.
    \end{equation*}
    It now remains to show that \(\bar m=\bar n\seq\),
    if and only if \(\bar n=\bar m\seq\), which we omit.
\end{proof}
