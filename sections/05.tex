\section{\texorpdfstring{The Cut-elimination theorem in \LK}{The Cut-elimination theorem in LK}}

This theorem is also known as \emph{Gentzen's Haubtsatz}.
In short, the theorem goes as follows: If a theorem is provable,
then it is provable without~\eqref{eq:cut}.
We need multiple assisting results,
before this can be proven,
and some of the details will be omitted for brevity.

\begin{lemma}[The mix rule, Lemma 5.2]\label{lem:mix rule}
    Let \(\Gamma\), \(\Delta\), \(\Delta^\ast\), \(\Pi\), \(\Pi^\ast\) and \(\Lambda\) be sequence formulae
    and \(A\) a single formula, such that \(A\in\Delta\), \(A\in\Pi\),
    \(A\notin\Delta^\ast\) and \(A\notin\Pi^\ast\). Then the rule~\eqref{eq:cut}
    can be replaced by the rule
    \begin{equation}\label{eq:mix}\tag{mix}
        \begin{prooftree}
            \hypo{\Gamma\seq\Delta}
            \hypo{\Pi\seq\Lambda}
            \infer2[\(\eqref{eq:mix}:A\)]{\Gamma,\Pi^\ast\seq\Delta^\ast\Lambda}
        \end{prooftree}\,.
    \end{equation}
\end{lemma}

\begin{proof}
    Let the number of instances of \(A\) in the sequence
    formulae \(\Delta\) and \(\Pi\) be \(k\) and \(l\),
    respectively. Also assume that the rule~\eqref{eq:cut}
    is usable in the following manner:
    \begin{equation*}
        \begin{prooftree}
            \hypo{\Gamma\seq\Delta, A}
            \hypo{A, \Pi\seq\Lambda}
            \infer2[\eqref{eq:cut}]{\Gamma,\Pi\seq\Delta,\Lambda}
        \end{prooftree}\,.
    \end{equation*}
    The same result can be obtained by applying the \eqref{eq:mix} rule with respect to \(A\):
    \begin{equation*}
        \begin{prooftree}
            \hypo{\Gamma\seq\Delta, A}
            \hypo{A, \Pi\seq\Lambda}
            \infer2[\(\eqref{eq:mix}:A\)]{\Gamma,\Pi^\ast\seq\Delta^\ast,\Lambda}
            \infer[rule style=double]1[\(l\times\eqref{eq:weakening:left}\) and \eqref{eq:exchange:left}]{
                \Gamma,\Pi\seq\Delta^\ast,\Lambda
            }
            \infer[rule style=double]1[\(k\times\eqref{eq:weakening:right}\) and \eqref{eq:exchange:right}]{
                \Gamma,\Pi\seq\Delta,\Lambda
            }
        \end{prooftree}
    \end{equation*}

    Then assume that the \eqref{eq:mix} rule with respect to \(A\) can be used in the proof
    \begin{equation*}
        \begin{prooftree}
            \hypo{\Gamma\seq\Delta}
            \hypo{\Pi\seq\Lambda}
            \infer2[\(\eqref{eq:mix}:A\)]{\Gamma,\Pi^\ast\seq\Delta^\ast,\Lambda}
        \end{prooftree} \,.
    \end{equation*}
    If we now assume that \(l\geq k\), the \eqref{eq:cut} rule can be used to prove the same theorem:
    \begin{equation*}
        \begin{prooftree}
            \hypo{\Gamma\seq\Delta}
            \infer[rule style=double]1[\eqref{eq:exchange:right}]{\Gamma\seq\Delta^\ast, A,\ldots, A}
            \infer[rule style=double]1[\eqref{eq:contraction:right}]{\Gamma\seq\Delta^\ast, A}
            \hypo{\Pi\seq\Lambda}
            \infer[rule style=double,left label=\eqref{eq:exchange:left}]1{A,\ldots,A,\Pi^\ast\seq\Lambda}
            \infer[rule style=double,left label=\eqref{eq:contraction:left}]1{A,\Pi^\ast\seq\Lambda}
            \infer2[\eqref{eq:cut}]{\Gamma,\Pi^\ast\seq\Delta^\ast,\Lambda}
        \end{prooftree}\,.
    \end{equation*}
    Therefore the rules \eqref{eq:cut} and \eqref{eq:mix} are equivalent in \LK.
\end{proof}

\begin{definition}[Rank of a proof]\label{def:rank of a proof}
    The \emph{rank} of a proof \(P\) is the supremum of the lengths
    of \eqref{eq:cut} formulae in \(P\), the longest formula
    removed from the proof by applying the~\eqref{eq:cut} rule
    in some inference.
\end{definition}

\begin{minipage}{0.7\textwidth}
The rank of a proof can also be formulated in another manner
using the \eqref{eq:mix} rule. But first let us consider the \emph{linear}
proof seen to the right of this paragraph. The \emph{left rank}
of this proof with respect to the formula \(A\),
\(\rank_{l,A}(P)\), is defined as the number of sequents
\(\Gamma_i\seq\Delta_i\) with the formula \(A\in\Gamma_i\).
Similarly, the \emph{right rank} with respect to the formula \(B\),
\(\rank_{r,B}(P)\), is the number of sequents \(\Gamma_j\seq\Delta_j\)
with \(B\in\Delta_j\). Here \(\rank_{l,A}(P) = \rank_{r,B}(P) = 3\).
\end{minipage}
\hfill
\begin{minipage}{0.3\textwidth}
\begin{equation*}
    \begin{prooftree}
        \hypo{\Gamma_1(A)\seq\Delta_1(B)}
        \infer1{\Gamma_2\seq\Delta_2(B)}
        \infer1{\Gamma_3(A)\seq\Delta_3}
        \infer1{\Gamma_4\seq\Delta_4(B)}
        \infer1{\Gamma_5(A)\seq\Delta_5}
    \end{prooftree}
\end{equation*}
\end{minipage}

Now let us look at a branching tree.
We will define the left and right \emph{maximum ranks}
of the tree with respect to the formula \(A\) as follows:
\begin{equation*}
    \maxrank_{l,A}(P) = \max(\rank_{l,A}(P_i))
    \qquad\text{and}\qquad
    \maxrank_{l,A}(P) = \max(\rank_{l,A}(P_i)),
\end{equation*}
where each \(P_i\) is a linear path from the root
to each leaf of the tree.

Now let \(P\) be a proof of the sequent
\(\Gamma,\Pi^*\seq\Delta^*,\Lambda\), whose last inference consists of
using the~\eqref{eq:mix} rule to remove the formula \(A\), so that \(P\)
has a left and a right branch \(P_l\) and \(P_r\) growing off the root,
as is seen in the below proof sketch.

\small
\begin{equation*}
    \begin{prooftree}
        % Left branch
        \hypo{\Gamma_1\seq\Delta_1(A)}
        \infer1{}
        \ellipsis{}{}
        \hypo{\Gamma_2\seq\Delta_2(A)}
        \hypo{\Gamma_4\seq\Delta_4(A)}
        \ellipsis{}{}
        \infer1{\Gamma_3\seq\Delta_3(A)}
        \infer2{}
        \ellipsis{}{}
        \infer2{\Gamma\seq\Delta(A)}
        \delims{P_l\left\{}{\right.}
        % Right branch
        \hypo{\Pi_4(A)\seq\Lambda_4}
        \infer1{}
        \ellipsis{}{}
        \hypo{\Pi_3(A)\seq\Lambda_3}
        \ellipsis{}{}
        \infer1{\Pi_2(A)\seq\Lambda_2}
        \hypo{\Pi_1(A)\seq\Lambda_1}
        \infer2{}
        \ellipsis{}{}
        \infer2{\Pi(A)\seq\Lambda}
        \delims{\left.}{\right\}P_r}
        \infer2[\(\eqref{eq:mix}:A\)]{\Gamma,\Pi^*\seq\Delta^*,\Lambda}
    \end{prooftree}
\end{equation*}
\normalsize

The rank of the above proof tree with respect to the~\eqref{eq:mix} formula \(A\)
is now defined as follows:
\begin{equation*}
    \rank_A(P) = \maxrank_{r,A}(P_l) + \maxrank_{l,A}(P_r) \geq 2 \,.
\end{equation*}
Here the lower limit \(2\) stems from the fact that
an application of the mix rule requires that there
be at least \(1\) instance of \(A\) in both \(P_l\) and \(P_r\).

\begin{definition}[The grade of a proof]\label{def:grade of a proof}
    The \emph{grade} of a proof \(P\) with an application of the \eqref{eq:mix}
    rule is the number of logical symbols in the \eqref{eq:mix} formula \(A\).
    In other words, we can set \(\grade(P) = \grade(A)\).
\end{definition}

With these definitions in place, we can now prove the following lemma.
It is the second central result in proving the \eqref{eq:cut}-elimination theorem,
and claims that proofs with \eqref{eq:mix} rules also work without them.

\begin{lemma}[Lemma 5.4]\label{lem:5.4}
Assume that the set of proofs \(\LK^\ast\) is obtained from
the set \LK{} by replacing the \eqref{eq:cut} rule by
the \eqref{eq:mix} rule. Assume that \(S\) is a sequent,
whose proof~\(P\in\LK^\ast\), such that the last inference rule of \(P\)
is the only instance of \eqref{eq:mix} in \(P\).
Then \(P\in\LK\).
\end{lemma}

\begin{proof}[Sketch of the proof]
    By double induction on the rank of the proof \(P\),
    denoted with \(n\). There are two cases:
    either \(\rank_A(P) = 2\), or \(\rank_A(P) > 2\).

    In the base case, let \(n=2\). Then by the above definitions,
    we know that
    \begin{equation*}
    \maxrank_{r,A}(P_l) = 1 = \maxrank_{l,A}(P_r)\,.
    \end{equation*}
    But for this to be true, the last inference must go as follows:
    \begin{equation*}
        \begin{prooftree}
            \hypo{A\seq A}
            \hypo{\Pi(A)\seq\Lambda}
            \infer2[\(\eqref{eq:mix}:A\)]{A,\Pi^*\seq,\Lambda}
        \end{prooftree}
        \quad\text{or}\quad
        \begin{prooftree}
            \hypo{\Gamma\seq\Delta(A)}
            \hypo{A\seq A}
            \infer2[\(\eqref{eq:mix}:A\)]{A,\Pi^*\seq,\Lambda}\,.
        \end{prooftree}
    \end{equation*}
    The left inference can be replaced with the following \eqref{eq:mix}-free proof:
    \begin{equation*}
        \begin{prooftree}
            \hypo{\Pi\seq\Lambda}
            \infer[double]1[\eqref{eq:exchange:left}]{A,\ldots,A,\Pi^*\seq\Lambda}
            \infer[double]1[\eqref{eq:contraction:left}]{A,\Pi^*\seq\Lambda}
        \end{prooftree}
    \end{equation*}
    The case where the right upper sequent is the axiom
    is entirely symmetrical.

    Now what if the upper sequents \(\Gamma\seq\Delta\) and
    \(\Pi\seq\Lambda\) in the \eqref{eq:mix} inference are not axioms?
    Then they have to be obtained by at least one inference rule.
    The only possible structural rule that can be applied to obtain
    \(\Gamma\seq\Delta\) is \eqref{eq:weakening:right}
    (assuming \(A\notin\Delta\)), because by assumption
    \(\maxrank_{r,A}(P_l) \geq 2\). Then the last inference is
    \begin{equation*}
        \begin{prooftree}
            \hypo{\Gamma\seq\Delta}
            \infer1[\eqref{eq:weakening:right}, \(A\notin\Delta\)]{\Gamma\seq\Delta,A}
            \hypo{\Pi\seq\Lambda}
            \infer2{\Gamma,\Pi^*\seq\Delta,\Lambda}
        \end{prooftree}\,.
    \end{equation*}
    This can be replaced with the \eqref{eq:mix} free proof
    \begin{equation*}
        \begin{prooftree}
            \hypo{\Gamma\seq\Delta}
            \infer[double]1[\eqref{eq:weakening:left}]{\Pi^*,\Gamma\seq\Delta}
            \infer[double]1[\eqref{eq:weakening:right}]{\Pi^*,\Gamma\seq\Delta,\Lambda}
            \infer[double]1[\eqref{eq:exchange:left}]{\Gamma,\Pi^*\seq\Delta,\Lambda}
        \end{prooftree}\,.
    \end{equation*}
    In the case where \(\Pi\seq\Lambda\) is obtained by structural rules,
    the arguments are done similarly.

    For the case where both \(\Gamma\seq\Delta\) and \(\Pi\seq\Lambda\),
    are obtained by logical inferences, we change our startegy.
    We now induce on the \emph{grade} of the proof, the number
    of logical symbols in the \eqref{eq:mix} formula \(A\).

    In the base case, \(\grade(P) = 0\), so there is nothing to prove.
    For the inductive hypothesis, we assume the claim holds for
    \(\grade(P) = k\), and in the inductive step we add logical symbols
    one at a time and show that we can still come up with a \eqref{eq:mix}
    free version of the proof, closing the induction.

    We start with \(\lnot\). In this case \(\Gamma\seq\Delta\) and \(\Pi\seq\Lambda\)
    can be obtained as follows:
    \begin{equation*}
        P:
        \begin{prooftree}
            \hypo{A,\Gamma\seq\Delta}
            \infer1[\eqref{eq:lnot:right}]{\Gamma\seq\Delta,\lnot A}
            \hypo{\Pi\seq\Lambda,A}
            \infer1[\eqref{eq:lnot:left}]{\lnot A,\Pi\seq\Lambda}
            \infer2[\eqref{eq:mix}\(:\lnot A\)]{\Gamma,\Pi^*\seq\Delta^*,\Lambda}
        \end{prooftree}\,.
    \end{equation*}
    Because there is no \(\lnot A\) in \(\Pi\) or \(\Delta\),
    \(\Pi^* = \Pi\) and \(\Delta^*=\Delta\).
    The above proof can then be replaced by the following proof
    with the mix formula being \(A\) and not \(\lnot A\):
    \begin{equation*}
        P':
        \begin{prooftree}
            \hypo{\Pi\seq\Lambda,A}
            \hypo{A,\Gamma\seq\Delta}
            \infer2[\eqref{eq:mix}\(:A\)]{\Pi,\Gamma^*\seq\Lambda^*,\Delta}
        \end{prooftree}\,.
    \end{equation*}
    By the inductive hypothesis \IH{}, this can be replaced by a \eqref{eq:mix} free proof \(P''\),
    as \(\grade(A) < \grade(\lnot A)\) and \(P'\) contains only one use of \eqref{eq:mix}.
    The missing formulae \(A\) can be obtained with \eqref{eq:weakening:left}
    and \eqref{eq:weakening:right}, so
    \begin{equation*}
        P'':
        \begin{prooftree}
            \hypo{\IH}
            \infer1{\Pi,\Gamma^*\seq\Lambda^*,\Delta}
            \infer[double,left label=\eqref{eq:exchange:left}]1[\eqref{eq:exchange:right}]{\Gamma,\Pi^*\seq\Delta^*,\Lambda}
        \end{prooftree}\,.
    \end{equation*}
    This closes induction on the symbol \(\lnot\).

    If the \eqref{eq:mix} formula is \(C\limplies D\),
    the proof \(P\) could end in the following manner:
    \begin{equation*}
        P:
        \begin{prooftree}
            \hypo{C,\Gamma\seq\Delta,D}
            \infer1[\eqref{eq:limplies:right}]{\Gamma\seq\Delta,C\limplies D}
            \hypo{\Pi\seq\Lambda}
            \infer1[\eqref{eq:weakening:left}, \(C\limplies D\notin\Pi\)]{C\limplies D,\Pi\seq\Lambda}
            \infer2[\eqref{eq:mix}\(:A\)]{\Pi,\Gamma^*\seq\Lambda^*,\Delta}
        \end{prooftree}\,.
    \end{equation*}
    This can be replaced with the \eqref{eq:mix} free proof
    \begin{equation*}
        \begin{prooftree}
            \hypo{\Pi\seq\Lambda}
            \infer[double, left label={\eqref{eq:weakening:left}, \eqref{eq:exchange:left}}]1[\eqref{eq:weakening:right}, \eqref{eq:exchange:right}]{\Pi,\Gamma^*\seq\Lambda^*,\Delta}
        \end{prooftree}\,.
    \end{equation*}
    There are also other ways that the sequents \(\Gamma\seq\Delta\) and
    \(\Pi\seq\Lambda\) could be constructed via \(C\limplies D\),
    but these are omitted. The rest of the logical connectives are also
    omitted, as the manner in which the reader should proceed in
    closing the induction in their case should be clear by now.

    \paragraph{The case \(\bm{\rank_A(P) > 2}\).}
    We use induction as follows:
    for any proof \(Q\) where \eqref{eq:mix} is used only once,
    it is the last inference, \(\grade(Q) < \grade(P)\) or
    \(\grade(Q) = \grade(P)\), \eqref{eq:mix} can be removed.
    It is then shown that \eqref{eq:mix} can also be replaced in \(P\).
    We also omit the details here, for brevity.
\end{proof}

\begin{theorem}[Theorem 5.3]\label{the:5.3}
    If a sequent \(S\) is provable by using the \eqref{eq:mix} rule,
    then it is also provable without it.
\end{theorem}

\begin{proof}[An idea of the proof]
    Let \(P\) be a proof of \(S\).
    Now investigate some branch of \(P\),
    where and instance of the \eqref{eq:mix} rule appears.
    By lemma~\ref{lem:5.4}, this branch can be replaced by
    a \eqref{eq:mix}-free proof \(Q\), ending in \(S\).
    We continue applying this procedure to all
    \eqref{eq:mix}-containing branches of \(P\),
    until no branches with instances of the \eqref{eq:mix} rule
    exist in \(P\). This concludes the proof.
\end{proof}

Together with Lemmas~\ref{lem:mix rule} and \ref{lem:5.4},
Theorem~\ref{the:5.3} proves Gentzen's Haubsatz.

\begin{theorem}[Theorem 5.5]\label{the:5.5}
    The cut-elimination theorem also holds in \LJ.
\end{theorem}
