\section{\texorpdfstring{Consequences of Cut-elimination -- Consistency of \LK{} and \LJ}{Consequences of Cut-elimination -- Consistency of LK{} and LJ}}%
\label{sec:consequences-of-cut-elimination}

To begin discussion on this topic, we need to have an idea
of what the word \emph{subformula} means. Thankfully,
the idea is not very difficult and is illustrated by
the following examples:
\begin{itemize}
    \item
        The subformulae of \(\exists xF(x)\) are the formula itself,
        \(F(a)\), and the subformulae of \(F(a)\).
    \item
        The subformulae of \(C\land D\) are \(C\land D\), \(C\) and \(D\),
        and the subformulae of \(C\) and \(D\).
\end{itemize}
In other words, subformulae of a formula \(A\) are the formulae that are used
to construct \(A\), including \(A\) itself.

If we think of the inferences
\begin{equation*}
    \begin{prooftree}
        \hypo{T'\seq\Delta'}
        \hypo{T''\seq\Delta''}
        \infer2{\Gamma\seq\Delta}
    \end{prooftree}
    \quad\text{and}\quad
    \begin{prooftree}
        \hypo{T'\seq\Delta'}
        \infer1{\Gamma\seq\Delta}
    \end{prooftree}\,,
\end{equation*}
we notice that the upper sequents are always composed of subformulae
of the lower sequents, but not conversely. We then have the following result.

\begin{theorem}[Subformula property]\label{the:subformula property}
    If \(P\) is a proof of the sequent \(\Gamma\seq\Delta\),
    then \(P\) contains only formulae that are subformulae of
    the sequence formulae \(\Gamma\) and \(\Delta\), but not conversely.
\end{theorem}

\begin{proof}
    Omitted.
\end{proof}

Now we can establish the following central result.

\begin{theorem}\label{eq:consistency of LK and LJ}
    The sets \LK{} and \LJ{} are consistent,
    as per definition~\ref{def:consistency}.
\end{theorem}

\begin{proof}
    We proceed by the contrapositive principle and
    assume to the contrary, so that the empty sequent \(\seq\)
    is provable in \LK{} or \LJ{}. Then by the previous results
    it can be proven without the \eqref{eq:cut} rule,
    and therefore the supposed proof can only contain empty sequents
    because of theorem~\ref{the:subformula property}.
    But for a proof to be a proof, it has to start with an
    axiom, which is of the form \(A\seq A\),
    which is not an empty sequent. This is a contradiction,
    so the claim must be true.
\end{proof}

\begin{theorem}\label{the:6.3}
    In a \eqref{eq:cut}-free proof \(P\) of \(\Gamma\seq\Delta\),
    all formulae are subformulae of \(\Gamma\) and \(\Delta\).
\end{theorem}

\begin{proof}[A sketch of a proof]
    By induction on the number of inferences in \(P\),
    denoted by \(n\). In the base case, \(n=0\)
    and we only have axioms \(A\seq A\) and the claim holds.

    As the inductive hypothesis, we assume that the claim holds
    for \(n=k\) inferences. In the inductive step, we simply
    add the known inference rules one at a time and check that
    the claim still holds.
\end{proof}
