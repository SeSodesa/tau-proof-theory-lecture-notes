\section{Predicate calculus with equality}%
\label{sec:predicate-calculus-with-equality}

This is also known as the ''Theory of Equality'' and is denoted with \(\LK_=\).
Here we add one additional symbol \(=\) to the alphabet of \LK{}, and use it to
form a set of binary predicates with the following set of axioms concerning the
terms \(s\), \(s_i\), \(t_i\), function symbols \(f\) and predicate symbols
\(R\).

\begin{axiom}[Axioms of equality]\label{axi:equality}
\begin{align}
\tag{=:ax:1}\label{eq:ax:1}
    &\seq s = s, \\
\tag{=:ax:2}\label{eq:ax:2}
    s_1 = t_1,\ldots,s_n = t_n &\seq f(s_1,\ldots,s_n) = f(t_1,\ldots,t_n) \\
\tag{=:ax:3}\label{eq:ax:3}
	s_1 = t_1,\ldots,s_n = t_n, R(s_1,\ldots,s_n) &\seq R(t_1,\ldots,t_n)
\end{align}
\end{axiom}
The statements~\eqref{eq:ax:1}--\eqref{eq:ax:3} form the equality axioms.

\begin{proposition}[7.2]\label{prop:7.2}
    If \(A(a_1,\ldots,a_n)\) is a formula,
    then the substitution formula
    \begin{equation}\tag{substitution}\label{eq:substitution}
        s_1 = t_1,\ldots,s_n=t_n,A(s_1,\ldots,s_n)\seq A(t_1,\ldots,t_n)
    \end{equation}
    is \LK-provable. In particular:
    \begin{align}
        \tag{=:symmetry}\label{eq:symmetry}
            s=t&\seq t=s\,,\\
        \tag{=:transitivity}\label{eq:transitivity}
            s=t,t=v&\seq s=v\quad\text{and} \\
        \tag{=:sequential symmetry}\label{eq:sequential symmetry}
            s_1 = t_1,\ldots,s_n = t_n&\seq t_1 = s_1, \ldots, t_n = s_n\,.
    \end{align}
\end{proposition}

\begin{proof}
    We start with~\eqref{eq:symmetry}.
    According to the axioms~\eqref{eq:ax:1} and \eqref{eq:ax:3},
    we may write
    \begin{equation*}
        \begin{prooftree}
            \hypo{\eqref{eq:ax:1}}
            \ellipsis{}{\seq s=s}
            \hypo{s_1 = t_1, s_2 = t_2, R(s_1,s_2) \seq R(t_1,t_2)}
            \infer1[\eqref{eq:ax:3}]{s=t, s=s, s=s\seq t=s}
            \infer[double]1[\eqref{eq:exchange:left}]{s=s, s=s, s=t \seq t=s}
            \infer1[\eqref{eq:contraction:left}]{s=s, s=t \seq t=s}
            \infer2[\eqref{eq:cut}]{s = t\seq t = s}
        \end{prooftree}
    \end{equation*}
    As for~\eqref{eq:transitivity}, the proof
    \begin{equation*}
        \begin{prooftree}
            \hypo{\eqref{eq:symmetry}}
            \ellipsis{}{s = t\seq t = s}
            \hypo{\eqref{eq:ax:1}}
            \ellipsis{}{\seq t = t}
            \hypo{s_1 = t_1, s_2 = t_2, R(s_1,s_2) \seq R(t_1,t_2)}
            \infer1[\eqref{eq:ax:3}]{t=s, t=t ,t=v\seq s=v}
            \infer[double]1[\eqref{eq:exchange:left}]{t=t, t=s, t=v \seq s=v}
            \infer2[\eqref{eq:cut}]{t=s, s=v\seq s=v}
            \infer2[\eqref{eq:cut}]{s = t, t = v\seq s = v}
        \end{prooftree}
    \end{equation*}
    should suffice.

    Proving \eqref{eq:sequential symmetry} is very simple:
    it is a special case of the following proof of
    \eqref{eq:substitution}, with \(n=1\), and will therefore
    be left as an optional exercise.

    To prove the general case of \eqref{eq:substitution},
    we induce on the number of logical symbols \(n\) in \(A\).
    In the base case, \(n = 0\) and \(A\) is a predicate constant \(R\).
    The claim then holds by axiom \eqref{eq:ax:3}.
    For the inductive hypothesis \IH{}, we assume that the claim holds
    for \(n\leq k\) symbols in \(A\). We then add the known logical symbols
    to \(A\) one at a time to close the induction.

    If \(A = \lnot B\), denote \(\bar s = s_1,\ldots,s_n\) and
    \(\bar t = t_1,\ldots,t_n\). Then
    \begin{equation*}
        \begin{prooftree}
            \hypo{\eqref{eq:symmetry}}
            \ellipsis{}{s_2 = t_2 \seq t_2 = s_2}
            \hypo{\eqref{eq:symmetry}}
            \ellipsis{}{s_1=t_1\seq t_1=s_1}
            \hypo{\IH}
            \ellipsis{}{t_1=s_1,\ldots,t_n=s_n,A(\bar t)\seq A(\bar s)}
            \infer1[\eqref{eq:lnot:left}]{\lnot A(\bar s), t_1=s_1,\ldots,t_n=s_n,A(\bar t)\seq}
            \infer[double]1[\eqref{eq:exchange:left}]{A(\bar t), t_1=s_1,\ldots,t_n=s_n,\lnot A(\bar s)\seq}
            \infer1[\eqref{eq:lnot:right}]{t_1=s_1,\ldots,t_n=s_n,\lnot A(\bar s)\seq \lnot A(\bar t)}
            \infer2[\eqref{eq:cut}]{s_1=t_1,t_2=s_2,\ldots,t_2=s_2,\lnot A(\bar s)\seq \lnot A(\bar t)}
            \infer2[\eqref{eq:cut}]{}
            \ellipsis{n×\eqref{eq:cut} and \eqref{eq:exchange:left}}{}
            \infer1{s_1 = t_1, \ldots,s_n=t_n, \lnot A(\bar s) \seq \lnot A(\bar t)}
        \end{prooftree}
    \end{equation*}
    If \(A = B\land C\), we proceed as follows:
    \small
    \begin{equation*}
        \begin{prooftree}
            \hypo{\IH}
            \ellipsis{}{s_1 = t_1, \ldots,s_n=t_n, B(\bar s) \seq B(\bar t)}
            \infer1{n\times\eqref{eq:exchange:left}~\&~\eqref{eq:land:left}~\&~n\times\eqref{eq:exchange:left}}
            \infer1{s_1 = t_1, \ldots,s_n=t_n, B(\bar s)\land C(\bar s) \seq B(\bar t)}
            \hypo{\IH}
            \ellipsis{}{s_1 = t_1, \ldots,s_n=t_n, C(\bar s) \seq C(\bar t)}
            \infer1{n\times\eqref{eq:exchange:left}~\&~\eqref{eq:land:left}~\&~n\times\eqref{eq:exchange:left}}
            \infer1{s_1 = t_1, \ldots,s_n=t_n, B(\bar s)\land C(\bar s) \seq C(\bar t)}
            \infer2[\eqref{eq:land:right}]{s_1 = t_1, \ldots,s_n=t_n, B(\bar s)\land C(\bar s) \seq B(\bar t)\land C(\bar t)}
        \end{prooftree}
    \end{equation*}
    \normalsize
    The case \(A = B\lor C\) goes almost identically, and is left as an optional exercise.

For the case \(A = B\limplies C\) we denote the sequence of equalities
\(s_1=t_1,\ldots,s_n=t_n\) with \(S\simeq T\). We can then argue as follows:

    \begin{equation*}
        \begin{prooftree}
            \hypo{\eqref{eq:sequential symmetry}}
            \ellipsis{}{S\simeq{}T\seq T\simeq S}
            \hypo{\IH}
            \ellipsis{}{S\simeq T, B(\bar t)\seq B(\bar s)}
            \infer2[\(n\times\)\eqref{eq:cut}]{S\simeq T,B(\bar t)\seq B(\bar s)}
            \hypo{\IH}
            \ellipsis{}{S\simeq T,C(\bar s)\seq C(\bar t)}
            \infer[double]1[\eqref{eq:exchange:left}]{C(\bar s), S\simeq{}T\seq C(\bar t)}
            \infer2[\eqref{eq:limplies:left}]{B(\bar s)\limplies C(\bar s), B(\bar t),S\simeq{}T\seq C(\bar t)}
            \infer[double]1[\eqref{eq:exchange:left}]{B(\bar t),S\simeq T, B(\bar s)\limplies C(\bar s)\seq C(\bar t)}
            \infer1[\eqref{eq:limplies:right}]{S\simeq T, B(\bar s)\limplies C(\bar s)\seq B(\bar t)\limplies C(\bar t)}
        \end{prooftree}
    \end{equation*}

    In the case of \(A = \forall xB(x)\),
    the additional symbol is actually an additional
    independent variable \(a\). Hence we can argue as follows:
    \begin{equation*}
        \begin{prooftree}
            \hypo{\IH}
            \ellipsis{}{S\simeq T,B(a,\bar s)\seq B(a,\bar t)}
            \infer[double]1[\eqref{eq:exchange:left}]{B(a,\bar s),S\simeq{}T\seq B(a,\bar t)}
            \infer1[\eqref{eq:forall:left}]{\forall xB(x,\bar s),S\simeq{}T\seq B(a,\bar t)}
            \infer1[\eqref{eq:forall:right}]{\forall xB(x,\bar s),S\simeq{}T\seq\forall xB(x,\bar t)}
            \infer[double]1[\eqref{eq:exchange:left}]{S\simeq T, \forall xB(x,\bar s)\seq\forall xB(x,\bar t)}
        \end{prooftree}
    \end{equation*}
    The case where \(A = \exists xB(x)\) is very similar to the case \(A = \forall xB(x)\),
    and will therefore be left as an extra exercise,h closes the induction.
\end{proof}

\begin{definition}[Essential formula]\label{def:essential formula}
    If the eliminated formula  in the application of \eqref{eq:cut}
    is \(s=t\), the cut is called \emph{unessential}.
    Otherwise the formula is \emph{essential}.
\end{definition}

\begin{theorem}[\eqref{eq:cut}-elimination in \(\LK_=\)]%
    \label{the:cut-elimination in LK=}
    If a sequent \(\Gamma\seq\Delta\) is provable in \(\LK_=\) with an essential
    \eqref{eq:cut}, then it is also provable without an essential \eqref{eq:cut}.
\end{theorem}

\begin{proof}
    \begin{exercise}[7.6]\label{exe:7.6}
        Left as an exercise. This is very similar to the general
        Gentzens Hauptsatz, with only two additional subcases:
        either \eqref{eq:cut}--\eqref{eq:mix} concerns an equality
        axiom or it is obtained by weakening.
    \end{exercise}
\end{proof}
