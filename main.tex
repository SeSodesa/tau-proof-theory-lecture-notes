% Document information
\def\myauthor{Esko Turunen and Santtu Söderholm}
\def\mycourse{Mathematical Logic -- Proof Theory}
\def\mytitle{Lecture Notes}
\def\myschool{Tampere University}
\def\implementation{Spring 2021}
% Document class
\documentclass[11pt,a4paper]{article}
% Fonttiasetuksia
\usepackage[T1]{fontenc}
\usepackage[utf8]{inputenc}
% Document metadata
\begin{filecontents}[overwrite]{\jobname.xmpdata}
\Title{\mycourse\ \mytitle, \implementation}
\Author{\myauthor}
\Publisher{\myschool}
\Subject{A lecture diary for the Tampere Universiry course ''Proof Theory''.}
\end{filecontents}
% Tavutuksia
\usepackage[finnish, main=english]{babel}
% Parnneltu matikkapaketti
\usepackage{mathtools}
% Paperikokoasetuksia
\usepackage{geometry}
\usepackage{amssymb}
% Ei sisennettyjä kappaleita
\usepackage{parskip}
% Robustimpi komentosyntaksi
\usepackage{xparse}
% Kuvia
\usepackage{tikz}
\usetikzlibrary{arrows, automata}
% Taulukoita
\usepackage{tabularx}
\usepackage{booktabs}
% Fontteja
% Aktivoidaan itse harjoituspohjan aluksi
% komennolla \fontfamily{put}\selectfont
\usepackage{fourier}
\usepackage{fontspec}
\setmainfont{Heuristica}
\setmonofont{DejaVu Sans Mono}
% Kuva-, taulukko- sun muut tekstit
\usepackage{caption}
\captionsetup{labelfont=bf}
\usepackage{csquotes}
% Boldattua matematiikkaa
\usepackage{bm}
% Arkistomuotoinen PDF
\usepackage[a-2b,mathxmp]{pdfx}
  \hypersetup{hidelinks}
\usepackage{url}

% ----- Komentoja -----

\NewDocumentCommand{\Nset}{}{\mathbb{N}}
\NewDocumentCommand{\Zset}{}{\mathbb{Z}}
\NewDocumentCommand{\Qset}{}{\mathbb{Q}}
\NewDocumentCommand{\Rset}{}{\mathbb{R}}
\NewDocumentCommand{\Cset}{}{\mathbb{C}}
\NewDocumentCommand{\Pset}{}{\mathbb{P}}


% Document start
\begin{document}
\pagenumbering{gobble}
\begin{titlepage}
\centering
\vspace*{0.1\textheight}
\Huge\mycourse{}

\huge\mytitle{}

\vspace{0.4\pageheight}

\Large\myauthor\\\myschool{}\\\implementation{}
\end{titlepage}

\tableofcontents
\pagebreak
\pagenumbering{arabic}

% -----  Main matter -----

\section{What is mathematics about?}

To answer the question set in the title of this chapter, \emph{proofs}.

\begin{example}[Proof by contradiction]
The infinite set of natural numbers \(\Nset = \set{1, 2, 3, \ldots}\) should already be familiar to you.
It is closed under addition \(+\) and multiplication \(\mult\).
In other words, if \(a, b\in\Nset\), then \(a + b \in\Nset\) and \(a\mult b\in\Nset\).
But what about division \(\div\)? We know that \(6 \div 2 = 3 \in\Nset\),
but \(3\div 6 = 1 / 2 \notin\Nset\).

It is known that some natural numbers are products of two \emph{other} natural numbers,
while others are not. For example, \(6 = 2 \mult 3\) and \(20 = 4 \mult 5\),
but
\[
2 = 2\mult 1,
\quad
3 = 3\mult 1,
\quad
5 = 5 \mult 1,
\quad
7 = 7 \mult 1,
\ldots\,.
\]
\end{example}
These numbers whose factors only consist of themselves and the number \(1\) are called \emph{prime numbers}
and denoted with \(\Pset\). How many prime numbers are there? We show that there are infinitely many via
\emph{proof by contradiction}.

Before we begin, let us analyze the products of prime numbers a bit more closely:
\begin{align*}
    2 \mult 3                   &= 6,   & 6 + 1     &= 7 \in\Pset \\
    2\mult 3 \mult 5            &= 30   & 30 + 1    &= 31 \in\Pset \\
    2 \mult 3 \mult 5 \mult 7   &= 210  & 210 + 1   &= 211\in\Pset\\
                                &\vdots &           &\vdots
\end{align*}
It seems like when the number \(1\) is added to any product of primes, the resulting number is prime.
We might also ask ourselves, whether the product of primes
\begin{equation*}\tag{hypothesis}
    \prod = 2 \mult 3 \mult 5 \mult\cdots\mult p_n \mult p_n+1 \in\Pset \,.
\end{equation*}
for all \(n\in\Nset\).

To prove this hypothesis, a natural seeming approach is to then make the \emph{contrapositive assumption},
that if \(p_i, p_n\in\Pset\), then
\begin{equation}
    \prod = 2\mult 3 \mult 5 \mult \cdots \mult p_i\mult\cdots \mult p_n \mult p_n + 1 = p_i\mult c \notin\Pset
\end{equation}
for some \(c\in\Nset\). But if this were the case, then
\begin{equation*}
    \frac\prod {p_i} = c = \frac{2 \mult 3 \mult 5 \mult \cdots \mult p_n + 1}{p_i} \notin\Nset\,.
\end{equation*}
This is in contradiction with our assumptions, which proves that there are infinitely many primes
(as the hypothessis provides us with a way of constructing them indefinitely).

\begin{example}[The derivative of a quadratic polynomial]
    See \url{https://www.youtube.com/watch?v=OhmexZXPVkE}.
\end{example}

\begin{example}[There exist non-rational real numbers]
    A rational number \(q\in\Qset\) is a real number,
    such that \(q = m\div n\) for some \(m,n\in\Zset\), where \(n\neq 0\).
    Real numbers \(r\in\Rset\) that are not rational are irrational,
    which can be stated bluntly as
    \begin{equation*}
        \nexists m, n \in \Zset, n\neq 0 : r = \frac m n : r \in \Rset - \Qset\,.
    \end{equation*}
    For example \(\sqrt 2\notin\Qset\).
\end{example}

\begin{theorem}
    There are \(a,b\notin\Qset\), so that \(a^b\in\Qset\).
    \label{the:a to b in Q}
\end{theorem}

\begin{proof}
    Let \(a, b = \sqrt 2\). If \(\sqrt 2 ^{\sqrt 2} \in\Qset\), then we are done.
    If this is not the case, let \(a = \sqrt2^{\sqrt 2}\) and \(b = \sqrt 2\).
    Then
    \begin{equation*}
        a^b = \arcs*{\sqrt2^{\sqrt 2}}^{\sqrt 2} = \sqrt 2 ^ 2 = 2 \in\Qset\,.
    \end{equation*}
    This concludes the proof.
\end{proof}

The proof of Theorem~\ref{the:a to b in Q} is an example of a \emph{non-constructive proof},
which are not allowed in \emph{non-intuitionistic} or \emph{constructive logic} \IK.

Classical logic \LK{} only concerns itself with the notions of ''true'' and ''false''.
In other words, classical logic functions based on the following principle:
\begin{quote}
    ''Any well-formed mathematical object \(A\) is either true or false;
    if \(A\) is false, then not-\(A\) is true.''
\end{quote}
\emph{Tertium non datur}, ''\(A\) or not \(A\)'' is alwaus true.
In intuitionistic logic \IK, this is not the case.
It imposes the additional constraint that for an object \(A\)
to exist in the first place, one has to be able to \emph{construct}
such an \(A\).

All of logic is based on an abstract language, that we now study.

\section{The language of classical logic \LK}

To be able to talk about any language, we must first define the set of symbols we are allowed to use.
This set is know as the \emph{alphabet}.

\begin{definition}[The alphabet of classical logic]%
    \label{def:classical alphabet}
    In the (formal) language of logic, valid strings consist of the following symbols:
    \begin{enumerate}
        \item\label{it:classical constants} Constants:
            \begin{enumerate}
                \item\label{it:classical constants individual}
                    Individual constants: \(k_j, j\in\Nset\cup\set 0\).
                \item\label{it:classical constants function}
                    Function constants with \(i\) argument places: \(f_j^i : i\in\Nset,j\in\Nset\cup\set0\).
                \item\label{it:classical constants predicates}
                    Predicate constants with \(i\) arguent places: \(R_j^i : i, j\in\Nset\)
            \end{enumerate}
        \item\label{it:classical variables} Variables:
            \begin{enumerate}
                \item\label{it:classical variables free} Free variables \(a_j, j\in\Nset\cup\set0\),
                \item\label{it:classical variables bound} Bound variables \(x_j, j\in\Nset\cup\set0\)
            \end{enumerate}
        \item\label{it:classical logical symbols} The logical symbols \(\lnot\) (not), \(\land\) (and), \(\lor\) (or), \(\limplies\) (implies),
            \(\forall\) (for all) and \(\exists\). The first four are called \emph{propositional connectives}
            and the last two are so-called \emph{quantifiers}.
        \item\label{it:classical auciliary symbols} Auxiliary symbols: parentheses \((\) and \()\), and the comma \(,\).
    \end{enumerate}
\end{definition}

\begin{definition}[Term]%
    \label{def:classical term}
    The \emph{terms} of the language of classical logic are defined recursively or inductively as follows:
    \begin{enumerate}
        \item\label{it:classical term 1}
            Every individual constant \(k_j\) is a term.
        \item\label{it:classical term 2}
            Every free variable \(a_j\) is a term.
        \item\label{it:classical term 3}
            If \(f^i\) is a function constant with \(i\) argument places and \(t_1,\ldots,t_i\) are terms,
            then \(f\arcs{t_1,\ldots,t_i}\) is a term.
        \item\label{it:classical term4}
            Terms are defined only by the items~\ref{it:classical term 1}--\ref{it:classical term 3} of this definition.
    \end{enumerate}
\end{definition}

\begin{definition}[Formulae and their outermost logical symbols]%
    \label{def:formulae and outermost logical symbols}
    If \(R^i\) is a predicate constant with \(i\) argument places and \(t_1,\ldots,t_i\) are terms,
    then \(R\arcs{t_1,\ldots,t_i}\) is an \emph{atomic formula}.
    \emph{Formulae} and their \emph{outermost logical symbols} are defined recursively as follows:
    \begin{enumerate}
        \item\label{it:formula 1}
            Every atomic formula is a formula. It has no outermost logical symbol.
        \item\label{it:formula 2}
            If \(A\) and \(B\) are fomulae, then \(\arcs{\lnot A}\), \(\arcs{A\land B}\),
            \(\arcs{A\lor B}\) and \(\arcs{A\limplies B}\) are formulae.
            Their outermost logical symbols are \(\lnot\), \(\land\), \(\lor\) and \(\limplies\),
            respectively.
        \item\label{it:formula 3}
            If \(A\) is a formula, \(a\) is a free variable and \(x\) a bound variable not occurring in \(A\),
            then \(\forall x A'\) and \(\exists x A'\) are formulae.
            Here \(A'\) refers to the expression obtained from \(A\),
            by replacing every occurrence of \(a\) with \(x\), wherever \(a\)
            occurs in \(A\).
            The outermost logical symbols of these formulae are \(\forall\) and \(\exists\), respectively.
        \item\label{it:formula 4}
            Formulae are only those expressions obtained by applying the items~\ref{it:formula 1}--\ref{it:formula 3}
            of this definition.
    \end{enumerate}
\end{definition}

\begin{definition}[The alphabet of formulae]
    The capital letters \(A_j, B_j, C_j, \ldots\) with or without the subindices \(j \in \Nset\cup\set0\) denote formulae.
    The Greek capital letters \(T_j, \Delta_j, \Pi_j,\ldots\) with or without the subindices \(j\in\Nset\cup\set0\)
    denote (finite) \emph{sequences} of formulae.
    \label{def:alphabet of formulae}
\end{definition}

For example, \(T\corr A_1, A_2(a), \exists x A_3(x), \forall y A_4(y)\) is a finite sequence,
as is \(\Delta \corr B_1, \exists x B_2(x)\).


\section{Sequents, inferences and proofs in \LK}

\begin{definition}[Sequents, antecedents and succedents]%
    \label{def:sequent}
    If \(T\) and \(\Delta\) are sequences of formulae,
    then the expression \(T\seq\Delta\) is the \emph{sequent} of the two sequences.
    Here \(T\) is the \emph{antecedent} and \(\Delta\) the \emph{succedent}
    of the sequent.
\end{definition}

The natural meaning of \(A_1, \ldots,A_m\seq B_1,\ldots,B_n\) is that
\begin{center}
''If \(A_1\) and \(\cdots\) and \(A_m\), then \(B_1\) or \(\cdots\) or \(B_n\)''.
\end{center}
Note that the symbol \(\seq\) does not correspond to \(\limplies\).

\begin{definition}[Axioms]%
    \label{def:axiom}
    Sequents of the form \(A\seq A\) are called \emph{initial sequents} or \emph{axioms}.
\end{definition}

\begin{definition}[Contradiction and tautology]%
    \label{def:contradiction and tautology}
    A sequent without a succedent such as \(A_1, \ldots,A_m\seq\)
    is used to denote that \(A_1, \ldots,A_m\) yields a \emph{contradiction}.
    A sequent without an antecedent such as \(\seq B_1,\ldots,B_m\) means that
    \(B_1,\ldots,B_m\) is a \emph{tautology}, or that \(B_1,\ldots,B_m\) holds.
    A sequent with neither an antecedent or a succedent, as in \(\seq\),
    means that \emph{There is a contradiction}.
\end{definition}

\begin{definition}[Inferences, lower and upper sequents]%
    \label{def:inference}
    An \emph{inference} is an expression
    \begin{center}
    \begin{prooftree}
        \hypo{S_1}\infer1{S}
    \end{prooftree}
    \quad
    or
    \quad
    \begin{prooftree}
        \hypo{S_1}\hypo{S_2}\infer2{S}
    \end{prooftree}
    \end{center}
    where \(S_1\) and \(S_2\) are \emph{upper sequents} and \(S\) is a \emph{lower sequent}.
\end{definition}

The meaning of this notation is that if \(S_1\) and/or \(S_2\) are given,
we can infer \(S\) from them.

\begin{definition}[Inference rules in \LK]%
\label{def:inference rules in LK}
The following \emph{left} and \emph{right inference rules} are allowed in classical logic.
\begin{enumerate}
    \item\label{it:structural rules} Structural rules:
        \begin{enumerate}
            \item\label{it:weakening} Weakening:
                \begin{center}
                    left:
                    \begin{prooftree}
                        \hypo{\Gamma\seq\Delta}
                        \infer1{D, \Gamma\seq\Delta}
                    \end{prooftree}
                    \quad
                    right:
                    \begin{prooftree}
                        \hypo{\Gamma\seq\Delta}
                        \infer1{\Gamma\seq\Delta, D}
                    \end{prooftree}
                \end{center}
                Here \(D\) is called the \emph{weakening formula}.
            \item\label{it:contraction} Contraction:
                \begin{center}
                    left:
                    \begin{prooftree}
                        \hypo{D, D, \Gamma\seq\Delta}
                        \infer1{D, \Gamma\seq\Delta}
                    \end{prooftree}
                    \quad
                    right:
                    \begin{prooftree}
                        \hypo{\Gamma\seq\Delta, D, D}
                        \infer1{\Gamma\seq\Delta, D}
                    \end{prooftree}
                \end{center}
            \item\label{it:exchange} Exchange:
                \begin{center}
                    left:
                    \begin{prooftree}
                        \hypo{\Gamma,C, D, \Pi\seq\Delta}
                        \infer1{\Gamma, D, C, \Pi\seq\Delta}
                    \end{prooftree}
                    \quad
                    right:
                    \begin{prooftree}
                        \hypo{\Gamma\seq\Delta, C, D, \Lambda}
                        \infer1{\Gamma\seq\Delta, D, C, \Lambda}
                    \end{prooftree}
                \end{center}
            \item\label{it:cut} Cut:
                \begin{center}
                    left:
                    \begin{prooftree}
                        \hypo{\Gamma\seq\Delta, D}
                        \hypo{D, \Pi\seq\lambda}
                        \infer2{\Gamma, \Pi\seq\Delta, \Lambda}
                    \end{prooftree}
                \end{center}
        \end{enumerate}
    \item\label{it:logical rules} Logical rules:
        \begin{enumerate}
            \item\label{it:negation} Negation:
                \begin{center}
                    \(\lnot\):left:
                    \begin{prooftree}
                        \hypo{\Gamma\seq\Delta, D}
                        \infer1{\lnot D, \Gamma\seq\Delta}
                    \end{prooftree}
                    \qquad
                    \(\lnot\):right:
                    \begin{prooftree}
                        \hypo{D, \Gamma\seq\Delta}
                        \infer1{\Gamma\seq\Delta, \lnot D}
                    \end{prooftree}
                \end{center}
                Here \(D\) and \(\lnot D\) are the \emph{auxiliary} and \emph{principal} formulae
                of this inference, respectively.
            \item\label{it:conjugation} Conjugation:
                \begin{center}
                    \(\land\):left:
                    \begin{prooftree}
                        \hypo{C, \Gamma\seq\Delta}
                        \infer1{C\land D, \Gamma\seq\Delta}
                    \end{prooftree}
                    \qquad
                    \(\land\):right:
                    \begin{prooftree}
                        \hypo{D, \Gamma\seq\Delta}
                        \infer1{C\land D\Gamma\seq\Delta}
                    \end{prooftree}
                \end{center}
                Here \(C\) and \(D\) are the \emph{auxiliary} formulae and \(C\land D\) the \emph{principal} formula
                of this inference.
            \item\label{it:disjunction} Disjunction:
                \begin{center}
                    \(\lor\):left:
                    \begin{prooftree}
                        \hypo{\Gamma\seq\Delta, C}
                        \infer1{\Gamma\seq\Delta, C\land D}
                    \end{prooftree}
                    \qquad
                    \(\lor\):right:
                    \begin{prooftree}
                        \hypo{\Gamma\seq\Delta, D}
                        \infer1{\Gamma\seq\Delta, C\land D}
                    \end{prooftree}
                \end{center}
                Here \(C\) and \(D\) are the \emph{auxiliary} formulae and \(C\lor D\) the \emph{principal} formula
                of this inference.
            \item\label{implication} Implication:
                \begin{center}
                    \(\limplies\):left:
                    \begin{prooftree}
                        \hypo{\Gamma\seq\Delta, C, D, \Pi\seq\Lambda}
                        \infer1{C\limplies D, \Gamma, \Pi\limplies\Delta, \Lambda}
                    \end{prooftree}
                    \qquad
                    \(\limplies\):right:
                    \begin{prooftree}
                        \hypo{C, \Gamma\seq\Delta, D}
                        \infer1{\Gamma\seq\Delta, C\limplies D}
                    \end{prooftree}
                \end{center}
                Here \(C\) and \(D\) are the \emph{auxiliary} formulae and \(C\limplies D\) the \emph{principal} formula
                of this inference.
            \item\label{it:universality} Universality:
                \begin{center}
                    \(\forall\):left:
                    \begin{prooftree}
                        \hypo{F(t), \Gamma\seq\Delta}
                        \infer1{\forall x F(x), \Gamma\seq\Delta}
                    \end{prooftree}
                    \qquad
                    \(\forall\):right:
                    \begin{prooftree}
                        \hypo{\Gamma\seq\Delta, F(a)}
                        \infer1{\Gamma\seq\Delta, \forall xF(x)}
                    \end{prooftree}
                \end{center}
                Here \(t\) is an arbitrary term and \(a\) does not occur in the lower sequent.
                \(F(t)\) and \(F(a)\) are the \emph{auxiliary formulae}, whereas \(\forall xF(x)\)
                is the \emph{principal formula} of this inference.
            \item\label{it:existence} Existence:
                \begin{center}
                    \(\exists\):left:
                    \begin{prooftree}
                        \hypo{F(a), \Gamma\seq\Delta}
                        \infer1{\exists{}x F(x), \Gamma\seq\Delta}
                    \end{prooftree}
                    \qquad
                    \(\exists\):right:
                    \begin{prooftree}
                        \hypo{\Gamma\seq\Delta, F(t)}
                        \infer1{\Gamma\seq\Delta, \exists{}xF(x)}
                    \end{prooftree}
                \end{center}
                Again, here \(a\) does not occur in the lower sequent and \(t\) is an arbitrary term.
                \(F(t)\) and \(F(a)\) are the \emph{auxiliary formulae}, whereas \(\forall xF(x)\)
                is the \emph{principal formula} of this inference.

        \end{enumerate}
\end{enumerate}
\end{definition}

\begin{theorem}[Uniqueness of the axiom]%
    \label{the:uniqueness of the axiom}
    The sequent \(A\seq A\) is the only axiom.
\end{theorem}

\begin{proof}

\end{proof}

\section{Metalanguage versus object langauge}

\section{Examples of proofs in \LK}

\section{Intuitionistic logic or \IK, proofs in \IK}

\section{The relation between \LK{} and \IK}

\section{Axiom systems}

\section{The Cut-elimination theorems of \LK}

\section{Consequences of Cut-elimination -- Consistency of \LK{} an \IK}

\section{Completeness of \LK}

\subsection{Structures, assignments, interpretations and validity}

\subsection{Soundness of \LK}

\subsection{Reduction trees, completeness of \LK}

\section{Predicate calculus with equality}

\section{Peano arithmetic or PA}

\section{Primitive recursive functions PRF and prmitive recursive relations PRR}

\section{The relation between Peano arithmetic and natural numbers}

\section{Gödel numbering}

\section{Arithmetization in Peano arithmetic}

\section{Tarski's theorem}

\section{Gödel's incompleteness theorems}

\end{document}

