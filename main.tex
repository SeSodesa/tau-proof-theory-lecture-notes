% Document information
\def\myauthor{Esko Turunen and Santtu Söderholm}
\def\mycourse{Mathematical Logic -- Proof Theory}
\def\mytitle{Lecture Notes}
\def\myschool{Tampere University}
\def\implementation{Spring 2021}
% Document class
\documentclass[11pt,a4paper]{article}
% Fonttiasetuksia
\usepackage[T1]{fontenc}
\usepackage[utf8]{inputenc}
% Document metadata
\begin{filecontents}[overwrite]{\jobname.xmpdata}
\Title{\mycourse\ \mytitle, \implementation}
\Author{\myauthor}
\Publisher{\myschool}
\Subject{A lecture diary for the Tampere Universiry course ''Proof Theory''.}
\end{filecontents}
% Tavutuksia
\usepackage[finnish, main=english]{babel}
% Parnneltu matikkapaketti
\usepackage{mathtools}
% Paperikokoasetuksia
\usepackage{geometry}
\usepackage{amssymb}
% Ei sisennettyjä kappaleita
\usepackage{parskip}
% Robustimpi komentosyntaksi
\usepackage{xparse}
% Kuvia
\usepackage{tikz}
\usetikzlibrary{arrows, automata}
% Taulukoita
\usepackage{tabularx}
\usepackage{booktabs}
% Fontteja
% Aktivoidaan itse harjoituspohjan aluksi
% komennolla \fontfamily{put}\selectfont
\usepackage{fourier}
\usepackage{fontspec}
\setmainfont{Heuristica}
\setmonofont{DejaVu Sans Mono}
% Kuva-, taulukko- sun muut tekstit
\usepackage{caption}
\captionsetup{labelfont=bf}
\usepackage{csquotes}
% Boldattua matematiikkaa
\usepackage{bm}
% Arkistomuotoinen PDF
\usepackage[a-2b,mathxmp]{pdfx}
  \hypersetup{hidelinks}
\usepackage{url}

% ----- Komentoja -----

\NewDocumentCommand{\Nset}{}{\mathbb{N}}
\NewDocumentCommand{\Zset}{}{\mathbb{Z}}
\NewDocumentCommand{\Qset}{}{\mathbb{Q}}
\NewDocumentCommand{\Rset}{}{\mathbb{R}}
\NewDocumentCommand{\Cset}{}{\mathbb{C}}
\NewDocumentCommand{\Pset}{}{\mathbb{P}}


% Document start
\begin{document}
\pagenumbering{gobble}
\begin{titlepage}
\centering
\vspace*{0.1\textheight}
\Huge\mycourse{}

\huge\mytitle{}

\vspace{0.4\pageheight}

\Large\myauthor\\\myschool{}\\\implementation{}
\end{titlepage}

\tableofcontents
\pagebreak
\pagenumbering{arabic}

% -----  Main matter -----

\section{What is mathematics about?}

To answer the question set in the title of this chapter, \emph{proofs}.

\begin{example}[Proof by contradiction]
The infinite set of natural numbers \(\Nset = \set{1, 2, 3, \ldots}\) should already be familiar to you.
It is closed under addition \(+\) and multiplication \(\mult\).
In other words, if \(a, b\in\Nset\), then \(a + b \in\Nset\) and \(a\mult b\in\Nset\).
But what about division \(\div\)? We know that \(6 \div 2 = 3 \in\Nset\),
but \(3\div 6 = 1 / 2 \notin\Nset\).

It is known that some natural numbers are products of two \emph{other} natural numbers,
while others are not. For example, \(6 = 2 \mult 3\) and \(20 = 4 \mult 5\),
but
\[
2 = 2\mult 1,
\quad
3 = 3\mult 1,
\quad
5 = 5 \mult 1,
\quad
7 = 7 \mult 1,
\ldots\,.
\]
\end{example}
These numbers whose factors only consist of themselves and the number \(1\) are called \emph{prime numbers}
and denoted with \(\Pset\). How many prime numbers are there? We show that there are infinitely many via
\emph{proof by contradiction}.

Before we begin, let us analyze the products of prime numbers a bit more closely:
\begin{align*}
    2 \mult 3                   &= 6,   & 6 + 1     &= 7 \in\Pset \\
    2\mult 3 \mult 5            &= 30   & 30 + 1    &= 31 \in\Pset \\
    2 \mult 3 \mult 5 \mult 7   &= 210  & 210 + 1   &= 211\in\Pset\\
                                &\vdots &           &\vdots
\end{align*}
It seems like when the number \(1\) is added to any product of primes, the resulting number is prime.
A natural seeming approach is to then make the contrapositive assumption that if \(p_i, p_n\in\Pset\),
a \emph{natural number}
\begin{equation}
b = 2\mult 3 \mult 5 \mult \cdots \mult p_i\mult\cdots \mult p_n \mult p_n + 1 = p_i\mult c \notin\Pset
\end{equation}
for some \(c\in\Nset\), then
\begin{equation*}
    \frac b {p_i} = c = \frac{2 \mult 3 \mult 5 \mult \cdots \mult p_n + 1}{p_i} \notin\Nset\,.
\end{equation*}
This is in contradiction with our assumptions, which proves that there are infinitely many primes.


\section{The language of classical logic}

\section{Sequents, inferences and proofs in LK}

\section{Metalanguage versus object langauge}

\section{Examples of proofs in LK}

\section{Intuitionistic logic or LI, proofs in LI}

\section{The relation between LK and LI}

\section{Axiom systems}

\section{The Cut-elimination theorems of LK}

\section{Consequences of Cut-elimination -- Consistency of LK an LJ}

\section{Completeness of LK}

\subsection{Structures, assignments, interpretations and validity}

\subsection{Soundness of LK}

\subsection{Reduction trees, completeness of LK}

\section{Predicate calculus with equality}

\section{Peano arithmetic or PA}

\section{Primitive recursive functions PRF and prmitive recursive relations PRR}

\section{The relation between Peano arithmetic and natural numbers}

\section{Gödel numbering}

\section{Arithmetization in Peano arithmetic}

\section{Tarski's theorem}

\section{Gödel's incompleteness theorems}

\end{document}

