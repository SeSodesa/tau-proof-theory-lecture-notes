% Document information
\def\myauthor{Esko Turunen and Santtu Söderholm}
\def\mycourse{Mathematical Logic -- Proof Theory}
\def\mytitle{Lecture Notes}
\def\myschool{Tampere University}
\def\implementation{Spring 2021}
% Document class
\documentclass[11pt,a4paper]{article}
% Fonttiasetuksia
\usepackage[T1]{fontenc}
\usepackage[utf8]{inputenc}
% Document metadata
\begin{filecontents}[overwrite]{\jobname.xmpdata}
\Title{\mycourse\ \mytitle, \implementation}
\Author{\myauthor}
\Publisher{\myschool}
\Subject{A lecture diary for the Tampere Universiry course ''Proof Theory''.}
\end{filecontents}
% Tavutuksia
\usepackage[finnish, main=english]{babel}
% Parnneltu matikkapaketti
\usepackage{mathtools}
% Paperikokoasetuksia
\usepackage{geometry}
\usepackage{amssymb}
% Ei sisennettyjä kappaleita
\usepackage{parskip}
% Robustimpi komentosyntaksi
\usepackage{xparse}
% Kuvia
\usepackage{tikz}
\usetikzlibrary{arrows, automata}
% Taulukoita
\usepackage{tabularx}
\usepackage{booktabs}
% Fontteja
% Aktivoidaan itse harjoituspohjan aluksi
% komennolla \fontfamily{put}\selectfont
\usepackage{fourier}
\usepackage{fontspec}
\setmainfont{Heuristica}
\setmonofont{DejaVu Sans Mono}
% Kuva-, taulukko- sun muut tekstit
\usepackage{caption}
\captionsetup{labelfont=bf}
\usepackage{csquotes}
% Boldattua matematiikkaa
\usepackage{bm}
% Arkistomuotoinen PDF
\usepackage[a-2b,mathxmp]{pdfx}
  \hypersetup{hidelinks}
\usepackage{url}

% ----- Komentoja -----

\NewDocumentCommand{\Nset}{}{\mathbb{N}}
\NewDocumentCommand{\Zset}{}{\mathbb{Z}}
\NewDocumentCommand{\Qset}{}{\mathbb{Q}}
\NewDocumentCommand{\Rset}{}{\mathbb{R}}
\NewDocumentCommand{\Cset}{}{\mathbb{C}}
\NewDocumentCommand{\Pset}{}{\mathbb{P}}


% Document start
\begin{document}
\pagenumbering{gobble}
\begin{titlepage}
\centering
\vspace*{0.1\textheight}
\Huge\mycourse{}

\huge\mytitle{}

\vspace{0.4\pageheight}

\Large\myauthor\\\myschool{}\\\implementation{}
\end{titlepage}

\tableofcontents
\pagebreak
\pagenumbering{arabic}

% -----  Main matter -----

\section{What is mathematics about?}

To answer the question set in the title of this chapter, \emph{proofs}.

\begin{example}[Proof by contradiction]
The infinite set of natural numbers \(\Nset = \set{1, 2, 3, \ldots}\) should already be familiar to you.
It is closed under addition \(+\) and multiplication \(\mult\).
In other words, if \(a, b\in\Nset\), then \(a + b \in\Nset\) and \(a\mult b\in\Nset\).
But what about division \(\div\)? We know that \(6 \div 2 = 3 \in\Nset\),
but \(3\div 6 = 1 / 2 \notin\Nset\).

It is known that some natural numbers are products of two \emph{other} natural numbers,
while others are not. For example, \(6 = 2 \mult 3\) and \(20 = 4 \mult 5\),
but
\[
2 = 2\mult 1,
\quad
3 = 3\mult 1,
\quad
5 = 5 \mult 1,
\quad
7 = 7 \mult 1,
\ldots\,.
\]
\end{example}
These numbers whose factors only consist of themselves and the number \(1\) are called \emph{prime numbers}
and denoted with \(\Pset\). How many prime numbers are there? We show that there are infinitely many via
\emph{proof by contradiction}.

Before we begin, let us analyze the products of prime numbers a bit more closely:
\begin{align*}
    2 \mult 3                   &= 6,   & 6 + 1     &= 7 \in\Pset \\
    2\mult 3 \mult 5            &= 30   & 30 + 1    &= 31 \in\Pset \\
    2 \mult 3 \mult 5 \mult 7   &= 210  & 210 + 1   &= 211\in\Pset\\
                                &\vdots &           &\vdots
\end{align*}
It seems like when the number \(1\) is added to any product of primes, the resulting number is prime.
We might also ask ourselves, whether the product of primes
\begin{equation*}\tag{hypothesis}
    \prod = 2 \mult 3 \mult 5 \mult\cdots\mult p_n \mult p_n+1 \in\Pset \,.
\end{equation*}
for all \(n\in\Nset\).

To prove this hypothesis, a natural seeming approach is to then make the \emph{contrapositive assumption},
that if \(p_i, p_n\in\Pset\), then
\begin{equation}
    \prod = 2\mult 3 \mult 5 \mult \cdots \mult p_i\mult\cdots \mult p_n \mult p_n + 1 = p_i\mult c \notin\Pset
\end{equation}
for some \(c\in\Nset\). But if this were the case, then
\begin{equation*}
    \frac\prod {p_i} = c = \frac{2 \mult 3 \mult 5 \mult \cdots \mult p_n + 1}{p_i} \notin\Nset\,.
\end{equation*}
This is in contradiction with our assumptions, which proves that there are infinitely many primes
(as the hypothessis provides us with a way of constructing them indefinitely).

\begin{example}[The derivative of a quadratic polynomial]
    See \url{https://www.youtube.com/watch?v=OhmexZXPVkE}.
\end{example}

\begin{example}[There exist non-rational real numbers]
    A rational number \(q\in\Qset\) is a real number,
    such that \(q = m\div n\) for some \(m,n\in\Zset\), where \(n\neq 0\).
    Real numbers \(r\in\Rset\) that are not rational are irrational,
    which can be stated bluntly as
    \begin{equation*}
        \nexists m, n \in \Zset, n\neq 0 : r = \frac m n : r \in \Rset - \Qset\,.
    \end{equation*}
    For example \(\sqrt 2\notin\Qset\).
\end{example}

\begin{theorem}
    There are \(a,b\notin\Qset\), so that \(a^b\in\Qset\).
    \label{the:a to b in Q}
\end{theorem}

\begin{proof}
    Let \(a, b = \sqrt 2\). If \(\sqrt 2 ^{\sqrt 2} \in\Qset\), then we are done.
    If this is not the case, let \(a = \sqrt2^{\sqrt 2}\) and \(b = \sqrt 2\).
    Then
    \begin{equation*}
        a^b = \arcs*{\sqrt2^{\sqrt 2}}^{\sqrt 2} = \sqrt 2 ^ 2 = 2 \in\Qset\,.
    \end{equation*}
    This concludes the proof.
\end{proof}

The proof of Theorem~\ref{the:a to b in Q} is an example of a \emph{non-constructive proof},
which are not allowed in \emph{non-intuitionistic} or \emph{constructive logic} \LJ.

Classical logic \LK{} only concerns itself with the notions of ''true'' and ''false''.
In other words, classical logic functions based on the following principle:
\begin{quote}
    ''Any well-formed mathematical object \(A\) is either true or false;
    if \(A\) is false, then not-\(A\) is true.''
\end{quote}
\emph{Tertium non datur}, ''\(A\) or not \(A\)'' is alwaus true.
In intuitionistic logic \LJ, this is not the case.
It imposes the additional constraint that for an object \(A\)
to exist in the first place, one has to be able to \emph{construct}
such an \(A\).

All of logic is based on an abstract language, that we now study.

\section{The language of classical logic \LK}

To be able to talk about any language, we must first define the set of symbols we are allowed to use.
This set is know as the \emph{alphabet}.

\begin{definition}[The alphabet of classical logic]%
    \label{def:classical alphabet}
    In the (formal) language of logic, valid strings consist of the following symbols:
    \begin{enumerate}
        \item\label{it:classical constants} Constants:
            \begin{enumerate}
                \item\label{it:classical constants individual}
                    Individual constants: \(k_j, j\in\Nset\cup\set 0\).
                \item\label{it:classical constants function}
                    Function constants with \(i\) argument places: \(f_j^i : i\in\Nset,j\in\Nset\cup\set0\).
                \item\label{it:classical constants predicates}
                    Predicate constants with \(i\) arguent places: \(R_j^i : i, j\in\Nset\)
            \end{enumerate}
        \item\label{it:classical variables} Variables:
            \begin{enumerate}
                \item\label{it:classical variables free} Free variables \(a_j, j\in\Nset\cup\set0\),
                \item\label{it:classical variables bound} Bound variables \(x_j, j\in\Nset\cup\set0\)
            \end{enumerate}
        \item\label{it:classical logical symbols} The logical symbols \(\lnot\) (not), \(\land\) (and), \(\lor\) (or), \(\limplies\) (implies),
            \(\forall\) (for all) and \(\exists\). The first four are called \emph{propositional connectives}
            and the last two are so-called \emph{quantifiers}.
        \item\label{it:classical auciliary symbols} Auxiliary symbols: parentheses \((\) and \()\), and the comma \(,\).
    \end{enumerate}
\end{definition}

\begin{definition}[Term]%
    \label{def:classical term}
    The \emph{terms} of the language of classical logic are defined recursively or inductively as follows:
    \begin{enumerate}
        \item\label{it:classical term 1}
            Every individual constant \(k_j\) is a term.
        \item\label{it:classical term 2}
            Every free variable \(a_j\) is a term.
        \item\label{it:classical term 3}
            If \(f^i\) is a function constant with \(i\) argument places and \(t_1,\ldots,t_i\) are terms,
            then \(f\arcs{t_1,\ldots,t_i}\) is a term.
        \item\label{it:classical term4}
            Terms are defined only by the items~\ref{it:classical term 1}--\ref{it:classical term 3} of this definition.
    \end{enumerate}
\end{definition}

\begin{definition}[Formulae and their outermost logical symbols]%
    \label{def:formulae and outermost logical symbols}
    If \(R^i\) is a predicate constant with \(i\) argument places and \(t_1,\ldots,t_i\) are terms,
    then \(R\arcs{t_1,\ldots,t_i}\) is an \emph{atomic formula}.
    \emph{Formulae} and their \emph{outermost logical symbols} are defined recursively as follows:
    \begin{enumerate}
        \item\label{it:formula 1}
            Every atomic formula is a formula. It has no outermost logical symbol.
        \item\label{it:formula 2}
            If \(A\) and \(B\) are fomulae, then \(\arcs{\lnot A}\), \(\arcs{A\land B}\),
            \(\arcs{A\lor B}\) and \(\arcs{A\limplies B}\) are formulae.
            Their outermost logical symbols are \(\lnot\), \(\land\), \(\lor\) and \(\limplies\),
            respectively.
        \item\label{it:formula 3}
            If \(A\) is a formula, \(a\) is a free variable and \(x\) a bound variable not occurring in \(A\),
            then \(\forall x A'\) and \(\exists x A'\) are formulae.
            Here \(A'\) refers to the expression obtained from \(A\),
            by replacing every occurrence of \(a\) with \(x\), wherever \(a\)
            occurs in \(A\).
            The outermost logical symbols of these formulae are \(\forall\) and \(\exists\), respectively.
        \item\label{it:formula 4}
            Formulae are only those expressions obtained by applying the items~\ref{it:formula 1}--\ref{it:formula 3}
            of this definition.
    \end{enumerate}
\end{definition}

\begin{definition}[The alphabet of formulae]
    The capital letters \(A_j, B_j, C_j, \ldots\) with or without the subindices \(j \in \Nset\cup\set0\) denote formulae.
    The Greek capital letters \(T_j, \Delta_j, \Pi_j,\ldots\) with or without the subindices \(j\in\Nset\cup\set0\)
    denote (finite) \emph{sequences} of formulae.
    \label{def:alphabet of formulae}
\end{definition}

For example, \(T\corr A_1, A_2(a), \exists x A_3(x), \forall y A_4(y)\) is a finite sequence,
as is \(\Delta \corr B_1, \exists x B_2(x)\).


\section{Sequents, inferences and proofs in \LK}

\begin{definition}[Sequents, antecedents and succedents]%
    \label{def:sequent}
    If \(T\) and \(\Delta\) are sequences of formulae,
    then the expression \(T\seq\Delta\) is the \emph{sequent} of the two sequences.
    Here \(T\) is the \emph{antecedent} and \(\Delta\) the \emph{succedent}
    of the sequent.
\end{definition}

The natural meaning of \(A_1, \ldots,A_m\seq B_1,\ldots,B_n\) is that
\begin{center}
''If \(A_1\) and \(\cdots\) and \(A_m\), then \(B_1\) or \(\cdots\) or \(B_n\)''.
\end{center}
Note that the symbol \(\seq\) does not correspond to \(\limplies\).

\begin{definition}[Axioms]%
    \label{def:axiom}
    Sequents of the form \(A\seq A\) are called \emph{initial sequents} or \emph{axioms}.
\end{definition}

\begin{definition}[Contradiction and tautology]%
    \label{def:contradiction and tautology}
    A sequent without a succedent such as \(A_1, \ldots,A_m\seq\)
    is used to denote that \(A_1, \ldots,A_m\) yields a \emph{contradiction}.
    A sequent without an antecedent such as \(\seq B_1,\ldots,B_m\) means that
    \(B_1,\ldots,B_m\) is a \emph{tautology}, or that \(B_1,\ldots,B_m\) holds.
    A sequent with neither an antecedent or a succedent, as in \(\seq\),
    means that \emph{There is a contradiction}.
\end{definition}

\begin{definition}[Inferences, lower and upper sequents]%
    \label{def:inference}
    An \emph{inference} is an expression
    \begin{center}
    \begin{prooftree}
        \hypo{S_1}\infer1{S}
    \end{prooftree}
    \quad
    or
    \quad
    \begin{prooftree}
        \hypo{S_1}\hypo{S_2}\infer2{S}
    \end{prooftree}
    \end{center}
    where \(S_1\) and \(S_2\) are \emph{upper sequents} or \emph{predecessors} and \(S\)
    is a \emph{lower sequent} or \emph{successor}.
\end{definition}

The meaning of this notation is that if \(S_1\) and \(S_2\) are given,
\(S\) can be inferred from them. This inference could consist of multiple
individual inferences in sequence.

\begin{definition}[Proofs in \LK]\label{def:proof in LK}
    In classical logic \LK, a \emph{proof} \(P\)
    is a \emph{tree} of sequents, where
    \begin{enumerate}
        \item
            the leaves or \emph{premises} are axioms \(A\seq A\),
        \item
            the root or \emph{conclusion} of \(P\) has no lower sequents (or successors) and
        \item
            for every other sequent \(T_i\seq\Delta_i\in P\)
            there exists a sequent \(T_j\seq\Delta_j\in P\), so that
            \begin{equation*}
            \begin{prooftree}
                \hypo{T_i\seq\Delta_i}
                \infer1{T_j\seq\Delta_j}
            \end{prooftree}\,.
            \end{equation*}
           .
    \end{enumerate}
\end{definition}

\begin{definition}[Inference rules in \LK]%
\label{def:inference rules in LK}
The following \emph{left} and \emph{right inference rules} are allowed in classical logic.
\begin{enumerate}
    \item\label{it:structural rules} Structural rules:
        \begin{enumerate}
            \item\label{it:weakening} Weakening:
                \begin{equation}
                    \tag{weakening:left}
                    \label{eq:weakening:left}
                    \begin{prooftree}
                        \hypo{\Gamma\seq\Delta}
                        \infer1{D, \Gamma\seq\Delta}
                    \end{prooftree}
                \end{equation}
                \begin{equation}
                    \tag{weakening:right}
                    \label{eq:weakening:right}
                    \begin{prooftree}
                        \hypo{\Gamma\seq\Delta}
                        \infer1{\Gamma\seq\Delta, D}
                    \end{prooftree}
                \end{equation}
                Here \(D\) is called the \emph{weakening formula}.
            \item\label{it:contraction} Contraction:
                \begin{equation}
                    \tag{contraction:left}
                    \label{eq:contraction:left}
                    \begin{prooftree}
                        \hypo{D, D, \Gamma\seq\Delta}
                        \infer1{D, \Gamma\seq\Delta}
                    \end{prooftree}
                \end{equation}
                \begin{equation}
                    \tag{contraction:right}
                    \label{eq:contraction:right}
                    \begin{prooftree}
                        \hypo{\Gamma\seq\Delta, D, D}
                        \infer1{\Gamma\seq\Delta, D}
                    \end{prooftree}
                \end{equation}
            \item\label{it:exchange} Exchange:
                \begin{equation}
                    \tag{exchange:left}
                    \label{eq:exchange:left}
                    \begin{prooftree}
                        \hypo{\Gamma,C, D, \Pi\seq\Delta}
                        \infer1{\Gamma, D, C, \Pi\seq\Delta}
                    \end{prooftree}
                \end{equation}
                \begin{equation}
                    \tag{exchange:right}%
                    \label{eq:exchange:right}
                    \begin{prooftree}
                        \hypo{\Gamma\seq\Delta, C, D, \Lambda}
                        \infer1{\Gamma\seq\Delta, D, C, \Lambda}
                    \end{prooftree}
                \end{equation}
            \item\label{it:cut} Cut:
                \begin{equation}
                    \tag{cut}
                    \label{eq:cut}
                    \begin{prooftree}
                        \hypo{\Gamma\seq\Delta, D}
                        \hypo{D, \Pi\seq\Lambda}
                        \infer2{\Gamma, \Pi\seq\Delta, \Lambda}
                    \end{prooftree}
                \end{equation}
        \end{enumerate}
    \item\label{it:logical rules} Logical rules:
        \begin{enumerate}
            \item\label{it:negation} Negation:
                \begin{equation}
                    \tag{\(\lnot\):left}
                    \label{eq:lnot:left}
                    \begin{prooftree}
                        \hypo{\Gamma\seq\Delta, D}
                        \infer1{\lnot D, \Gamma\seq\Delta}
                    \end{prooftree}
                \end{equation}
                \begin{equation}
                    \tag{\(\lnot\):right}
                    \label{eq:lnot:right}
                    \begin{prooftree}
                        \hypo{D, \Gamma\seq\Delta}
                        \infer1{\Gamma\seq\Delta, \lnot D}
                    \end{prooftree}
                \end{equation}
                Here \(D\) and \(\lnot D\) are the \emph{auxiliary} and \emph{principal} formulae
                of this inference, respectively.
            \item\label{it:conjugation} Conjugation:
                \begin{equation}
                    \tag{\(\land\):left}
                    \label{eq:land:left}
                    \begin{prooftree}
                        \hypo{C, \Gamma\seq\Delta}
                        \infer1{C\land D, \Gamma\seq\Delta}
                    \end{prooftree}
                    \quad\text{and}\quad
                    \begin{prooftree}
                        \hypo{D,\Gamma\seq\Delta}
                        \infer1{C\land D, \Gamma\seq\Delta}
                    \end{prooftree}
                \end{equation}
                \begin{equation}
                    \tag{\(\land\):right}
                    \label{eq:land:right}
                    \begin{prooftree}
                        \hypo{\Gamma\seq\Delta, C}
                        \hypo{\Gamma\seq\Delta, D}
                        \infer2{\Gamma\seq\Delta, C\land D}
                    \end{prooftree}
                \end{equation}
                Here \(C\) and \(D\) are the \emph{auxiliary} formulae and \(C\land D\) the \emph{principal} formula
                of this inference.
            \item\label{it:disjunction} Disjunction:
                \begin{equation}
                    \tag{\(\lor\):left}
                    \label{eq:lor:left}
                    \begin{prooftree}
                        \hypo{C, \Gamma\seq\Delta}
                        \hypo{D, \Gamma\seq\Delta}
                        \infer2{C\lor D, \Gamma\seq\Delta}
                    \end{prooftree}
                \end{equation}
                \begin{equation}
                    \tag{\(\lor\):right}
                    \label{eq:lor:right}
                    \begin{prooftree}
                        \hypo{\Gamma\seq\Delta, C}
                        \infer1{\Gamma\seq\Delta, C\lor D}
                    \end{prooftree}
                    \quad\text{and}\quad
                    \begin{prooftree}
                        \hypo{\Gamma\seq\Delta, D}
                        \infer1{\Gamma\seq\Delta, C\lor D}
                    \end{prooftree}
                \end{equation}
                Here \(C\) and \(D\) are the \emph{auxiliary} formulae and \(C\lor D\) the \emph{principal} formula
                of this inference.
            \item\label{implication} Implication:
                \begin{equation}
                    \tag{\(\limplies\):left}
                    \label{eq:limplies:left}
                    \begin{prooftree}
                        \hypo{\Gamma\seq\Delta, C}
                        \hypo{D, \Pi\seq\Lambda}
                        \infer2{C\limplies D, \Gamma, \Pi\seq\Delta, \Lambda}
                    \end{prooftree}
                \end{equation}
                \begin{equation}
                    \tag{\(\limplies\):right}
                    \label{eq:limplies:right}
                    \begin{prooftree}
                        \hypo{C, \Gamma\seq\Delta, D}
                        \infer1{\Gamma\seq\Delta, C\limplies D}
                    \end{prooftree}
                \end{equation}
                Here \(C\) and \(D\) are the \emph{auxiliary} formulae and \(C\limplies D\) the \emph{principal} formula
                of this inference.
            \item\label{it:universality} Universality:
                \begin{equation}
                    \tag{\(\forall\):left}
                    \label{eq:forall:left}
                    \begin{prooftree}
                        \hypo{F(t), \Gamma\seq\Delta}
                        \infer1{\forall x F(x), \Gamma\seq\Delta}
                    \end{prooftree}
                \end{equation}
                \begin{equation}
                    \tag{\(\forall\):right}
                    \label{eq:forall:right}
                    \begin{prooftree}
                        \hypo{\Gamma\seq\Delta, F(a)}
                        \infer1{\Gamma\seq\Delta, \forall xF(x)}
                    \end{prooftree}
                \end{equation}
                Here \(t\) is an arbitrary term and \(a\) does not occur in the lower sequent.
                \(F(t)\) and \(F(a)\) are the \emph{auxiliary formulae}, whereas \(\forall xF(x)\)
                is the \emph{principal formula} of this inference.
            \item\label{it:existence} Existence:
                \begin{equation}
                    \tag{\(\exists\):left}
                    \label{eq:exists:left}
                    \begin{prooftree}
                        \hypo{F(a), \Gamma\seq\Delta}
                        \infer1{\exists{}x F(x), \Gamma\seq\Delta}
                    \end{prooftree}
                \end{equation}
                \begin{equation}
                    \tag{\(\exists\):right}
                    \label{eq:exists:right}
                    \begin{prooftree}
                        \hypo{\Gamma\seq\Delta, F(t)}
                        \infer1{\Gamma\seq\Delta, \exists{}xF(x)}
                    \end{prooftree}
                \end{equation}
                Again, here \(a\) does not occur in the lower sequent and \(t\) is an arbitrary term.
                \(F(t)\) and \(F(a)\) are the \emph{auxiliary formulae}, whereas \(\forall xF(x)\)
                is the \emph{principal formula} of this inference.
        \end{enumerate}
\end{enumerate}
\end{definition}

\paragraph{Note:}
Proof manipulation using inference rules always happens ''at the ends'' of sequents:
the first formula on the left of \(\seq\) and the last formula on the right of \(\seq\).

\begin{theorem}[Uniqueness of the axiom]%
    \label{the:uniqueness of the axiom}
    The sequent \(A\seq A\) is the only axiom.
\end{theorem}

\begin{proof}
    For any sequent \(T\seq\Lambda\), there exists a finite proof tree
    \begin{center}
    \begin{prooftree}
        \hypo{A\seq A}
        \infer1{\vdots}
        \hypo{B\seq B}
        \infer1{\vdots}
        \infer2{T_1\seq\Lambda_1}
        \hypo{C\seq C}
        \infer1{\vdots}
        \infer2{\vdots}
        \infer1{T\seq\Lambda}
    \end{prooftree}
    \end{center}
\end{proof}

If there is a tree whose root is \(\seq A\), then \(A\) is \LK-provable.
If a proof does not use the cut rule of definition~\ref{def:inference rules in LK}.\ref{it:cut},
then a proof is deemed \emph{cut free}.

\begin{definition}[Equivalence]\label{def:equivalence}
Two formulae \(A\) and \(B\) are \emph{equivalent},
if and only if
\begin{equation*}
\begin{prooftree}
    \hypo{\seq A\limplies B}
    \hypo{\seq B\limplies A}
    \infer2[\eqref{eq:land:right}]{\seq\arcs{A\limplies B}\land\arcs{B\limplies A}}
\end{prooftree}\,.
\end{equation*}
This is denoted with \(A\equiv B\).
\end{definition}

\section{Metalanguage versus object language}

In order to be able to discuss mathematics (or any other set of objects), we need to separate the ideas of
\emph{object language} and \emph{metalanguage}. Metalanguage is the natural language
that we use every day, whereas an object language consists of certain strings formed from
a certain alphabet, with a specific goal of describing a certain set of objects.
In this course, the object language is used to describe mathematical logic and
is formed from the definitions~\ref{def:classical alphabet}--\ref{def:inference}.
A hierarchical description of it is given in table~\ref{tab:language hierarchy}.

\begin{tabenv}{Object language hierarchy from highest to lowest. Objects higher in the hierarchy are constructed from objects that are lower in the hierarchy.}%
\label{tab:language hierarchy}
\small
\begin{tabular}{c|l|p{0.6\textwidth}}
    \toprule
    Object level& Object name & Examples\\
    \midrule
    6.  & Proofs    & {%
        \begin{prooftree}
            \hypo{A\seq{}A}
            \infer1{\vdots}
            \hypo{B\seq{}B}
            \infer1{\vdots}
            \infer2{T\seq\Delta}
        \end{prooftree}
        \qquad\qquad
        \begin{prooftree}
            \hypo{S_1}
            \ellipsis{}{}
            \hypo{S_2}
            \ellipsis{}{}
            \infer2{S_4}
            \hypo{S_3}
            \ellipsis{}{}
            \infer2{S}
        \end{prooftree}
    }\\
    \midrule
    5.  & Inferences & {
        \begin{prooftree}
            \hypo{S_1}
            \infer1{S}
        \end{prooftree}
        \qquad
        \begin{prooftree}
            \hypo{S_1}
            \hypo{S_2}
            \infer2{S}
        \end{prooftree}
    }\\
    \midrule
    4.  & Sequents  &   \(A_1,\ldots, A_m\seq B_1,\ldots,B_n\), \(T\seq\Delta\), \(\Pi\seq\Gamma\)\\
    \midrule
    3.  & Formulae  &   \(A, B, C,\ldots\), \(R\arcs{t_1,\ldots,t_n}\), \(a = b\), \(a + b = c\), \(0\mult a  = b \mult c\), \(a' = 0\),
                        \(\forall x\arcs{x + b = c}\), \(\exists x\arcs{x = a} \lor \lnot\exists z\arcs{z = b}\)\\
    \midrule
    2.  & Terms     &   \(0, \ldots\), \(a, b, c, \ldots\), \(0 + a\), \(a\mult b\), \(0'\),
i                       \(\arcs{a + b} \mult c\), \(t\), \(t_i\) \\
    \midrule
    1.  & Alphabet  &   \(0, \ldots\), \(a, b, c, \ldots\), \(x, y, z, \ldots\), \(+, \mult, ', \ldots\),
    \(\lnot, \land,\lor, \limplies, \forall, \exists, = \), \(R\arcs{,\ldots,}\), \((\), \()\)\\
    \bottomrule
\end{tabular}
\end{tabenv}

\section{Examples of proofs in \LK}

\subsection{Examples}

\begin{example}
    To prove \(\seq A \lor\lnot A\), we reason as follows:
\begin{equation*}
    \begin{prooftree}
        \hypo{A\seq A}
        \infer1[\eqref{eq:lnot:right}]{\seq A, \lnot A}
        \infer1[\eqref{eq:lor:right}]{\seq A, A\lor\lnot A}
        \infer1[\eqref{eq:exchange:right}]{\seq A\lor\lnot A, A}
        \infer1[\eqref{eq:lor:right}]{\seq A\lor\lnot A, A\lor\lnot A}
        \infer1[\eqref{eq:contraction:right}]{\seq A \lor\lnot A}
    \end{prooftree}
\end{equation*}
\end{example}

\begin{example}[Fully indicated axiom]\label{exa:fully indicated axiom}
    Fully indicated \(F(a)\) simply means that we may apply quantifier rules.
    To prove \(\seq\lnot\forall y\lnot F(y)\limplies\exists x F(x)\),
    we reason as follows:
\begin{equation*}
    \begin{prooftree}
        \hypo{F(a)\seq F(a)}
        \infer1[\eqref{eq:exists:right}]{F(a)\seq\exists x F(x)}
        \infer1[\eqref{eq:lnot:right}]{\seq\exists x F(x), \lnot F(a)}
        \infer1[\eqref{eq:forall:right}]{\seq\exists x F(x), \forall y \lnot F(y)}
        \infer1[\eqref{eq:lnot:left}]{\lnot\forall y\lnot F(y)\seq\exists x F(x)}
        \infer1[]{\seq\lnot\forall y\lnot F(y)\limplies\exists x F(x)}
    \end{prooftree}
\end{equation*}
\end{example}



\begin{example}[Exercise 2.5.1]\label{exa:2.5.1}
    To prove \(A\lor B\equiv \lnot\arcs{\lnot A\land\lnot B}\) we need to prove two things,
    according to the above note. First we show that
    \(\seq A\lor B\limplies\lnot\arcs{\lnot A\land\lnot B}\):

\begin{equation}\label{eq:equivalence example part 1}
    \begin{prooftree}
        % left branch
        \hypo{A\seq A}
        \infer1[\eqref{eq:lnot:left}]{\lnot A, A\seq}
        \infer1[\eqref{eq:land:left}]{\lnot A\land\lnot B, A\seq}
        \infer1[\eqref{eq:exchange:left}]{A, \lnot A\land\lnot B}
        % right branch
        \hypo{B\seq B}
        \infer1[\eqref{eq:lnot:left}]{\lnot B, B\seq}
        \infer1[\eqref{eq:land:left}]{\lnot A\land\lnot B, B\seq}
        \infer1[\eqref{eq:exchange:left}]{B, \lnot A\land\lnot B}
        % Trunk
        \infer2[\eqref{eq:lor:left}]{A\lor B, \lnot A \land\lnot B\seq}
        \infer1[\eqref{eq:exchange:left}]{\lnot A \land\lnot B, A\lor B\seq}
        \infer1[\eqref{eq:lnot:right}]{A\lor B \seq\lnot\arcs{\lnot A \land\lnot B}}
        \infer1[\eqref{eq:limplies:right}]{\seq A\lor B\limplies\lnot\arcs{\lnot A\land\lnot B}}
    \end{prooftree}
\end{equation}

Then we prove the statement \(\seq\lnot\arcs{\lnot A \land\lnot B} \limplies A\lor B\):

\begin{equation}\label{eq:equivalence example part 2}
    \begin{prooftree}
        \hypo{A\seq A}
        \infer1[\eqref{eq:lor:right}]{A\seq A\lor B}
        \infer1[\eqref{eq:lnot:right}]{\seq A\lor B, \lnot A}
        \hypo{B\seq B}
        \infer1[\eqref{eq:lor:right}]{B\seq A\lor B}
        \infer1[\eqref{eq:lnot:right}]{\seq A\lor B, \lnot B}
        \infer2[\eqref{eq:land:right}]{\seq A\lor B, \lnot A\land\lnot B}
        \infer1[\eqref{eq:lnot:left}]{\lnot\arcs{\lnot A \land\lnot B}\seq A\lor B}
        \infer1[\eqref{eq:limplies:right}]{\seq\lnot\arcs{\lnot A \land\lnot B}\limplies A\lor B}
    \end{prooftree}
\end{equation}

As the valid proofs~\ref{eq:equivalence example part 1} and \ref{eq:equivalence example part 2}
could be constructed, the claim holds. \qed
\end{example}

\begin{example}[Exercise 2.5.4]\label{exa:2.5.4}
    We wish to show that \(\lnot\forall y F(y)\equiv\exists x\lnot F(x)\).
    As per usual with equivalences, we need to do the proof in two parts.
    First we show that \(\seq\lnot\forall y F(y)\limplies\exists x\lnot F(x)\):

\begin{proof}
    \begin{equation}\label{exe:2.5.4:eq:1}
        \begin{prooftree}
            \hypo{F(a)\seq F(a)}
            \infer1[\eqref{eq:lnot:right}]{\seq F(a), \lnot F(a)}
            \infer1[\eqref{eq:exists:right}]{\seq F(a), \exists x\lnot F(x)}
            \infer1[\eqref{eq:exchange:right}]{\seq \exists x\lnot F(x), F(a)}
            \infer1[\eqref{eq:forall:right}]{\seq \exists x\lnot F(x), \forall y F(y)}
            \infer1[\eqref{eq:lnot:left}]{\lnot\forall y F(y)\seq \exists x\lnot F(x)}
            \infer1[\eqref{eq:limplies:right}]{\seq\lnot\forall y F(y)\limplies\exists x\lnot F(x)}
        \end{prooftree}
    \end{equation}
\end{proof}

    Then we prove the statement in the opposite direction,
    as in \(\seq\exists x\lnot F(x)\limplies\lnot\forall y F(y)\):

    \begin{proof}
        \begin{equation}\label{exe:2.5.4:eq:2}
            \begin{prooftree}
                \hypo{F(a)\seq F(a)}
                \infer1[\eqref{eq:forall:left}]{\forall y F(y)\seq F(a)}
                \infer1[\eqref{eq:lnot:right}]{\seq F(a), \lnot\forall y F(y)}
                \infer1[\eqref{eq:exchange:right}]{\seq\lnot\forall y F(y), F(a)}
                \infer1[\eqref{eq:lnot:left}]{\lnot F(a)\seq\lnot\forall yF(y)}
                \infer1[\eqref{eq:exists:left}]{\exists x\lnot F(x)\seq\lnot\forall yF(y)}
                \infer1[\eqref{eq:limplies:right}]{\seq\exists x\lnot F(x)\limplies\lnot\forall y F(y)}
            \end{prooftree}
        \end{equation}
    \end{proof}

    In light of proofs~\eqref{exe:2.5.4:eq:1} and \eqref{exe:2.5.4:eq:2}, the claim holds.
\end{example}

\begin{example}[Double negation in \LK]\label{exa:double negation in LK}
    We prove the law of double negation in \LK{} as follows:
\begin{equation*}
    \begin{prooftree}
        \hypo{A\seq A}
        \infer1[\eqref{eq:lnot:left}]{\lnot A, A\seq}
        \infer1[\eqref{eq:lnot:right}]{A\seq\lnot\lnot A}
        \infer1[\eqref{eq:limplies:right}]{\seq A\limplies\lnot\lnot A}
        \hypo{A\seq A}
        \infer1[\eqref{eq:lnot:right}]{\seq A, \lnot A}
        \infer1[\eqref{eq:lnot:left}]{\lnot\lnot A\seq A}
        \infer1[\eqref{eq:limplies:right}]{\seq\lnot\lnot A\limplies A}
        \infer2[\eqref{eq:land:right}]{\seq\arcs{A\limplies\lnot\lnot A}\land\arcs{\lnot\lnot A\limplies A}}
    \end{prooftree}
\end{equation*}
    Note that this also means that \(A\equiv A\).
    Note also that this proof is not possible in \LJ.
\end{example}

\begin{example}[A cut free proof]\label{exa:cut free proof}
    A cut free proof of \(\forall x A(x)\limplies B \seq \exists x \arcs{A(x)\limplies B}\),
    where \(A(a)\) and \(B\) are atomic and disjoint goes as follows:
    \begin{equation*}
        \begin{prooftree}
            \hypo{A(a)\seq A(a)}
            \infer1[\eqref{eq:weakening:right}]{A(a)\seq A(a), \forall x A(x)}
            \hypo{B\seq B}
            \infer2[\eqref{eq:limplies:left}]{A(a), \forall xA(x)\limplies B\seq A(a), B}
            \infer1[\eqref{eq:exchange:left}]{\forall xA(x)\limplies B, A(a)\seq, A(a), B}
            \infer1[\eqref{eq:limplies:right}]{\forall xA(x)\limplies B\seq A(a), A(a)\limplies B}
            \infer1[\eqref{eq:exists:right}]{\forall xA(x)\limplies B\seq A(a), \exists x\arcs{A(x)\limplies B}}
            \infer1[\eqref{eq:exchange:right}]{\forall xA(x)\limplies B\seq\exists x\arcs{A(x)\limplies B}, A(a)}
            \infer1[\eqref{eq:forall:right}]{\forall xA(x)\limplies B\seq\exists x\arcs{A(x)\limplies B}, \forall x A(x)}
            \hypo{B\seq B}
            \infer1[\eqref{eq:weakening:left}]{A(a), B\seq B}
            \infer1[\eqref{eq:limplies:right}]{B\seq A(a)\limplies B}
            \infer1[\eqref{eq:exists:right}]{B\seq\exists x\arcs{A(x)\limplies B}}
            \infer2[\eqref{eq:limplies:left}]{\forall x A(x)\limplies B, \forall x A(x)\limplies B \seq \exists x \arcs{A(x)\limplies B}, \exists x \arcs{A(x)\limplies B}}
            \infer1[\eqref{eq:contraction:left}]{\forall x A(x)\limplies B \seq \exists x \arcs{A(x)\limplies B}, \exists x \arcs{A(x)\limplies B}}
            \infer1[\eqref{eq:contraction:right}]{\forall x A(x)\limplies B \seq \exists x \arcs{A(x)\limplies B}}
            \infer1[\eqref{eq:limplies:right}]{\seq\arcs{\forall x A(x)\limplies B}\limplies\arcs{\exists x \arcs{A(x)\limplies B}}}
        \end{prooftree}
    \end{equation*}
\end{example}

\paragraph{Note:}
We also accept the inferences
\begin{equation*}
    \begin{prooftree}
        \hypo{T\seq \Delta}
        \infer1[\(\set{\eqref{eq:land:left},\eqref{eq:lor:left},\eqref{eq:limplies:left}}\)]{A\set{\land,\lor,\limplies } B, T\seq \Delta}
    \end{prooftree}
    \quad\text{and}\quad
    \begin{prooftree}
        \hypo{T\seq \Delta, A}
        \infer1[\eqref{eq:lor:right}]{T\seq\Delta, \arcs{B\land C}\lor A}
    \end{prooftree}
\end{equation*}

\begin{proposition}\label{prop:if provable, then atomic}
    If \(T\seq\Delta\) is provable without~\eqref{eq:cut},
    then we may assume that all of the related axioms are atomic,
    and that the proof is without cut.
\end{proposition}

\begin{proof}
    Omitted. See Takeuti's book~\cite[14-17]{Takeuti-1987} for the details.
\end{proof}

\begin{definition}[Alphabetical variants]\label{def:alphabetical variants}
    The formulae \(A\) and \(B\) are \emph{alphabetical variants}
    if they differ only in the names of their bound variables.
    This is denoted with \(A\varof{B}\). This is an equivalence relation.
\end{definition}

For example, it should be easily seen that
\begin{equation*}
    \exists x\arcs{f(x) = g(x)} \varof\exists y\arcs{f(y) = g(y)}\,.
\end{equation*}
It can be shown that if \(A\sim B\), then \(\seq\arcs{A\limplies B}\land\arcs{B\limplies A}\).
In other words, if \(A\varof B\) then \(A\equiv B\).



\subsection{Exercises}

\begin{exercise}[Exercise 2.5.2]\label{exe:2.5.2}
    Prove \(A\limplies B \equiv \lnot A\lor B\).
\end{exercise}

\begin{exercise}[Exercise 2.5.3]\label{exe:2.5.3}
    Prove \(\exists x F(x) \equiv\lnot\forall y\lnot F(y)\).
\end{exercise}
\begin{exercise}[Exercise 2.5.4]\label{exe:2.5.4}
    Prove \(\lnot\forall yF(y) \equiv\exists x\lnot F(x)\).
\end{exercise}
\begin{exercise}[Exercise 2.5.5]\label{exe:2.5.5}
    Prove \(\lnot\arcs{A\land B} \equiv \lnot A\lor\lnot B\).
\end{exercise}
\begin{exercise}[Exercise 2.6.1]\label{exe:2.6.1}
    Prove \(\exists x\arcs{A\limplies B(x)}\equiv A\limplies\exists xB(x)\).
\end{exercise}
\begin{exercise}[Exercise 2.6.2]\label{exe:2.6.2}
    Prove \(\exists x\arcs{A(x)\limplies B}\equiv\forall xA(x)\limplies B\),
    where \(B\) does not contain \(x\).
\end{exercise}
\begin{exercise}[Exercise 2.6.3]\label{exe:2.6.3}
    Prove \(\exists x\arcs{A(x)\limplies B(x)}\equiv\forall xA(x)\limplies\exists xB(x)\).
\end{exercise}
\begin{exercise}[Exercise 2.6.4]\label{exe:2.6.4}
    Prove \(\lnot A\limplies B\seq\lnot B\limplies A\)
\end{exercise}
\begin{exercise}[Exercise 2.6.5]\label{exe:2.6.5}
    Prove \(\lnot A\limplies\lnot B\seq B\limplies A\)
\end{exercise}


\section{Intuitionistic predicate calculus \LJ}

Intuitionistic logic \LJ{} is very similar to classical logic \LK,
except that in its inference rules, the succedents \(\Delta\) of upper sequents \(T\seq\Delta\)
may only contain \emph{at most} a single formula. For example, the inferences
\begin{equation*}
    \begin{prooftree}
        \hypo{T\seq D, D}
        \infer1[\eqref{eq:contraction:right}]{T\seq D}
    \end{prooftree}
    \quad\text{and}\quad
    \begin{prooftree}
        \hypo{T\seq C, D}
        \infer1[\eqref{eq:exchange:right}]{T\seq D, C}
    \end{prooftree}
\end{equation*}
are \emph{not} allowed in \LJ{}. This of course means, that \LJ{} is a true subset of \LK:
some proofs~\(P\) that are in \LK{} are not in \LJ{}.

\subsection{Examples}

\begin{example}[A proof in \LJ]\label{exa:a proof in LJ}
    \begin{equation*}
        \begin{prooftree}
            \hypo{A\seq A}
            \infer1[\eqref{eq:land:left}]{A\land\lnot A&\seq A}
            \infer1[\eqref{eq:lnot:left}]{\lnot A, A\land\lnot A&\seq}
            \infer1[\eqref{eq:land:left}]{A\land\lnot A, A\land\lnot A&\seq}
            \infer1[\eqref{eq:contraction:left}]{A\land\lnot A&\seq}
            \infer1[\eqref{eq:lnot:right}]{&\seq \lnot\arcs{A\land\lnot A}}
        \end{prooftree}
    \end{equation*}
\end{example}

\begin{example}[Another proof in \LJ]\label{exa:another proof in LJ}
    Prove that \(\lnot\exists xF(x)\seq\forall yF(y)\).
    \begin{equation*}
        \begin{prooftree}
            \hypo{F(a)&\seq F(a)}
            \infer1[\eqref{eq:exists:right}]{F(a)&\seq\exists xF(x)}
            \infer1[\eqref{eq:lnot:left}]{\lnot\exists xF(x), F(a)&\seq}
            \infer1[\eqref{eq:exchange:left}]{F(a), \lnot\exists xF(x)&\seq}
            \infer1[\eqref{eq:lnot:right}]{\lnot\exists xF(x)&\seq \lnot F(a)}
            \infer1[\eqref{eq:forall:right}]{\lnot\exists xF(x)&\seq \forall x\lnot F(x)}
            \infer1[\eqref{eq:limplies:right}]{&\seq \lnot\exists xF(x)\limplies\forall x\lnot F(x)}
        \end{prooftree}
    \end{equation*}
\end{example}

\subsection{Exercises}

\begin{exercise}[3.9.1]\label{exe:3.9.1}
    Prove in \LJ{:} \(\lnot A\lor B\seq A\limplies B\).
\end{exercise}
\begin{exercise}[3.9.2]\label{exe:3.9.2}
    Prove in \LJ{:} \(\exists xF(x)\seq\lnot\forall y\lnot F(y)\).
\end{exercise}
\begin{exercise}[3.9.3]\label{exe:3.9.3}
    Prove in \LJ{:} \(A\land B\seq A\).
\end{exercise}
\begin{exercise}[3.9.4]\label{exe:3.9.4}
    Prove in \LJ{:} \(A\seq A\lor B\).
\end{exercise}
\begin{exercise}[3.9.5]\label{exe:3.9.5}
    Prove in \LJ{:} \(\lnot A\lor\lnot B\seq \lnot\arcs{A\land B}\).
\end{exercise}
\begin{exercise}[3.9.6]\label{exe:3.9.6}
    Prove in \LJ{:} \(\lnot\arcs{A\lor B}\equiv\lnot A\land\lnot B\).
\end{exercise}
\begin{exercise}[3.9.7]\label{exe:3.9.7}
    Prove in \LJ{:} \(\arcs{A\lor C}\land\arcs{B\lor C} \equiv \arcs{A\land B}\lor C\).
\end{exercise}
\begin{exercise}[3.9.8]\label{exe:3.9.8}
    Prove in \LJ{:} \(\exists x\lnot F(x)\seq\lnot\forall xF(x)\).
\end{exercise}
\begin{exercise}[3.9.9]\label{exe:3.9.9}
    Prove in \LJ{:} \(\forall x\arcs{F(x)\land G(x)}\equiv\forall xF(x)\land\forall xG(x)\).
\end{exercise}
\begin{exercise}[3.9.10]\label{exe:3.9.10}
    Prove in \LJ{:} \(A\limplies\lnot B\seq B\limplies\lnot A\).
\end{exercise}
\begin{exercise}[3.9.11]\label{exe:3.9.11}
    Prove in \LJ{:} \(\exists x\arcs{A\limplies B(x)}\seq A\limplies\exists xB(x)\).
\end{exercise}
\begin{exercise}[3.9.12]\label{exe:3.9.12}
    Prove in \LJ{:} \(\exists x\arcs{A(x)\limplies B}\seq\forall xA(x)\limplies B\).
\end{exercise}
\begin{exercise}[3.9.13]\label{exe:3.9.13}
    Prove in \LJ{:} \(\exists x\arcs{A(x)\limplies B(x)}\seq\forall xA(x)\limplies\exists xB(x)\).
\end{exercise}
\begin{exercise}[3.10.1]\label{exe:3.10.1}
    Prove in \LJ{:} \(\lnot\lnot\arcs{A\limplies B}, A\seq\lnot\lnot B\).
\end{exercise}
\begin{exercise}[3.10.2]\label{exe:3.10.2}
    Prove in \LJ{:} \(\lnot\lnot B\limplies B, \lnot\lnot\arcs{A\limplies B}\seq A\limplies B\).
\end{exercise}
\begin{exercise}[3.10.3]\label{exe:3.10.3}
    Prove in \LJ{:} \(\lnot\lnot\lnot A\equiv\lnot A\).
\end{exercise}


\begin{example}[3.11]\label{exa:3.11}
    Let \(R\) be an atomic formula, and let \(J'\) be obtained from \LJ{} by adding the sequent~\(\lnot\lnot R\seq R\) as an axiom.
    Also let \(A\) be a formula which does not contain the symbols \(\lor\) or \(\exists\).
    Then \(\lnot\lnot A\seq A\) is \(J'\)-provable.

    \begin{proof}
        Let \(n\) be the number of logical symbols in \(A\).
        We prove the claim by structurally inducing on \(n\):
        \begin{enumerate}
            \item
                If \(n = 0\), then \(A = R\), as \(R\) is an atomic formula.
                By assumption \(\lnot\lnot R \seq R\), so the claim holds for the base case.
            \item
                Now assume that the inductive hypothesis \(\lnot\lnot A\seq A, \lnot\lnot B\seq B \in J''\) holds for the formulae \(A\) and \(B\),
                with at most \(n = k\) logical symbols. We add one additional allowed symbol to these formulae one at at time to close the induction:
                \begin{description}
                    \item[\(\lnot\)]
                        By exercise~\ref{exe:3.10.3}, \(\lnot\lnot\arcs{\lnot A}\seq\lnot A\) can be proven.
                    \item[\(\limplies\)]
                        We show that the claim \(\lnot\lnot\arcs{A\limplies B} \seq A\limplies B\) can be proven.
                        In the proof, IH refers to the inductive hypothesis:
                        \begin{equation*}
                            \begin{prooftree}
                                \hypo{B\seq B}
                                \ellipsis{\IH}{\lnot\lnot B\seq B}
                                \infer1[\eqref{eq:limplies:right}]{\seq\lnot\lnot B\limplies B}
                                \hypo{A\seq A}
                                \hypo{B\seq B}
                                \infer2{}
                                \ellipsis{(Exercise~\ref{exe:3.10.2})}{\lnot\lnot\arcs{A\limplies B} \seq A\limplies B}
                                \infer2[\eqref{eq:cut}]{\lnot\lnot\arcs{A\limplies B} \seq A\limplies B}
                            \end{prooftree}
                        \end{equation*}
                    \item[\(\land\)]
                        Again, we show that the claim \(\lnot\lnot\arcs{A\land B} \seq A\land B\) can be proven.
                        Also again, IH refers to the inductive hypothesis:
                        \tiny
                        \begin{equation*}
                            \begin{prooftree}
                                \hypo{A&\seq A}
                                \infer1[\eqref{eq:land:left}]{A\land B&\seq A}
                                \infer1[\eqref{eq:lnot:left}]{\lnot A, A\land B&\seq}
                                \infer1[\eqref{eq:exchange:left}]{A\land B,\lnot A&\seq}
                                \infer1[\eqref{eq:lnot:right}]{\lnot A&\seq\lnot\arcs{ A\land B}}
                                \infer1[\eqref{eq:lnot:left}]{\lnot\lnot\arcs{A\land B}, \lnot A&\seq}
                                \infer1[\eqref{eq:exchange:left}]{\lnot A, \lnot\lnot\arcs{A\land B}&\seq}
                                \infer1[\eqref{eq:lnot:right}]{\lnot\lnot\arcs{A\land B}&\seq\lnot\lnot A}
                                \hypo{A\seq A}
                                \ellipsis{\IH}{\lnot\lnot A\seq A}
                                \infer2[\eqref{eq:cut}]{\lnot\lnot\arcs{A\land B}\seq A}
                                \hypo{B&\seq B}
                                \infer1[\eqref{eq:land:left}]{A\land B&\seq B}
                                \infer1[\eqref{eq:lnot:left}]{\lnot B, A\land B&\seq}
                                \infer1[\eqref{eq:exchange:left}]{A\land B,\lnot B&\seq}
                                \infer1[\eqref{eq:lnot:right}]{\lnot B&\seq\lnot\arcs{ A\land B}}
                                \infer1[\eqref{eq:lnot:left}]{\lnot\lnot\arcs{A\land B}, \lnot B&\seq}
                                \infer1[\eqref{eq:exchange:left}]{\lnot B, \lnot\lnot\arcs{A\land B}&\seq}
                                \infer1[\eqref{eq:lnot:right}]{\lnot\lnot\arcs{A\land B}&\seq\lnot\lnot B}
                                \hypo{B\seq B}
                                \ellipsis{\IH}{\lnot\lnot B\seq B}
                                \infer2[\eqref{eq:cut}]{\lnot\lnot\arcs{A\land B}\seq B}
                                \infer2[\eqref{eq:land:right}]{\lnot\lnot\arcs{A\land B} \seq A\land B}
                            \end{prooftree}
                        \end{equation*}
                        \normalsize
                    \item[\(\forall\)]
                        Lastly, we show that the claim \(\lnot\lnot\forall xA(x) \seq \forall xA(x)\) is a theorem:
                        \tiny
                        \begin{equation*}
                            \begin{prooftree}
                                \hypo{A(a)\seq A(a)}
                                \infer1[\eqref{eq:forall:left}]{\forall xA(x)&\seq A(a)}
                                \infer1[\eqref{eq:lnot:left}]{\lnot A(a),\forall xA(x)&\seq}
                                \infer1[\eqref{eq:exchange:left}]{\forall xA(x),\lnot A(a)&\seq}
                                \infer1[\eqref{eq:lnot:right}]{\lnot A(a)&\seq\lnot\forall xA(x)}
                                \infer1[\eqref{eq:lnot:left}]{\lnot\lnot\forall xA(x), \lnot A(a)&\seq}
                                \infer1[\eqref{eq:exchange:left}]{\lnot A(a), \lnot\lnot\forall xA(x)&\seq}
                                \infer1[\eqref{eq:lnot:right}]{\lnot\lnot\forall xA(x)&\seq \lnot\lnot A(a)}
                                \hypo{A(a)\seq A(a)}
                                \ellipsis{\IH}{\lnot\lnot A(a)\seq A(a)}
                                \infer2[\eqref{eq:cut}]{\lnot\lnot\forall xA(x) \seq A(a)}
                                \infer1[\eqref{eq:forall:right}]{\lnot\lnot\forall xA(x) \seq \forall xA(x)}
                            \end{prooftree}
                        \end{equation*}
                        \normalsize
                \end{description}
        \end{enumerate}
        By the inductive principle, we are then done.
    \end{proof}
\end{example}

\section{The relation between \LK{} and \LJ}

Our last goal in discussinga what the specific connection between \LK{} and \LJ{} is.
For this purpose we declare the notational definitions in definition~\ref{def:starred formulae in LK}.

\begin{definition}[Starred formula variants in \LK]\label{def:starred formulae in LK}
    Table~\ref{tab:starred variants in LK} defines a recursive mapping,
    between formulae \(A\) and their starred variants \(A^\ast\).

    \begin{tabenv}{Mapping between fomulae \(A\) and their starred transformations \(A^\ast\).}%
        \label{tab:starred variants in LK}
        \begin{tabular}{C|C}
            \toprule
            \text{Formula } A & \text{Starred formula } A^\ast \\
            \midrule
            \text{atomic } A & \lnot\lnot A \\
            \lnot B & \lnot\arcs{B^\ast} \\
            B\land C & B^\ast\land C^\ast \\
            B\lor C & \lnot\arcs{\lnot B^\ast\land\lnot C^\ast} \\
            \forall xB(x) & \forall xB^\ast(x) \\
            \exists xB(x) & \lnot\forall xB^\ast(x) \\
            B\limplies C & B^\ast\limplies C^\ast \\
            \bottomrule
        \end{tabular}
    \end{tabenv}
\end{definition}

With definition~\ref{def:starred formulae in LK} in place, we can proceed to make the propositions that follow.

\begin{proposition}[Claim 1]\label{prop:claim 1 in LK vs LJ}
    \(A\equiv A^\ast\) for all \(A\) in \LK.
\end{proposition}

\begin{proof}
    Just like with example~\ref{exa:3.11}, the proof can be achieved via induction over the number of logical symbols in \(A\).
    \begin{enumerate}
        \item
            For the base case, we observe that by the double negation law proved in example~\ref{exa:double negation in LK},
            \(A\equiv\lnot\lnot A = A^\ast\).
        \item
            For the induction hypothesis \IH, we assume that for the formulae \(B\) and \(C\),
            \(B\equiv B^\ast\) and \(C\equiv C^\ast\). To close the hypothesis,
            we add one allowed logical symbol at a time to \(A\) formed from \(B\) and/or \(C\),
            as necessary:
            \begin{description}
                \item[\lnot]
                    If \(A = \lnot B\), then \(A^\ast = \lnot\lnot\lnot B\),
                    and \(A\equiv A^\ast\) by exercise~\ref{exe:3.10.3}.
                \item[\land]
                    If \(A = B\land C\), then \(A^\ast = B^\ast\land C^\ast\) and we argue as follows:
                    \begin{equation*}
                        \begin{prooftree}
                            \hypo{B\seq B}
                            \ellipsis{\IH}{B\seq B^\ast}
                            \infer1{B\land C\seq B^\ast}
                            \hypo{C\seq C}
                            \ellipsis{\IH}{C\seq C^\ast}
                            \infer1{B\land C\seq C^\ast}
                            \infer2{B\land C \seq B^\ast\land C^\ast}
                            \hypo{B\seq B}
                            \ellipsis{\IH}{B^\ast\seq B}
                            \infer1{B^\ast\land C^\ast\seq B}
                            \hypo{C\seq C}
                            \ellipsis{\IH}{C^\ast\seq C}
                            \infer1{B^\ast\land C^\ast\seq C}
                            \infer2{B^\ast\land C^\ast \seq B\land C}
                            \infer2{B\land C \equiv B^\ast\land C^\ast}
                        \end{prooftree}
                    \end{equation*}
                \item[\lor]
                    \begin{exercise}[3.12.1 part iii]\label{exe:3.12.1.iii}
                        This is left as an exercise. Show that \(A \equiv A^\ast\), when \(A=B\lor C\).
                    \end{exercise}
                \item[\forall]
                    If \(A = \forall xB(x)\), then \(A^\ast = \forall xB^\ast(x)\).
                    Now we can argue as follows:
                    \begin{equation*}
                        \begin{prooftree}
                            \hypo{B(a\seq B(a))}
                            \ellipsis{\IH}{B(a)\seq B^\ast(a)}
                            \infer1[\eqref{eq:forall:left}]{\forall xB(x)\seq B^\ast(a)}
                            \infer1[\eqref{eq:forall:right}]{\forall xB(x)\seq\forall xB^\ast(x)}
                            \hypo{B(a\seq B(a))}
                            \ellipsis{identical to left branch}{\forall xB(x)\seq\forall xB^\ast(x)}
                            \infer2{\forall xB^\ast(x)\equiv\forall xB(x)}
                        \end{prooftree}
                    \end{equation*}
                \item[\limplies]
                    If \(A = B\limplies C\), then \(A^\ast = B^\ast\limplies C^\ast\).
                    The argument is then
                    \begin{equation*}
                        \begin{prooftree}
                            \hypo{B\seq B}
                            \ellipsis{\IH}{B^\ast\seq B}
                            \hypo{C\seq C}
                            \ellipsis{\IH}{C\seq C^\ast}
                            \infer[left label=\eqref{eq:limplies:left}]2{B\limplies C, B^\ast\seq C^\ast}
                            \infer[left label=\eqref{eq:exchange:left}]1{B^\ast, B\limplies C\seq C^\ast}
                            \infer[left label=\eqref{eq:limplies:right}]1{B\limplies C\seq B^\ast\limplies C^\ast}
                            \hypo{B\seq B}
                            \ellipsis{\IH}{B\seq B^\ast}
                            \hypo{C\seq C}
                            \ellipsis{\IH}{C^\ast\seq C}
                            \infer[left label=\eqref{eq:limplies:left}]2{B^\ast\limplies C^\ast, B\seq C}
                            \infer[left label=\eqref{eq:exchange:left}]1{B, B^\ast\limplies C^\ast\seq C}
                            \infer[left label=\eqref{eq:limplies:right}]1{B^\ast\limplies C^\ast\seq B\limplies C}
                            \infer2{B^\ast\limplies C^\ast\equiv B\limplies C}
                        \end{prooftree}
                    \end{equation*}
                \item[\exists]
                    \begin{exercise}[3.12.1 part vi]\label{exe:3.12.1.vi}
                        This is left as an exercise. Show that if \(A = \exists xB(x)\), then \(A\equiv A^\ast\).
                    \end{exercise}
            \end{description}
    \end{enumerate}
    By the inductive principle, we are then done.
\end{proof}

\begin{proposition}[Claim 2]\label{prop:claim 2 in LK vs LJ}
    If \(S\) is the sequent \(A_1,\ldots,A_m\seq B_1,\ldots,B_n\),
    then we may define the sequent \(S'\coloneqq A_1^\ast,\ldots,A_m^\ast,\lnot B_1^\ast,\ldots,\lnot B^\ast_n\seq\).
    Now \(S\) is \LK-provable, if and only if \(S'\) is \LK-provable.
\end{proposition}

\begin{proof}[Sketch of proof]
    We start by proving that if \(S\) is provable (meaning we can use it as a premise of a proof),
    then \(S'\) is also provable:
    \tiny

    \begin{equation*}
        \begin{prooftree}
            \hypo{A_2^\ast\seq\lnot\lnot A_2^\ast}
            \hypo{A_1^\ast\seq\lnot\lnot A_1^\ast}
            % Tallest branch
            \hypo{\overbrace{A_1,\ldots,A_m\seq B_1,\ldots,B_n}^{S}}
            \hypo{B_n\seq B_n^\ast}
            \infer2[\eqref{eq:cut}]{A_1,\ldots,A_m\seq B_1,\ldots,B_{n-1}, B_n^\ast}
            \infer1[\eqref{eq:exchange:right}]{A_1,\ldots,A_m\seq B_1,\ldots,B_n^\ast, B_{n-1}}
            \hypo{B_{n-1}\seq B_{n-1}^\ast}
            \infer2[\eqref{eq:cut}]{A_1,\ldots,A_m\seq B_1,\ldots,B_n^\ast, B_{n-1}^\ast}
            \infer[double]1[\eqref{eq:exchange:right}s and \eqref{eq:cut}s]{A_1,\ldots,A_m\seq B_n^\ast,\ldots,B_1^\ast}
            \infer[double]1[\eqref{eq:lnot:left}s and \eqref{eq:exchange:left}s]{A_1,\ldots,A_m, \lnot B_n^\ast,\ldots,\lnot B_1^\ast\seq}
            \infer[double]1[\eqref{eq:lnot:right}s and \eqref{eq:exchange:right}s]{\lnot B_n^\ast,\ldots,\lnot B_1^\ast\seq\lnot A_m,\ldots,\lnot A_1}
            \hypo{\lnot A_1\seq \lnot A^\ast_1}
            \infer[left label=\eqref{eq:cut} and \eqref{eq:exchange:right}]2{\lnot B_n^\ast,\ldots,\lnot B_1^\ast\seq\lnot A_m,\ldots,\lnot A_1^\ast,\lnot A_2}
            \hypo{\lnot A_2\seq\lnot A_2^\ast}
            \infer[double,left label=\eqref{eq:cut}s and \eqref{eq:exchange:right}s]2{\lnot B_n^\ast,\ldots,\lnot B_1^\ast\seq\lnot A_1^\ast,\ldots,\lnot A_m^\ast}
            \infer[double,left label=\eqref{eq:lnot:right}s and \eqref{eq:exchange:left}s]1{\lnot\lnot A_1^\ast,\ldots,\lnot\lnot A_m^\ast, \lnot B_n^\ast,\ldots,\lnot B_1^\ast\seq}
            \infer2[\eqref{eq:cut}]{\lnot\lnot A_2^\ast,\ldots,\lnot\lnot A_m^\ast, \lnot B_n^\ast,\ldots,\lnot B_1^\ast\seq}
            \infer[double]2[multiple \eqref{eq:cut}s]{A_1^\ast,\ldots,A_m^\ast, \lnot B_n^\ast,\ldots,\lnot B_1^\ast\seq}
        \end{prooftree}
    \end{equation*}
    \normalsize
    On the last line, \(A_1^\ast,\ldots,A_m^\ast, \lnot B_n^\ast,\ldots,\lnot B_1^\ast\seq = S'\).
    The proof in the other direction is performed in a similar manner.
\end{proof}

\begin{proposition}[Claim 3]\label{prop:claim 3 in LK vs LJ}
    \(A^\ast\equiv\lnot\lnot A^\ast\) in \LJ.
\end{proposition}

\begin{proof}
    In \LJ{}, the proof
    \begin{equation*}
        \begin{prooftree}
            \hypo{A\seq A}
            \infer1{\lnot A, A\seq}
            \infer1{A\seq\lnot\lnot A}
        \end{prooftree}
    \end{equation*}
    is valid for all formulae \(A\). This means it must also be true for \(A^\ast\seq\lnot\lnot A^\ast\) as well.

    By example~\ref{exa:3.11}, \(\lnot\lnot A\seq A\) in \LJ{'}.
    Also, by exercise~\ref{exe:3.10.3} we have \(\lnot\lnot\lnot\arcs{\lnot R}\equiv\lnot\lnot R\) in \LJ{},
    for all atomic formula \(R\).
    As the formula \(A^\ast\) does not contain the symbols \(\lor\) or \(\exists\) by assumption,
    we can conlcude that \(\lnot\lnot A^\ast\seq A\).
    Therefore \(A^\ast\equiv \lnot\lnot A^\ast\) for all \(A\) in \LJ{}.
\end{proof}

\begin{proposition}[Claim 4]\label{prop:claim 4 in LK vs LJ}
    If the sequent \(S\) of proposition~\ref{prop:claim 2 in LK vs LJ} is provable in \LK,
    then the sequence formula \(S'\) defined in the same proposition is provable in \LJ.
\end{proposition}

\begin{proof}[A sketch of a proof]
    Let \(n\) refer to the length of a proof, as in the number of inferences of a proof in \LK{}.
    We induce on this \(n\):
    \begin{enumerate}
        \item
            For the base case \(n=0\), all proofs \(P = A\seq A\) in \LK{}.
            The in \LJ{} we have \(A^\ast\seq A^\ast\) and
            \begin{equation*}
                \begin{prooftree}
                    \hypo{A^\ast\seq A^\ast}
                    \infer1[\eqref{eq:lnot:left}]{\lnot A^\ast, A^\ast\seq}
                    \infer1[\eqref{eq:exchange:left}]{A^\ast, \lnot A^\ast\seq}
                \end{prooftree}
            \end{equation*}
        \item
            For the inductive hypothesis \IH, let the claim in \LK{} hold for a proof of length \(n \leq k\).
            Note that based on the number of inference rules in definition~\ref{def:inference rules in LK},
            the number of \emph{subcases} of \(n = k+1\) is \(21\). Let us cover some of them:
            \begin{enumerate}
                \item
                    In \LK{} we have
                    \begin{equation*}
                        \begin{prooftree}
                            \hypo{A_1,\ldots,A_m\seq B_1,\ldots,B_n}
                            \infer1[\eqref{eq:weakening:left}]{A, A_1,\ldots,A_m\seq B_1,\ldots B_1,\ldots,B_n}
                        \end{prooftree}
                    \end{equation*}
                    whereas in \LJ{} this means that
                    \begin{equation*}
                        \begin{prooftree}
                            \hypo{A_1^\ast,\ldots, A_m^\ast, \lnot B_1^\ast,\ldots,\lnot B_n^\ast\seq}
                            \infer1[\eqref{eq:weakening:left}]{A^\ast, A_1,\ldots,A_m, \lnot B_1^\ast,\ldots,\lnot B_n^\ast\seq}
                        \end{prooftree}
                    \end{equation*}
                    by the inductive hypothesis.
                \item
                    In \LK{} we have
                    \begin{equation*}
                        \begin{prooftree}
                            \hypo{A_1,\ldots, A_i,A_j,\ldots,A_m\seq B_1,\ldots,B_n}
                            \infer1[\eqref{eq:exchange:left}]{A_1,\ldots, A_j,A_i,\ldots,A_m\seq B_1,\ldots,B_n}
                        \end{prooftree}
                    \end{equation*}
                    whereas in \LJ{} this means that
                    \begin{equation*}
                        \begin{prooftree}
                            \hypo{A_1^\ast,\ldots, A_i^\ast,A_j^\ast,\ldots,A_m^\ast, \lnot B_1^\ast,\ldots,\lnot B_n^\ast\seq}
                            \infer1[\eqref{eq:exchange:left}]{A_1^\ast,\ldots, A_j^\ast,A_i^\ast,\ldots,A_m^\ast, \lnot B_1^\ast,\ldots,\lnot B_n^\ast\seq},
                        \end{prooftree}\,,
                    \end{equation*}
                    again by the inductive hypothesis.
                \item
                    In \LK{}, by the rule \eqref{eq:cut} (the \(k+1\)st inference):
                    \begin{equation*}
                        \begin{prooftree}
                            \hypo{A_1,\ldots,A_i\seq B_1,\ldots,B_j, D}
                            \hypo{D, A_{i+1},\ldots,A_m\seq B_{j+1},\ldots,B_n, D}
                            \infer2{A_1,\ldots,A_m\seq B_1,\ldots,B_n}
                        \end{prooftree}
                    \end{equation*}
                    Then by the inductive hypothesis the inference
                    \tiny
                    \begin{equation*}
                        \begin{prooftree}
                            \hypo{A_1^\ast,\ldots,A_i^\ast, \lnot B_1^\ast,\ldots,\lnot B_j^\ast,\lnot D^\ast\seq}
                            \infer[double]1[multiple \eqref{eq:exchange:left}]{\lnot D^\ast, A_1^\ast,\ldots,A_i^\ast, \lnot B_1^\ast,\ldots,\lnot B_j^\ast,\lnot D^\ast\seq}
                            \infer1[\eqref{eq:lnot:right}]{A_1^\ast,\ldots,A_i^\ast, \lnot B_1^\ast,\ldots,\lnot B_j^\ast,\lnot D^\ast\seq\lnot\lnot D^\ast}
                            \hypo{\text{proposition}~\ref{prop:claim 3 in LK vs LJ}}
                            \ellipsis{}{\lnot\lnot D^\ast\seq D^\ast}
                            \infer2[\eqref{eq:cut}]{A_1^\ast,\ldots,A_i^\ast, \lnot B_1^\ast,\ldots,\lnot B_j^\ast,\lnot D^\ast\seq D^\ast}
                            \hypo{\IH}
                            \ellipsis{}{D^\ast, A_{i+1}^\ast,\ldots,A_m^\ast, \lnot B_{j+1}^\ast,\ldots,\lnot B_n^\ast\seq}
                            \infer2[\eqref{eq:cut}]{A_1^\ast,\ldots,A_i^\ast, \lnot B_1^\ast,\ldots,\lnot B_j^\ast,\lnot D^\ast\seq D^\ast}
                            \infer1[\eqref{eq:cut}]{A_1^\ast,\ldots,A_i^\ast, \lnot B_1^\ast,\ldots,\lnot B_j^\ast,\lnot D^\ast\seq D^\ast}
                        \end{prooftree}
                    \end{equation*}
                    \normalsize
                \item
                    \begin{exercise}[3.12.4 part vii]\label{exe:3.12.4.vii}
                    This is left as an exercise.
                    The \(\arcs{k+1}\)st inference is by \eqref{eq:lnot:right} in \LK{}.
                    \end{exercise}
            \end{enumerate}
    \end{enumerate}
    By the inductive principle, we can now conclude that if
    \(A_1,\ldots,A_m\seq B_1,\ldots,B_n\) is in \LK{},
    then \(A_1^\ast,\ldots,A_m^\ast, \lnot B_1^\ast,\ldots,\lnot B_n^\ast\seq\) is in \LJ{}.
\end{proof}

The converse of proposition~\ref{prop:claim 4 in LK vs LJ} also holds,
but the proof is very long.
Lastly, we cover the most important fact related to this section.

\begin{theorem}[Connection between \LK{} and \LJ]\label{the:connection between LK and LJ}
    For all formulae \(A\), \(\seq A\) in \LK{}, if and only if \(\seq A^\ast\) in \LJ{}.
\end{theorem}

\begin{proof}
    A particular instance of exercise~\ref{exe:3.12.4.vii} is the following:
    if \(\seq A\) in \LJ{}, then \(\lnot A^\ast\seq\) in LJ{}.
    Therefore
    \begin{equation*}
        \begin{prooftree}
            \hypo{\seq\lnot A^\ast}
            \infer1{\seq\lnot\lnot A^\ast}
            \hypo{A\seq A}
            \ellipsis{Proposition~\ref{prop:claim 3 in LK vs LJ}}{A^\ast\seq A}
            \infer2[\eqref{eq:cut}]{\seq A^\ast}
        \end{prooftree}
    \end{equation*}

    Conversely, if \(\seq A^\ast\) in \LJ{} and by proposition~\ref{prop:claim 1 in LK vs LJ} we know that \(A\equiv A^\ast\) in \LK{},
    we also have \(A^\ast\seq A\) in \LK{}. Thus
    \begin{equation*}
        \begin{prooftree}
            \hypo{\seq A^\ast}
            \hypo{A^\ast\seq A}
            \infer2[\eqref{eq:cut}]{\seq A}
        \end{prooftree}\,.
        \qedhere
    \end{equation*}
\end{proof}



\section{Axiom systems}

\section{The Cut-elimination theorems of \LK}

\section{Consequences of Cut-elimination -- Consistency of \LK{} an \LJ}

\section{Completeness of \LK}

\subsection{Structures, assignments, interpretations and validity}

\subsection{Soundness of \LK}

\subsection{Reduction trees, completeness of \LK}

\section{Predicate calculus with equality}

\section{Peano arithmetic or PA}

\section{Primitive recursive functions PRF and prmitive recursive relations PRR}

\section{The relation between Peano arithmetic and natural numbers}

\section{Gödel numbering}

\section{Arithmetization in Peano arithmetic}

\section{Tarski's theorem}

\section{Gödel's incompleteness theorems}

{\raggedright%
\printbibliography[heading=bibintoc]%
}

\end{document}

